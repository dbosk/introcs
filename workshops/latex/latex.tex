% $Id$
% Author:	Daniel Bosk <daniel.bosk@miun.se>
\documentclass[handout]{beamer}
\usepackage[utf8]{inputenc}
\usepackage[T1]{fontenc}
\usepackage[swedish,english]{babel}
\usepackage{verbatim}
\usepackage{url}
\usepackage{natbib}
\usepackage{varioref,prettyref}
\usepackage{amsmath,amsthm,amssymb}
\usepackage[amssymb,noams,binary]{SIunits}
\usepackage{listings}
\usepackage{graphicx}
\usepackage[today,nofancy]{svninfo}
\usepackage[listings,varioref,prettyref]{miunmisc}
\setcitestyle{numbers,square}
\bibliographystyle{swealpha}

\svnInfo $Id$

\lstset{style=text}

\mode<presentation>
{
	\usetheme{Frankfurt}
	\setbeamercovered{transparent}
	\usecolortheme{seagull}
}

\title{%
  Övningar för \LaTeX
}
\author{Daniel Bosk\footnote{%
  Detta verk är tillgängliggjort under licensen Creative Commons 
  Erkännande-DelaLika 2.5 Sverige (CC BY-SA 2.5 SE).
	För att se en sammanfattning och kopia av licenstexten besök URL 
	\url{http://creativecommons.org/licenses/by-sa/2.5/se/}.
}}
\institute{%
  %Department of Information and Communication Systems (ICS),\\
  %Mid Sweden University, Sundsvall.
  Avdelningen för informations- och kommunikationssytem (IKS),\\
  Mittuniversitetet, Sundsvall.
}
\date{\svnId}

\pgfdeclareimage[height=0.65cm]{university-logo}{MU_logotyp_int_CMYK.pdf}
\logo{\pgfuseimage{university-logo}}

\AtBeginSection[]{%
	\begin{frame}<beamer>{Overview}
		\tableofcontents[currentsection]
	\end{frame}
}

\begin{document}

\begin{frame}
  \titlepage
\end{frame}

%\begin{frame}{Läsning}
%  % $Id$
Föreläsningen går igenom kursstruktur och organisation.
Den ger en översikt över undervisning och examination.
Den motsvarar således att läsa igenom allt kursmaterial och lite därtill.

Utöver detta täcks även kapitel 0 i \citetitle{Brookshear2012csa} 
\cite{Brookshear2012csa}.
Kapitlet introducerar ämnena datateknik och datalogi (datavetenskap).
Det ger även en historisk överblick av området vilket är bra för att förstå 
varför området är som det är och dess framtida utveckling.

%\end{frame}

\begin{frame}{Övningar}
  \begin{enumerate}
    \item Skapa ett enkelt dokument med article-klassen.
    \item Kompilera dokumentet med latexmk(1).
    \item Skapa en punktlista.
    \item Skapa en numrerad lista.
    \item Skapa en figur, hänvisa till den i texten.
    \item Skapa en tabell, hänvisa till den i texten.
    \item Referera till kursboken \citep{Brookshear2012csa}.
    \item Inkludera källkoden för dokumentet som en bilaga i dokumentet.
    \item Titta på källkoden för kursens projektlydelse.
  \end{enumerate}
\end{frame}


%%%%%%%%%%%%%%%%%%%%%%

\begin{frame}{References}
  \bibliography{literature}
\end{frame}

\end{document}

