% $Id$
% Author:	Daniel Bosk <daniel.bosk@miun.se>
\documentclass[handout]{beamer}
\usepackage[utf8]{inputenc}
\usepackage[T1]{fontenc}
\usepackage[swedish,english]{babel}
\usepackage{verbatim}
\usepackage{url}
\usepackage{natbib}
\usepackage{varioref,prettyref}
\usepackage{amsmath,amsthm,amssymb}
\usepackage[amssymb,noams,binary]{SIunits}
\usepackage{listings}
\usepackage{graphicx}
\usepackage[today,nofancy]{svninfo}
\usepackage[listings,varioref,prettyref]{miunmisc}
\setcitestyle{numbers,square}
\bibliographystyle{alpha}

\svnInfo $Id$

\lstset{style=text}

\mode<presentation>
{
	\usetheme{Frankfurt}
	\setbeamercovered{transparent}
	\usecolortheme{seagull}
}

\title{%
  Övningar för UNIX-lika skal
}
\author{Daniel Bosk\footnote{%
  Detta verk är tillgängliggjort under licensen Creative Commons 
  Erkännande-DelaLika 2.5 Sverige (CC BY-SA 2.5 SE).
	För att se en sammanfattning och kopia av licenstexten besök URL 
	\url{http://creativecommons.org/licenses/by-sa/2.5/se/}.
}}
\institute{%
  %Department of Information and Communication Systems (ICS),\\
  %Mid Sweden University, Sundsvall.
  Avdelningen för informations- och kommunikationssytem (IKS),\\
  Mittuniversitetet, Sundsvall.
}
\date{\svnId}

\pgfdeclareimage[height=0.65cm]{university-logo}{MU_logotyp_int_CMYK.pdf}
\logo{\pgfuseimage{university-logo}}

\AtBeginSection[]{%
	\begin{frame}<beamer>{Overview}
		\tableofcontents[currentsection]
	\end{frame}
}

\begin{document}

\begin{frame}
  \titlepage
\end{frame}

%\begin{frame}{Läsning}
%  % $Id$
Föreläsningen går igenom kursstruktur och organisation.
Den ger en översikt över undervisning och examination.
Den motsvarar således att läsa igenom allt kursmaterial och lite därtill.

Utöver detta täcks även kapitel 0 i \citetitle{Brookshear2012csa} 
\cite{Brookshear2012csa}.
Kapitlet introducerar ämnena datateknik och datalogi (datavetenskap).
Det ger även en historisk överblick av området vilket är bra för att förstå 
varför området är som det är och dess framtida utveckling.

%\end{frame}

\begin{frame}{Användbara verktyg}
	\begin{description}
		\item[echo(1)] display a line of text
		\item[test(1)] check file types and compare values
		\item[find(1)] search for files in a directory hierarchy
		\item[tr(1)] translate or delete characters
		\item[uniq(1)] report or omit repeated lines
		\item[sort(1)] sort lines of text files
		\item[wc(1)] print newline, word, and byte counts for each file
		\item[cut(1)] remove sections from each line of files
		\item[join(1)] join lines of two files on a common field
		\item[paste(1)] merge lines of files
		\item[xargs(1)] build and execute command lines from standard input
		\item[grep(1)] print lines matching a pattern
		\item[sed(1)] stream editor for filtering and transforming text
	\end{description}
\end{frame}

\begin{frame}{Övningar}
  \begin{enumerate}
    \item Titta på några olika skal: bash(1), ksh(1), sh(1), csh(1).
    \item Titta på manualsidorna för skalen.
    \item Vilket kommando kan användas för att hitta filer vid namn?
    \item Spara listan av filer i rotkatalogen till en fil.
    \item Vad gör kommandot yes(1)?
    \item Från listan av filer i rotkatalogen, filtrera ut alla filnamn som är 
      exakt fyra tecken långa.
    \item Vad innehåller miljövariablerna \term{HOME}, \term{PATH} och 
      \term{EDITOR}?
    \item Titta på \term{.profile}, \term{.kshrc} och \term{.bashrc}.
    \item Undersök \term{rm} och \term{libris}.
  \end{enumerate}
\end{frame}


%%%%%%%%%%%%%%%%%%%%%%

%\begin{frame}{References}
%  \bibliography{literature}
%\end{frame}

\end{document}

