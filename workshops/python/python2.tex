% $Id$
% Author:	Daniel Bosk <daniel.bosk@miun.se>
\documentclass[handout]{beamer}
\usepackage[utf8]{inputenc}
\usepackage[T1]{fontenc}
\usepackage[swedish,english]{babel}
\usepackage{verbatim}
\usepackage{url}
\usepackage{natbib}
\usepackage{varioref,prettyref}
\usepackage{amsmath,amsthm,amssymb}
\usepackage[amssymb,noams,binary]{SIunits}
\usepackage{listings}
\usepackage{graphicx}
\usepackage[today,nofancy]{svninfo}
\usepackage[listings,varioref,prettyref]{miunmisc}
\setcitestyle{numbers,square}
\bibliographystyle{alpha}

\svnInfo $Id$

\lstset{style=text}

\mode<presentation>
{
	\usetheme{Frankfurt}
	\setbeamercovered{transparent}
	\usecolortheme{seagull}
}

\title{%
  Övningar för Python, del 2
}
\author{Daniel Bosk\footnote{%
  Detta verk är tillgängliggjort under licensen Creative Commons 
  Erkännande-DelaLika 2.5 Sverige (CC BY-SA 2.5 SE).
	För att se en sammanfattning och kopia av licenstexten besök URL 
	\url{http://creativecommons.org/licenses/by-sa/2.5/se/}.
}}
\institute{%
  %Department of Information and Communication Systems (ICS),\\
  %Mid Sweden University, Sundsvall.
  Avdelningen för informations- och kommunikationssytem (IKS),\\
  Mittuniversitetet, Sundsvall.
}
\date{\svnId}

\pgfdeclareimage[height=0.65cm]{university-logo}{MU_logotyp_int_CMYK.pdf}
\logo{\pgfuseimage{university-logo}}

\AtBeginSection[]{%
	\begin{frame}<beamer>{Overview}
		\tableofcontents[currentsection]
	\end{frame}
}

\begin{document}

\begin{frame}
  \titlepage
\end{frame}

%\begin{frame}{Läsning}
%  Innan du påbörjar laborationen ska du ha läst kapitel 1, 2, 5 och 6.1--6.4 
i \cite{Brookshear2012csa} och avsnitt 2--4 i \cite{pythonkramaren1}.

%\end{frame}

\begin{frame}{Övningar}
  \begin{enumerate}
    \item Utöka speed1.py så att användaren kan ange enhet på inmatat data 
      istället för att måsta ange med en given enhet.

    \item Utöka programmet vidare med en slinga som låter användaren köra fler 
      än en konvertering per körning.

    \item Utöka programmet med en loggningsfunktion som skriver alla 
      konverteringar till en fil på disk.

  \end{enumerate}
\end{frame}

\begin{frame}[fragile,allowframebreaks]{Lösningsförslag}
  \lstinputlisting[style=code,language=python]{speed2.py}
\end{frame}


%%%%%%%%%%%%%%%%%%%%%%

%\begin{frame}{References}
%  \bibliography{literature}
%\end{frame}

\end{document}

