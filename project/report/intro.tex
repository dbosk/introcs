% $Id$
\section{Introduktion}
\label{ch:intro}
Denna studie har undersökt dataöverföringshastigheten hos olika individer 
användandes penna och post-it-lappar som överföringsmedium.

De höga överföringshastigheter hos individer som använder rosa post-it-lappar 
har alltid varit en fascination, varför är den högre hos dessa individer än de 
som använder gula eller oranga?


\subsection{Syfte}
\label{sec:aim}
Syftet med undersökningen har varit att utröna huruvida färgen hos 
post-it-lappar påverkar individen som använder dem för dataöverföring.


\subsection{Avgränsningar}
\label{sec:delimit}
Undersökningen tar enbart upp skillnaderna mellan rosa och gula post-it-lappar.
Andra fäger som gröna eller blå får vänta till en senare studie.

Vidare fokuserar undersökningen på målgruppen universitetsadjunkter som 
undervisar på program inriktade på kommunikation i nätverk.
Omfattningen skulle bli för stor att titta på fler användarkategorier.


\subsection{Frågeställning}
\label{sec:problemstatement}
Undersökningen syftar till att besvara följande frågor:
\begin{enumerate}
  \item Har färgen på post-it-lapparna någon inverkan på 
    dataöverföringshastigheten?
  \item Har det någon påverkan om personen är vänster- eller högerhänt?
  \item Hur skiljer entropin i överföringsprotokollen hos de olika individerna?
\end{enumerate}
