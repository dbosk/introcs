% $Id$
\section{Diskussion}
\label{sec:discussion}
Analysen antyder ett mycket intressant resultat som visar att deltagare 
L föredrar rosa post-it-lappar medan det verkar spela någon roll för deltagare 
D.
Eftersom att det är för få deltagare i undersökningen går det inte att dra 
några generella slutsatser gällande den första frågan i frågeställningen: ''Har 
färgen på post-it-lapparna någon inverkan på dataöverföringshastigheten?''
Det verkar finnas subjektiva skillnader, men dessa kan även bero på vad som 
serverades till lunch den veckan, hur mycket personen hade att göra och vilken 
typ av penna personen använde -- faktorer som står utanför undersökningen.

Eftersom att båda fösökspersonerna var högerhänta kunde frågan om huruvida 
detta påverkade inte besvaras.

Den ickenormerade entropin skiljer avsevärt mellan de två personerna.
Detta är svårt att avgöra varför utifrån våra resultat.
Det skulle kunna bero på att person L är mer mentalt nedbruten än person D, 
eller att de är båda lika mentalt nedbrutna men person D är mycket effektivare 
än person L.
Det skulle också kunna vara så att person D haft mer inflytande i studien än 
person L, och därför kunnat vinkla den till sin fördel.
Men anledningen är sannolikt mer komplex än så, det skulle krävas ytterligare 
en studie för att besvara denna fråga.

\subsection{Slutsats}
\noindent
Studien bör göras om.
Framför allt för att den är rolig att genomföra.
