% $Id$
\section{Resultat}
\label{sec:results}
Entropin för de olika insamlingsperioderna är listade 
i \prettyref{tbl:entropy}.
Deltagarna benämns D och L.
Båda deltagarna är högerhänta.
Dessa bokstäver har inget som helst att göra med deltagarnas namn.
Alla lappar finns transkriberade i \prettyref{app:data}.

\begin{table}
  \centering
  \begin{tabular}{crr}
    \toprule
    \textbf{Omgång} & \textbf{D} & \textbf{L} \\
    \midrule
    1   & 5   & 2 \\
    2   & 6   & 1 \\
    3   & 3   & 3 \\
    4   & 7   & 4 \\
    5   & 4   & 1 \\
    \midrule
    Alla  & 8 & 2 \\
    \bottomrule
  \end{tabular}
  \caption{Sammanställning av entropin för post-it-lapparna för de två 
  deltagarna D och L.}
  \label{tbl:entropy}
\end{table}


\section{Analys}
\label{sec:analysis}
Det är intressant att se person D genomgående har högre entropi i sina lappar 
än person L.
På grund av denna genomgående skillnad normaliserar vi värdena för att enklare 
kunna se hur färgerna påverkar.
Detta visas i \prettyref{tbl:entropynorm}.

\begin{table}
  \centering
  \begin{tabular}{c*{2}{D{.}{.}{2}}}
    \toprule
    \textbf{Omgång} & \textbf{D} & \textbf{L} \\
    \midrule
    \vdots  & \vdots  & \vdots \\
    \midrule
    Alla  & 4.45 & 2.23 \\
    \bottomrule
  \end{tabular}
  \caption{Normaliserad sammanställning av entropin för post-it-lapparna.}
  \label{tbl:entropynorm}
\end{table}
