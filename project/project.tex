% $Id$
% Author: Daniel Bosk <daniel.bosk@miun.se>
\documentclass[a4paper]{miunasgn}
\usepackage[utf8]{inputenc}
\usepackage[T1]{fontenc}
\usepackage[english,swedish]{babel}
\usepackage[hyphens]{url}
\usepackage{hyperref}
\usepackage{prettyref,varioref}
\usepackage{verbatim}
\usepackage[binary]{SIunits}
\usepackage[today,nofancy]{svninfo}
\usepackage[natbib,style=numeric-comp,maxbibnames=99]{biblatex}
\addbibresource{literature.bib}
\usepackage[varioref,prettyref]{miunmisc}

\svnInfo $Id$

\courseid{DT001G}
\course{Informationsteknologi grundkurs}
\assignmenttype{Slutprojekt}
\title{En undersökande jämförelse}
\author{Daniel Bosk\thanks{%
	Detta verk är tillgängliggjort under licensen Creative Commons 
	Erkännande-DelaLika 2.5 Sverige (CC BY-SA 2.5 SE).
	För att se en sammanfattning och kopia av licenstexten besök URL 
	\url{http://creativecommons.org/licenses/by-sa/2.5/se/}.
	}
}
\date{\svnId}

\begin{document}
\maketitle
\thispagestyle{foot}
\tableofcontents


\section{Introduktion}
\noindent
All forskning som producerar ny kunskap måste använda sig av någon metod som 
kan skapa ett tillförlitligt resultat för att kunna användas.
De metoder som används måste vara objektiva nog för att ge resultat som är 
reproducerbara.
Utifrån dessa resultat dras sedan en slutsats som är den nya kunskapen.
Därefter kan detta byggas vidare på och dessa resultat kan användas i samband 
för att formulera ytterligare resultat.


\section{Syfte}
\noindent
Syftet med uppgiften är att examinera att studenten ska kunna:
\begin{itemize}
  % $Id$
% Author:	Daniel Bosk <daniel.bosk@miun.se>
\item få en inblick i textbaserade användargränssnitt,
\item se sambandet mellan vad som händer i det textbaserade och det
	grafiska gränssnittet,
\item få en förståelse för skillnaden mellan absoluta och relativa
	sökvägar, samt
\item få en ökad förståelse för hur filsystemet fungerar.

\end{itemize}


\section{Läsanvisningar}
\noindent
Innan du påbörjar laborationen ska du ha läst kapitel 1, 2, 5 och 6.1--6.4 
i \cite{Brookshear2012csa} och avsnitt 2--4 i \cite{pythonkramaren1}.



\section{Genomförande}

Notera först att omfattningen för detta projekt är närmare tre veckor 
heltidsarbete, eftersom att kursen läses på halvfart innebär detta strax över 
fem veckors arbete om 20 timmar i veckan.
Om du kommer direkt från gymnasiet kan du tänka dig att arbetet är något större 
än projektarbetet som görs under den sista årskursen.
Kraven på innehållet är dock något högre.

När ett projekt genomförs och ska resultera i en rapport, vilket majoriteten av 
projekt faktiskt gör, är det viktigt att rapporten inte blir en 
efterhandskonstruktion.
Projektet i denna kurs är därför indelad i fem deluppgifter: introduktion, 
teori, metod, resultat och analys, samt diskussion; en för respektive del i en 
klassisk akademisk rapport.

\subsection{Introduktion}

Denna uppgift utgör alltså den första av fem delar.
Utformningen gör att de olika delarna ska genomföras i ordning och lämnas in 
till din handledare allteftersom projektet genomförs -- projektet ska 
genomföras i precis denna ordning.
Naturligtvis kan det hända ibland att små detaljer behöver läggas till 
i efterhand, men i stort ska projektet genomföras enligt denna progression.

Introduktionsavsnittet i en akademisk rapport, vilket denna del går ut på att 
skriva, ska ta upp bakgrunden och syftet med undersökningen.
Introduktionen ska beskriva bakgrunden och tidigare studier på området, en 
sammanfattning av området och en sorts motivering för läsaren av rapporten 
varför detta är intressant och varför denne ska fortsätta att läsa.

Därutöver ska ett tydligt syfte med lika tydliga avgränsningar och konkreta 
frågeställningar ges.
Meningen med dessa är att läsaren ska få veta exakt vad som ska besvaras 
i rapporten.

Börja med att fundera på vad du vill fokusera på.
När du väl valt ett område formulerar du mål och en frågeställning, dessa 
publicerar du i forumet i lärplattformen för att diskutera med andra studenter 
och handledarna.
Läs igenom några andra målformuleringar som finns publicerade och kommentera på 
dessa.
En målformulering och frågeställning bör vara tydlig och inte gå att 
missförstå.
Det är lätt att bli ''hemmablind'' i sitt eget tänkande och formulerande, 
därför bör någon annan som inte är insatt titta på det och ge återkoppling.
Syftet med mål och frågeställningen är att en annan student skulle kunna 
använda exakt samma mål och frågeställning och genomföra samma undersökning 
oberoende.

När du känner dig säker på ditt mål och frågeställning, exempelvis efter 
återkoppling, kan du sätta igång med din undersökning.

\subsection{Teori}

Teoriavsnittet ska behandla all teori som krävs för att förstå dels relevansen 
av undersökningen och dels undersökningen i sig.
Detta avsnitt ska därför behandla de begrepp och definitioner som är relevanta 
för området, exempelvis en sammanfattning av relevanta delar för protokoll som 
ska undersökas.
Om prestandan för att använda UDP eller TCP för SNMP ska undersökas, då bör de 
fundamentala skillnaderna mellan UDP och TCP som kan ha inverkan på detta 
beskrivas.
Likaså funktionaliteten hos SNMP bör beskrivas.

Det är även i detta avsnitt som tidigare forskning på området ska redogöras 
för.
Vilka andra undersökningar finns och vad kom de fram till?

Detaljnivån i teoriavsnittet ska lämpa sig för en läsare med en bakgrund inom 
området.

\subsection{Metod}

Metodavsnittet behandlar hela den undersökande delen.
Det är här själva undersökningsmetoden redogörs för, och denna ska presenteras 
så tydligt att \emph{läsaren ska kunna göra om undersökningen på egen hand för 
att verifiera resultatet}.
Det bör dock påpekas att onödiga detaljer som ''jag råkade skriva fel under det 
första försöket'' ska utelämnas.
Det ska vara kort och koncist, men ändå detaljerat till den grad att läsaren 
kan göra om undersökningen.

I detta avsnitt ska alltså versioner av hårdvara, versioner av protokoll, 
versioner av programvaror och konfigurationer av dessa redogöras för.
Den logiska nätverkstopologin, och den fysiska om detta är relevant för 
undersökningen, bör tas upp i detta avsnitt.
(Den logiska topologin antas relevant med tanke på kursens och programmets 
inriktning, men om undersökningen inte är nätverksinriktad finns kanske ingen 
anledning att ha den med.)

Du ska nu utforma din undersökningsmetod, det arbete som är tänkt ska besvara 
din frågeställning -- du ska utforma den men \emph{ännu inte genomföra den}.
Fundera över hur du ska gå tillväga för att besvara din uppsatta frågeställning 
från introduktionsavsnittet.
Kom ihåg att läsaren ska kunna utgå från samma frågeställning, använda din här 
specificerade metod och sedan få samma resultat i sina egna mätningar.

\subsection{Resultat och analys}

I resultatavsnittet ska undersökningens resultat \emph{objektivt} redogöras 
för.
Efter att ha genomfört allt som står i metodavsnittet är det som beskrivs 
i resultatavsnittet utfallet.
Detta måste beskrivas helt objektivt \emph{utan något som helst inslag av 
värdering}.
Exempelvis ''vid den första mätningen mellan dator A och dator B var 
genomstömningen \unit{23}{\mega\bit\per\second}'' -- varken mer eller mindre 
ska anges.

Om resultatet innefattar mycket data ges en sammanfattning av datat i form av 
medelvärden etcetera här i resultatavsnittet, och den fullständiga datamängden 
ges som bilagor.
Hur denna sammanfattning har framställts utifrån datat ska också framgå 
tydligt.

Det är sedan i analysavsnittet som resultatet analyseras och relateras till 
teorin.
Det vill säga, resultaten ska förklaras: varför hände detta?
Det är även här det redogörs för hur ''\unit{23}{\mega\bit\per\second} 
genomströmning'' ska tolkas utifrån teorin, varför blev det 
\unit{23}{\mega\bit\per\second} istället för närmare den teoretiska maxgränsen 
\unit{1000}{\mebi\bit\per\second}?
Huruvida detta resultat är bra eller ej ska inte diskuteras här, det hör hemma 
i diskussionsavsnittet.
Resultatet ska analyseras och jämföras med eventuella tidigare resultat som 
presenterats i teoriavsnittet, även hur resultatet förhåller sig till de 
teoretiska förväntningarna.

Ibland kan det vara passande att avsnitten resultat och analys slås samman till 
ett gemensamt avsnitt, men oftast hålls de separerade.

Det är nu du faktiskt genomför din undersökning, det vill säga hela din 
redogjorda metod i föregående avsnitt.
Följ din specificerade metod och notera resultaten, anteckna vad som händer och 
spara loggfiler för senare analys.
Sammanfatta resultaten av undersökningen i kapitlet Resultat.

Därefter påbörjar du din analys av dessa resultat.
Sammanställ dessa, jämför dem med förväntningarna från teorin och ställ upp 
mått för hur du objektivt avgör avvikanden från de förväntningarna.
Notera att ''det kändes som ett dåligt resultat'' inte är objektivt, däremot 
''om medelvärdet av alla mätpunkter är större än \(0.5\) är det inom rimliga 
gränser från de teoretiska förväntningarna'' är ett objektivt avgörande.
Läsaren kan förvisso ha synpunkter på just värdet \(0.5\), men denne kan 
fortfarande göra exakt samma avgörande som du.

\subsection{Diskussion}

Efter den objektiva presentationen av resultaten och den lika objektiva 
analysen är det i diskussionen som dessa ska värderas.
I diskussionen ska författarens egna slutsatser av resultaten presenteras; det 
är viktigt att diskutera tillförlitligheten hos resultaten, denna beror på 
metoden som använts för att uppmäta resultatet och beroende på 
tillförlitligheten kan olika slutsatser dras.
Generaliserbarheten hos resultaten ska också diskuteras.
Resultat kan gå att generalisera, beroende på utformning, men ibland kan de 
vara alldeles för specifika för att detta ska gå att göra.

Det ska också diskuteras huruvida undersökningen uppfyller syftet och besvarar 
frågeställningen.
Det är i samband med detta resonemang som slutsatsen dras, alltså den konkreta 
sammanfattande besvaringen av frågeställningen.

Det är nu du ska värdera dina resultat, är de tillförlitliga?
Detta avgör du genom att kritiskt granska din undersökningsmetod.
Hur spelar din undersökningsmetod in på resultatet, tror du att resultatet 
skulle ha varit annorlunda om du använt ett annat metodval?

Du kan därefter gå vidare genom att diskutera huruvida undersökningen har 
uppfyllt det uppställda syftet och besvarat frågeställningen.
De konkreta svaren på frågeställningen, och hur tillförlitliga dessa är, 
skriver du som en slutsats för undersökningen.

Du kan därefter avsluta diskussionsavsnittet med några förslag på vidare 
forskning: finns det något relaterat som skulle vara intressant att 
vidareutveckla denna undersökning till?

Viktigt att påpeka är att alla frågor i frågeställningen ska besvaras.
Dock kan det visa sig att någon av frågorna inte går att säkert besvara med den 
använda metoden, då ska detta tas upp i och eventuellt förslag för hur denna 
kan besvaras ska ges.


\section{Examination}

Den färdiga rapporten i PDF-format tillsammans med källkod lämnas in 
i inlämningslådan i lärplattformen.
Omfattningen bör vara omkring sju (7) sidor text, utöver detta tillkommer 
titel, sammanfattning (abstract), innehållsförteckning och referenslista.
Rapportens förstasida ska ha titel och författare med e-postadress\footnote{%
  Ett tips är att använda kommandot \texttt{\textbackslash footnote}.
}, datum och sammanfattning (abstract).
Därefter följer innehållet direkt, se den bifogade rapportmallen för ett 
exempel.

Utöver detta föreligger följande krav:
\begin{itemize}
  \item Rapporten ska vara typsatt med LaTeX.
  \item Rapporten ska ha minst en figur eller tabell.
  \item Rapportens referenser ska göras med diverse \texttt{\textbackslash 
    cite}-kommandon tillsammans med en BibTeX-databas med alla referenser.
    Ett tips är att använda kursens BibTeX-databas.
  \item Rapportens studie ska innehålla minst en matematisk formel.
  \item Rapportens disposition ska följa en akademisk rapport och den ska vara 
    skriven på formell akademisk svenska eller engelska.
  \item Referenser ska ges enligt \citetitle{IEEEcitation} \cite{IEEEcitation}.
\end{itemize}

Rapporten presenteras även muntligen vid tillfälle för helklass, omfattning på 
presentationen är 14--15 minuter.
Slides är obligatoriskt för godkänd presentation.


\printbibliography
\end{document}
