% $Id$
% Author: Daniel Bosk <daniel.bosk@miun.se>
\documentclass[a4paper]{miunasgn}
\usepackage[utf8]{inputenc}
\usepackage[T1]{fontenc}
\usepackage[english,swedish]{babel}
\usepackage{url,hyperref}
\usepackage{prettyref,varioref}
\usepackage{natbib}
\usepackage{verbatim}
\usepackage[today,nofancy]{svninfo}
\usepackage[natbib,varioref,prettyref]{miunmisc}

\svnInfo $Id$

\courseid{DT001G}
\course{Informationsteknologi grundkurs}
\assignmenttype{Slutprojekt}
\title{En undersökande jämförelse}
\author{Daniel Bosk\thanks{%
		Baserad på en uppgift författad av Magnus Eriksson.
	}
}
\date{\svnId}

\begin{document}
\maketitle
\thispagestyle{foot}
\tableofcontents


\section{Introduktion}
\noindent
All forskning som producerar ny kunskap måste använda sig av någon metod som 
kan skapa ett tillförlitligt resultat för att kunna användas.
De metoder som används måste vara objektiva nog för att ge resultat som är 
reproducerbara.
Utifrån dessa resultat dras sedan en slutsats som är den nya kunskapen.
Därefter kan detta byggas vidare på och dessa resultat kan användas i samband 
för att formulera ytterligare resultat.


\section{Syfte}
\noindent
Syftet med uppgiften är
\begin{itemize}
	\item att fördjupa sig inom något delområde som tagits upp i kursen,
	\item att öva på genomförandet en enklare objektiv undersökning och dra en 
		slutsats av resultatet, samt
	\item att öva på att skriva en rapport och hålla en muntlig presentation av 
		densamma.
\end{itemize}


\section{Läsanvisningar}
\noindent
% $Id$
Föreläsningen går igenom kursstruktur och organisation.
Den ger en översikt över undervisning och examination.
Den motsvarar således att läsa igenom allt kursmaterial och lite därtill.

Utöver detta täcks även kapitel 0 i \citetitle{Brookshear2012csa} 
\cite{Brookshear2012csa}.
Kapitlet introducerar ämnena datateknik och datalogi (datavetenskap).
Det ger även en historisk överblick av området vilket är bra för att förstå 
varför området är som det är och dess framtida utveckling.



\section{Genomförande}
\noindent
Börja med att fundera på vad du vill fokusera på.
När du väl valt ett område formulerar du mål och en frågeställning, dessa 
publicerar du i forumet i lärplattformen.
Läs igenom några andra målformuleringar som finns publicerade och kommentera på 
dessa.
En målformulering och frågeställning bör vara tydlig och inte gå att 
missförstå.
Det är lätt att bli ''hemmavan'' i sitt eget tänkande och formulerande, därför 
bör någon annan som inte är insatt titta på det och ge återkoppling.
Genom detta får du från både lärare och andra studenter.
Syftet med mål och frågeställningen är att en annan student skulle kunna 
använda exakt samma mål och frågeställning och genomföra samma undersökning 
oberoende.

När du känner dig säker på ditt mål och frågeställning, exempelvis efter 
återkoppling, kan du sätta igång med din undersökning.
Du följer rapportmallen när du genomför undersökningen och skriver din rapport.

\begin{comment}
\subsection{Inspiration för projektidéer}
\noindent
The topic should be comparative analyses of (minimum two) different 
technologies.  Here are four possible alternatives: 

Compare two data communication standards. Explain their basic principles with 
your own words, and draw your own conclusions about their advantages and 
disadvantages, as well as which of them are expected to be improved in couple 
of years. http://wikipedia.org , and the course book can be used as resources.  
If there are RFCs for these standards (Request for Comments – which exist for 
the standards at the network and higher layers), read them. RFC are normally 
easy to read. The standards on physical layer or data‐link layer, for example 
IEEE standards are often RFCs, but that would be more difficult to understand 
than for example the transport protocols UDP and TCP or the e‐mail protocols 
POP3 and IMAP4.  You can even compare two xDSL – modem standards (for example 
ADSL2+ and VDSL2), or different wireless communication schemes (for example 
IEEE802.11a versus b, different security schemes for wireless LAN, Bluetooth 
versus IRDA, etc.). Here you will probably find the most difficult to read the 
documents about the standard. Describe their advantages and disadvantages.  
Draw your own conclusions about when the respective protocol is most convenient 
to be used and what is expected to be largely used in the coming years. 

Test and compare two types of software in the area of data communication.  
Install and test two comparable software packages, for example two firewalls, 
two instant message‐programs, two IP telephony programs, etc. You can use 
a freeware, shareware or demo versions of commercial packages. One site that 
offers downloads www.download.com. Once you decide what you are going to 
compare, try to find out the protocols and other paradigms used in each of 
them. Evaluate their usability, functionality and performance, for example in 
a specific usage scenario or for specific user group. Criticize the marketing 
motivations for the developers. Compile your thoughts about both programs 
‐ advantages, disadvantages and performance‐ into a table or a bulleted list, 
and draw conclusions about which of the programs is the best taking into 
account all the properties, or according to your own beliefs will dominate the 
market in the coming years.

Compare two types of equipment. You can for example design a home network from 
a given set of specifications or suggest changes in your company’s network.  
Compare alternative solutions and recommend one of them. Test some of the 
equipment and describe how it is configured and installed. Draw your own 
conclusions about the advantages and disadvantages of the solution proposed 
when compared to the others. Criticize the producer marketing motivation and 
policy.

Compare two data communications services, for example two Internet or broadband 
suppliers or two wireless communication services. Describe the technical 
differences, test them yourself and make a questionnaire to get opinion from 
other users, or interview the suppliers about their future plans. Criticize the 
suppliers about their marketing policy.
\end{comment}


\section{Examination}
\noindent
Den färdiga rapporten lämnas in i inlämningslådan i lärplattformen.
Omfattningen bör vara omkring fyra (4) sidor tät text, utöver detta tillkommer 
titel, sammanfattning (abstract), innehållsförteckning och referenslista.
Rapportens förstasida ska ha titel, författare med e-postadress och 
användarnamn\footnote{%
	Ett tips är att använda kommandot \texttt{\textbackslash footnote} i \LaTeX.
}, datum och sammanfattning (abstract).
Därefter följer innehållet direkt.

Rapporten presenteras även muntligen vid tillfälle för helklass, omfattning på 
presentationen är cirka 15 minuter.
Slides är obligatoriskt för godkänd presentation.


\bibliography{../literature}
\end{document}
