\documentclass[a4paper,11pt,logo]{miunart}
\usepackage[utf8]{inputenc}
\usepackage{natbib}
\bibliographystyle{sweplnat}
\setcitestyle{numbers,square}
\usepackage{prettyref,varioref}
\usepackage[swedish]{babel}
\usepackage{url}
\usepackage[natbib,prettyref,varioref]{miunmisc}

\title{Kurslitteratur för\\DT001G Informationsteknologi grundkurs}
\author{Daniel Bosk}
\date{\today}

\begin{document}
\maketitle
\noindent
Den bok som används som huvudlitteratur på kursen är Brookshears bok 
\emph{Computer Science: An Overview} \citep{Brookshear2012csa}.
Utöver denna bok kommer en kort skrift av Post- och Telestyrelsen (PTS) 
\citep{PTStgr} samt en intressant analys \citep{Hunt2011abs} lösenorden som 
publicerades från Sonys lösenordsdatabas i en attack under 2011 att användas 
som litteratur.

Utöver dessa finns
RFC 791 \citep{RFC791},
5735 \citep{RFC5735} och
2460 \citep{RFC2460}
som referenslitteratur.


\section*{Läsanvisningar}
\noindent
\citet{Brookshear2012csa} tar upp grunden inom datateknik, den är värd att läsa 
i sin helhet -- detta rekommenderas!
De kapitel och avsnitt som behandas i denna kurs är dock enbart
\begin{itemize}
	\item kapitlen 0 till och med 5 i sina helheter,
	\item avsnitten 6.1 till och med 6.4, samt
	\item avsnitten 9.1 och 9.2.
\end{itemize}


\bibliography{../itgrund}
\end{document}
