\documentclass[11pt,a4paper]{miunasgn}
\usepackage[utf8]{inputenc}
\usepackage[english,swedish]{babel}
\usepackage{url,hyperref}
\usepackage{prettyref,varioref}
\usepackage{natbib}
\usepackage{listings}
\usepackage[today,nofancy]{svninfo}
\usepackage[natbib,varioref,prettyref,listings]{miunmisc}

\svnInfo $Id$
%\printanswers

\courseid{DT001G}
\course{Informationsteknologi grundkurs}
\assignmenttype{Laboration}
\title{Säkerhet}
\author{Daniel Bosk\footnote{%
	E-post: \href{mailto:daniel.bosk@miun.se}{daniel.bosk@miun.se}.
}}
\date{\svnId}

\begin{document}
\maketitle
\thispagestyle{foot}
\tableofcontents

\section{Introduktion}
\label{sec:Introduktion}
\noindent
Ett av de äldsta protokollen på internet är File Transfer Protocol (FTP) 
\citep{rfc959}.
Detta är ett enkelt filöverföringsprotokoll som utvecklades under en tid innan 
säkerhet började bli ett krav på internet.
Det finns många sätt att göra filöverföringar säkra ett av dem är att göra dem 
över krypterade tunnlar.
Ett annat sätt är SSH File Transfer Protocol (SFTP) där all kommunikation sker 
över en SSH-tunnel.
Det är detta protokoll som ska användas i denna laboration.


\section{Syfte}
\label{sec:Syfte}
\noindent
Syftet med uppgiften är att du ska lära dig använda en SFTP-klient.
Anledningen är för att kunna logga in och överföra filer på ett säkert sätt.
Du ska
\begin{itemize}
	\item kunna ladda hem filer från en server,
	\item kunna ladda upp filer till en server, och
	\item kunna skapa och verifiera kontrollsummor.
\end{itemize}
Du ska kunna använda verktyg i terminalen för att göra detta.


\section{Läsanvisningar}
\label{sec:Lasanvisningar}
\noindent
Innan du påbörjar laborationen ska du ha läst kapitel 1, 2, 5 och 6.1--6.4 
i \cite{Brookshear2012csa} och avsnitt 2--4 i \cite{pythonkramaren1}.



\section{Genomförande}
\label{sec:Genomforande}
\noindent
Skriv en vanlig textfil med exempelvis \emph{gedit} där du anger vilka 
kommandon som ska köras för att ladda upp en textfil till en annan dator med 
hjälp av sftp(1).

Skapa en SHA256-kontrollsumma för filen och ladda sedan upp denna textfil med 
namnet \emph{dt001g.txt} till servern \url{myfiles.miun.se} i katalogen 
\emph{public\_html} i hemkatalogen.


\section{Examination}
\label{sec:Examination}
\noindent
I textfältet i inlämningslådan i lärplattformen lämnar du in kontrollsumman för 
din textfil.
Denna kontrollsumma kommer vid rättning att kontrolleras mot den fil ni laddat 
upp till servern, den går att komma åt publikt via URL
\begin{center}
	\url{http://myfiles.miun.se/~username/dt001g.txt}.
\end{center}


\bibliography{../../itgrund,rfc}
\end{document}
