\documentclass[11pt,a4paper]{miunasgn}
\usepackage[utf8]{inputenc}
\usepackage[english,swedish]{babel}
\usepackage{url,hyperref}
\usepackage{prettyref,varioref}
\usepackage{natbib}
\usepackage{listings}
\usepackage[today,nofancy]{svninfo}
\usepackage[natbib,varioref,prettyref,listings]{miunmisc}

\svnInfo $Id$
%\printanswers

\courseid{DT001G}
\course{Informationsteknologi grundkurs}
\assignmenttype{Laboration}
\title{Säkerhet}
\author{Daniel Bosk\footnote{%
	E-post: \href{mailto:daniel.bosk@miun.se}{daniel.bosk@miun.se}.
}}
\date{\svnId}

\begin{document}
\maketitle
\thispagestyle{foot}
\tableofcontents


\section{Introduktion}
\label{sec:Introduktion}
\noindent
Använd valfritt operativsystem.
Du behöver valfri webbläsare och PDF-läsare.


\section{Syfte}
\label{sec:Syfte}
\noindent
Syftet med denna uppgift är att ge insikt i grundläggande datasäkerhet.
Du ska
\begin{itemize}
	\item få insikt i risker med din egen internetanvändning,
	\item ha vetskap om någon av de vanligaste attackmetoderna,
	\item insikt i några av konsekvenserna av attacker mot webbsidor och
		företag, och
	\item få insikt i hur utsatt man kan vara i dagens informationssamhälle.
\end{itemize}

Syftet med uppgiften är att du ska lära dig använda en SFTP-klient.
Anledningen är för att kunna logga in och överföra filer på ett säkert sätt.
Du ska
\begin{itemize}
    \item kunna ladda hem filer från en server,
    \item kunna ladda upp filer till en server, och
	\item kunna skapa och verifiera kontrollsummor.
\end{itemize}
Du ska kunna använda verktyg i terminalen för att göra detta.


\section{Läsanvisningar}
\label{sec:Lasanvisningar}
\noindent
\dots


\section{Genomförande}
\label{sec:Genomforande}
\noindent
Du ska i denna uppgift få ett säkerhetsperspektiv på datorer och kommunikation.
Genomför uppgifterna i den ordning de ges.
Svaren skriver du i svarsmallen (\emph{svar\_3.2.doc}) som finns bifogad i
WebCT.

\begin{questions}															   
	\question\label{q:PTS}
	Börja med att läsa igenom \emph{Tolv goda råd} \citep{PTStgr} om
	Internetsäkerhet för hemmet.
	%\footnote{%
	%	URL: \url{http://www.pts.se/sv/Internet/Internetsakerhet/For-hemmet/Tretton-goda-rad/}.
	%} 
	Titta därefter igenom några av de andra sidorna under \emph{För hemmet}.
	\begin{parts}
		\part
		Vilka risker har du hittills utsatt dig för?

		\part\label{q:SkyddaSig}
		Hur kan du skydda dig?
	\end{parts}

	\question
	\label{q:OWASPTopTen}
	Läs igenom en av attackerna i OWASP Top 10 \citep{OWASP2010ot} och
	sammanfatta hur den fungerar.
	\begin{solution}
		Se referensen \citet{OWASP2010ot}.
	\end{solution}

	\question
	Läs Wired Magazines Threat Level-artikel \emph{Facebook Enables HTTPS So
	You Can Share Without Being Hijacked} \citep{Singel2011feh}.
	%\footnote{%
	%	URL: \url{http://www.wired.com/threatlevel/2011/01/facebook-https/}.
	%}.
	Hur säker är din Facebook-användning?
	\begin{solution}
		Troligtvis inte säker \dots
	\end{solution}

	\question\label{q:Attack}
	Sök på Internet efter dokumenterade attacker, exempelvis attackerna mot
	CitiBank,
	RSA,
	Lockheed Martin,
	Gawker samt
	Sony Playstation Network,
	Sony BMG och
	Sony Pictures
	under våren 2011 eller det eviga attackerandet av
	Facebook och
	Twitter\footnote{%
		Bra källor kan vara:
			\url{http://www.Wired.com},
			\url{http://www.TheRegister.co.uk} och
			\url{http://NakedSecurity.Sophos.com}.
	}.
	\begin{parts}
		\part
		Redogör för en attack, hur den gick tillväga, och försök att
		klassificera denna, om möjligt, utifrån OWASP Top 10 från
		\prettyref{q:OWASPTopTen}.

		\part
		Diskutera konsekvenserna av attacken i en kort text.
	\end{parts}
	\begin{solution}
		CitiGroup-hacket använde \emph{A4: Insecure Direct Object References}
		för att gå igenom alla bankens konton.
		Det räckte med att logga in med giltiga uppgifter och därefter utan
		problem komma åt alla andra konton i banken.

		RSA hackades med en lite mer avancerad strategi:
		\begin{enumerate}
			\item Efterforskning om detaljer om personal.
				Via exempelvis företagets webbsidor tillsammans med sociala
				medier (ex. Facebook, Twitter).
			\item Skicka specialskrivna mail till ett antal anställda som
				verkade äkta.
				Bifogat var Exceldokument (.xls) som använde en zero-day
				exploit för Adobes Flashplayer för att ladda hem en trojan.
			\item Trojanen tillät fjärrstyrning av datorn.
			\item Dessa användares konton användes för att internt hacka konton
				med fler privilegier, exempelvis administratörskonton för
				servrar.
			\item Väl inne i servrarna laddades informationen över till en av
				RSAs egna FTP-servrar och från denna till externa servrar
				(hackade).
		\end{enumerate}
		De data som stals från RSA användes för företagets SecureID.
		En av RSAs SecureID-kunder, Lockheed Martin, blev hackade till följd av
		detta.

		Sonys Playstation Network blev hackade och information om 77 miljoner
		användare togs.
		Servrarnas programvara var ej uppdaterad.

		Sony Pictures: SQL injection gav hela databasen.
	\end{solution}

	\question\label{q:Losenord}
	Hur säkra är dina lösenord?
	Kolla på Post- och Telestyrelsens tjänst Testa lösenord\footnote{%
		URL: \url{http://www.testalosenord.pts.se/}.
	} vad som krävs för ett säkert lösenord.
	\begin{parts}
		\part
		Försök att hitta ett säkert lösenord som även är lätt att komma ihåg.
		(Lösenordet \emph{ql!3d2d@} klassificeras som ej lätt att komma ihåg.)
		\begin{solution}
			Lösenfraser är bättre än lösenord, exempelvis ''Apan äter 1 banan''
			är enkelt att komma ihåg och är ett starkt lösenord enligt PTS.
			Vi har tur i Sverige eftersom att å, ä och ö räknas som
			specialtecken.
		\end{solution}

		\part
		Vad är en bra lösenordsstrategi?
		\begin{solution}
			Använd alltid starka lösenord och använd inte samma lösenord på
			flera ställen.
		\end{solution}

		\part
		Varför är det viktigt med en bra lösenordsstrategi?
		Läs \emph{A brief Sony password analysis} \citep{Hunt2011abs}.
	\end{parts}

	\question\label{q:Kamkar}
	Se föreläsningen \emph{How I met your girlfriend} \citep{Kamkar2010him}
	%\footnote{%
	%	YouTube: del 1 -- \url{http://www.youtube.com/watch?v=fEmO7wQKCMw},
	%	del 2 -- \url{http://www.youtube.com/watch?v=2ctRfWnisSk},
	%	del 3 -- \url{http://www.youtube.com/watch?v=vJtmZZGcR54}.
	%	Totalt ca 40 minuter.
	%}
	från datasäkerhetskonferensen DefCon18.
	Sammanfatta dina tankar om innehållet i en kort text.
	\begin{solution}
		Det finns väldigt många olika sätt att attackera dagens
		informationssystem.
		Det finns inget perfekt skyddat system och systemets användare måste
		vara vaksamma för att inte råka illa ut.
	\end{solution}

\end{questions}

\subsection{Del två: SFTP}
\noindent
Använd valfritt operativsystem.
Du behöver en SFTP-klient, ett program för att generera kontrollsummor och en
webbläsare.
\begin{description}
	\item[Windows] För en SFTP-klient kan du installera PuTTY\footnote{%
			URL: \url{ftp://ftp.chiark.greenend.org.uk/users/sgtatham/putty-latest/x86/putty-0.61-installer.exe}.
		}.
		SFTP-klienten används genom kommandot \emph{psftp}.
		PuTTY innehåller även en SSH-klient (kommandot \emph{putty}) och en
		SCP-klient (kommandot \emph{pscp}).
		Det finns även grafiska SFTP-klienter, ett exempel är WinSCP\footnote{%
			URL: \url{http://winscp.net/download/winscp433setup.exe}.
		}

		För att beräkna kontrollsummor kan du använda
		\emph{sha256sum}\footnote{%
			URL: \url{http://blog.nfllab.com/archives/152-Win32-native-md5sum,-sha1sum,-sha256sum-etc..html}.
		} som kompilerats för Windows från samma källkod som för UNIX.
	\item[UNIX] De flesta UNIX-baserade system har som standard en SFTP-klient
		(kommandot \emph{sftp}), SSH-klient (kommandot \emph{ssh}), och
		SCP-klient (kommandot \emph{scp}).
		De flesta UNIX-baserade system har även som standard program för att
		generera och verifiera kontrollsummor.
		Du kan använda kommandot \emph{sha256sum}.
\end{description}


\section{Examination}
\label{sec:Examination}
\noindent
Exportera dokumentet \emph{svar\_3.2.doc} till PDF-format, döp dokumentet till
\begin{center}
	\emph{FornamnEfternamn\_inlupp3.2\_datum}
\end{center}
och lämna in under inlämningsuppgifter i WebCT.

Du ska även bidra till diskussionen i WebCT.
Diskutera vad som är viktigt att tänkta på ur säkerhetssynpunkt.
Du kan använda dig av dina svar från frågorna
\ref{q:PTS}\ref{q:SkyddaSig},
\ref{q:OWASPTopTen},
\ref{q:Attack},
\ref{q:Losenord} och
\ref{q:Kamkar}.


\bibliography{../../itgrund}
\end{document}
