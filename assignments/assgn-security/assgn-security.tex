\documentclass[11pt,a4paper]{miunasgn}
\usepackage[utf8]{inputenc}
\usepackage[english,swedish]{babel}
\usepackage{url,hyperref}
\usepackage{prettyref,varioref}
\usepackage{natbib}
\usepackage{listings}
\usepackage[today,nofancy]{svninfo}
\usepackage[natbib,varioref,prettyref,listings]{miunmisc}

\svnInfo $Id$
%\printanswers

\courseid{DT001G}
\course{Informationsteknologi grundkurs}
\assignmenttype{Laboration}
\title{Säkerhet}
\author{Daniel Bosk\footnote{%
	E-post: \href{mailto:daniel.bosk@miun.se}{daniel.bosk@miun.se}.
}}
\date{\svnId}

\begin{document}
\maketitle
\thispagestyle{foot}
\tableofcontents


\section{Introduktion}
\label{sec:Introduktion}
\noindent
Säkerheten i användandet av datorn blir allt viktigare.
Detta ses tydligt på året 2011.
Många stora företag hackades och alltifrån deras e-post till deras 
lösenordsdatabaser publicerades öppet på internet.
Ett tydligt exempel är DigiNotar, en av Nederländernas största certificate 
authorities (CA) då som gick i konkurs efter en av de största skandalerna 
i historien av säkerhet på internet.
Ett annat exempel från 2012 är Mat Honan, journalist på techtidningen Wired, 
vars digitala liv helt försvann i en attack.


\section{Syfte}
\label{sec:Syfte}
\noindent
Syftet med denna uppgift är att ge insikt i grundläggande datasäkerhet.
Du ska
\begin{itemize}
	\item få insikt i risker med din egen internetanvändning,
	\item ha vetskap om någon av de vanligaste attackmetoderna,
	\item insikt i några av konsekvenserna av attacker mot webbsidor och
		företag, och
	\item få insikt i hur utsatt man kan vara i dagens informationssamhälle.
\end{itemize}


\section{Läsanvisningar}
\label{sec:Lasanvisningar}
\noindent
Innan du påbörjar laborationen ska du ha läst kapitel 1, 2, 5 och 6.1--6.4 
i \cite{Brookshear2012csa} och avsnitt 2--4 i \cite{pythonkramaren1}.



\section{Genomförande}
\label{sec:Genomforande}
\noindent
Du ska i denna uppgift få ett säkerhetsperspektiv på datorer och kommunikation.
Genomför uppgifterna i den ordning de ges.

\begin{questions}															   
	\question\label{q:PTS}
	Börja med att läsa igenom \emph{Tolv goda råd} \citep{PTStgr} om
	Internetsäkerhet för hemmet.
	%\footnote{%
	%	URL: \url{http://www.pts.se/sv/Internet/Internetsakerhet/For-hemmet/Tretton-goda-rad/}.
	%} 
	Titta därefter igenom några av de andra sidorna under \emph{För hemmet}.
	\begin{parts}
		\part
		Vilka risker har du hittills utsatt dig för?

		\part\label{q:SkyddaSig}
		Hur kan du skydda dig?
	\end{parts}

	\question
	Läs och fundera över Mat Honans öde \citep{Honan2012haa} och hur man bör 
	undvika att hamna i samma sits \citep{Zetter2012hnt}.
	Hur ser säkerheten ut för ditt digitala liv och vad kan du göra för att 
	skydda dig?
	\begin{solution}
		Troligtvis inte säker \dots
		Se \citet{Zetter2012hnt}.
	\end{solution}

	\question\label{q:Losenord}
	Hur säkra är dina lösenord?
	Kolla på Post- och Telestyrelsens tjänst \emph{Testa lösenord}\footnote{%
		URL: \url{http://www.testalosenord.pts.se/}.
	} vad som krävs för ett säkert lösenord.
	\begin{parts}
		\part
		Undersök vilka olika typer av lösenord som PTS klassificerar som starka 
		respektive svaga, ge några exempel på vardera kategori.
		Vilka är enklast att komma ihåg?
		\begin{solution}
			Lösenfraser är oftast lättare att komma ihåg än lösenord, exempelvis 
			''Apan äter 1 banan'' är enkelt att komma ihåg och är ett starkt lösenord 
			enligt PTS.
			Vi har tur i Sverige eftersom att å, ä och ö räknas som
			specialtecken.
		\end{solution}

		\part
		Vad är en bra lösenordsstrategi?
		\begin{solution}
			Använd alltid starka lösenord och använd inte samma lösenord på
			flera ställen.
		\end{solution}
	\end{parts}

	\question\label{q:Kamkar}
	Sammanfatta dina tankar om föreläsningen \emph{How I met your girlfriend} 
	\citep{Kamkar2010him} och intervjun om hur Facebook har information om 
	icke-medlemmar \citep{Cherry2012fny} i en kort text.
	\begin{solution}
		Det finns väldigt många olika sätt att attackera dagens
		informationssystem.
		Det finns inget perfekt skyddat system och systemets användare måste
		vara vaksamma för att inte råka illa ut.
		
		Det är svårt när andra än en själv publicerar information om en själv.
	\end{solution}

\end{questions}


\section{Examination}
\label{sec:Examination}
\noindent
Besvara frågorna direkt i textfältet i inlämningslådan i lärplattformen.

Diskutera gärna vad som är viktigt att tänkta på ur säkerhetssynpunkt och dina 
övriga tankar från laborationen i diskussionsforumet i lärplattformen.


\bibliography{../../itgrund}
\end{document}
