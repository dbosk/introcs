Du ska inför laborationen ha läst igenom avsnittet om säkerhet för nätverk och 
internet i kurslitteraturen \citep[avsnitt 4.5]{Brookshear2012csa}.
Därefter kan du läsa Post- och Telestyrelsens (PTS) \emph{Tolv goda råd} 
\citep{PTStgr}.

Två intressanta artiklar om lösenord ska också läsas, en från 2011 
\citep{Hunt2011abs} och en från 2012 \citep{Cluley2012twp}.

För att få en bra perspektiv på vad som enkelt kan hända ska du läsa om 
Wired-journalisten Mat Honans digitala livs öde \citep{Honan2012haa} samt några 
tips om att undvika samma öde \citep{Zetter2012hnt}.

En annan viktig del av säkerheten är informationen som finns tillgänglig om 
dig, en transkriberad intervju av Steven Cherry \citep{Cherry2012fky} ska också 
läsas.
Samy Kamkars föreläsning \emph{How I met your girlfriend} \citep{Kamkar2010him} 
handlar också om hur känslig denna information är, den ska också ses i sin 
helhet.
