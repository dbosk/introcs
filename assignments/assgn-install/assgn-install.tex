\documentclass[11pt,a4paper]{miunasgn}
\usepackage[utf8]{inputenc}
\usepackage[english,swedish]{babel}
\usepackage{url,hyperref}
\usepackage{prettyref,varioref}
\usepackage{natbib}
\usepackage{listings}
\usepackage[natbib,varioref,prettyref,listings]{miunmisc}

\printanswers

\courseid{DT001G}
\course{Informationsteknologi grundkurs}
\assignmenttype{Laboration}
\title{Installationer}
\author{Daniel Bosk\footnote{%
	E-post: \href{mailto:daniel.bosk@miun.se}{daniel.bosk@miun.se}.
}}
\date{\today}

\begin{document}
\maketitle
\thispagestyle{foot}
\tableofcontents


\section{Introduktion}
\label{sec:Introduktion}
\noindent
I denna laboration ska du installera program som du behöver för kursen och 
eventuellt för kommande kurser på din dator.
Du ska även installera ett operativsystem, Ubuntu, som kommer att användas 
under kursens gång.

De programvaror du kommer att installera är följande:
\begin{itemize}
	\item TeX Live för Ubuntu och eventuellt MikTeX för Windows,
	\item eventuellt LibreOffice för Windows (redan installerat för Ubuntu), och
	\item 7-zip för Windows.
\end{itemize}


\section{Syfte}
\label{sec:Syfte}
\noindent
Syftet med laborationen är att du ska förbereda din dator med den programvara 
som behövs för kursen samt att få vana för enkel administration av programvaror 
i din dator.


\section{Läsanvisningar}
\label{sec:Lasanvisningar}
\noindent
För att genomföra denna laboration bör du ha läst kapitlet om operativsystem 
och bootprocessen \citep[kapitel 3]{Brookshear2012csa}.

Innan du genomför laborationen bör du också läsa igenom dokumentationen för 
installationen av Ubuntu \citep{UbuntuInstall}, detta är inte för att 
installationen är svår att genomföra utan för att du ska kunna fundera igenom 
dina beslut på förhand.

När Ubuntu väl är installerat finns dokumentationen \citep{UbuntuDesktop} som 
stöd för att börja använda systemet.
Det kan vara bra att orientera sig i denna för senare enkelt hitta vid behov.
Det rekommenderas att läsa igenom de första fem avsnitten -- \emph{Welcome to 
Ubuntu 12.04} till och med \emph{Log out, power off, switch users} -- innan 
installationen.
Du ska även läsa om hur program installeras \citep[se Install additional 
software]{UbuntuDesktop}.

Om du använder Windows och vill testa att installera programmen även där finns 
en instruktion för att installera \LaTeX\ för Windows \citep{Bosk2012lui}.
Hur \LaTeX\ installeras under Ubuntu täcks senare i denna lydelse.


\section{Genomförande}
\label{sec:Genomforande}
\noindent
Här följer genomförandet för laborationen.
Det rekommenderas att du läser igenom hela genomförandet innan du sätter igång.

\subsection{Ubuntu Desktop}
\noindent
Börja med att installera den senaste versionen av Ubuntu Desktop, för 
närvarande version 12.04.
Ubuntu går att ladda hem på URL
\begin{center}
	\url{http://www.ubuntu.com/download/}.
\end{center}
Det rekommenderas att du installerar Ubuntu parallelt med redan befintligt 
operativsystem om du har ett sådant, men du väljer själv vilken form du vill 
installera \citep[för detaljer, se][]{UbuntuInstall}.
Att enbart använda LiveCD rekommenderas inte eftersom att det blir problem med 
att installera programvaror och att du inte kan spara filer annat än på 
USB-minnen.

\subsection{Programvaror}
\noindent
När du har loggat in i din Ubuntu-installation är det dags att installera de 
programvaror du behöver.
Den enda programvara du behöver installera i Ubuntu är TeX Live för att senare 
kunna använda \LaTeX.
I GNU/Linux\footnote{%
	Ubuntu är en linuxdistribution, det vill säga det är Linux som används som 
	kärna.
} kallas installationsfiler för paket och de installeras med en pakethanterare.
Det finns flera sätt att installera paket.
Det sätt som rekommenderas enligt \citet{UbuntuDesktop} för nya användare av 
Ubuntu är att använda Ubuntu Software Centre.
Det alternativa sättet är att använda pakethanteraren direkt från terminalen.
Detta görs genom att först starta ett terminalfönster, det vill säga starta 
programmet \emph{Terminal}, därefter skrivs \emph{sudo apt-get install vim} med 
följande som resultat:
\begin{lstlisting}
danbos@ID20809793:assgn-install:(0)$ sudo apt-get install vim
[sudo] password for danbos: 
Reading package lists... Done
Building dependency tree       
Reading state information... Done
vim is already the newest version.
The following package was automatically installed and is no longer required:
  john-data
Use 'apt-get autoremove' to remove them.
0 upgraded, 0 newly installed, 0 to remove and 59 not upgraded.
\end{lstlisting}
(Observera att dollartecknet är en del av prompten.)

Med valfritt tillvägagångssätt, installera följande paket:
\begin{itemize}
	\item texlive,
	\item texlive-latex-extra,
	\item texlive-lang-swedish, och
	\item texlive-lang-ukenglish.
\end{itemize}

Om du vill ha tillgång till dessa program i Windows kan du installera MikTeX 
\citep{Bosk2012lui} och LibreOffice\footnote{%
	URL: \url{http://www.libreoffice.org/}.
}.
För MikTeX kan du hoppa över steget att installera universitetets 
dokumentklasser, detta tas upp i en senare laboration.
Du behöver också programmet 7-zip för Windows, detta finns tillgängligt från 
URL
\begin{center}
	\url{http://www.7-zip.org/}.
\end{center}


\section{Examination}
\label{sec:Examination}
\noindent
Skriv ett enkelt textdokument med hjälp av GEdit i Ubuntu där du sammanfattar 
dina erfarenheter av denna laboration och ger en kort jämförelse mellan Ubuntu 
och Windows, eller MacOS om du har mer erfarenhet av det.


\bibliography{../../itgrund}
\end{document}
