\section{Uppgift}
\label{sec:Exercises}
\noindent
Du ska i denna uppgift skapa en fil som du sedan ska ber�kna en kontrollsumma
f�r och sedan ladda upp denna fil till ditt filutrymme p� skolans server.

\begin{questions}															   
	\question\label{q:Checksums}
	Skapa och verifiera kontrollsummor.
	\begin{parts}
		\part
		Skapa en vanligt textfil (.txt).
		Exempelvis:
		\begin{lstlisting}
$ echo test > textfil.txt
		\end{lstlisting}

		\part
		Ber�kna kontrollsumman f�r filen och testa att kontrollera den.
		Exempelvis i en UNIX-terminal:
		\begin{lstlisting}
$ sha256sum textfil.txt > textfil.txt.sha256
$ sha256sum -c textfil.txt.sha256
		\end{lstlisting}

		\part
		�ndra ett tecken i filens inneh�ll och kontrollera kontrollsumman igen.

		\part
		Vad �r kontrollsummor bra f�r?
	\end{parts}
	\begin{solution}
		\begin{lstlisting}
$ echo test > fil.txt
$ sha256sum fil.txt > fil.txt.sha256
$ sha256sum -c fil.txt.sha256
$ echo >> fil.txt
$ sha256sum -c fil.txt.sha256
		\end{lstlisting}
	\end{solution}

	\question
	Anv�nd SFTP-klienten f�r att ansluta till servern \url{myfiles.miun.se}.
	(Den anv�nder standardporten, port 22.)
	F�rsta g�ngen du ansluter kommer du troligtvis att tillfr�gas om du litar
	p� serverns public key.
	Den b�r vara:
	\begin{center}
		% TODO uppdatera public key f�r myfiles.miun.se.
		RSA: aa:48:69:55:e9:85:1d:37:33:68:94:0a:0f:83:59:2d
	\end{center}
	Om du anv�nder UNIX:
	\begin{lstlisting}
$ sftp anvandarnamn@myfiles.miun.se
...
	\end{lstlisting}
	Om du anv�nder Windows:
	\begin{lstlisting}
> psftp anvandarnamn@myfiles.miun.se
...
	\end{lstlisting}
	Du ska givetvis ers�tta \emph{anvandarnamn} med ditt anv�ndarnamn som du
	anv�nder f�r att logga in till studentportalen.
	N�r du anslutit.
	\begin{parts}
		\part
		Kolla vad du har f�r filer i din hemkatalog p� servern\footnote{%
			Du ser arbetskatalogen med kommandot \emph{pwd}.
			Din hemkatalog �r \emph{/home/anvandarnamn}.
		}.

		\part
		Testa att ladda upp textfilen filen med kontrollsumman fr�n
		\prettyref{q:Checksums} till katalogen \emph{public\_html}.

		\part
		Anv�nd webbl�saren och g� in p� sidan
		\begin{center}
			\url{http://myfiles.miun.se/~anvandarnamn/textfil.txt},
		\end{center}
		om du d�pte textfilen till \emph{textfil.txt}.
	\end{parts}
	\begin{solution}
		\begin{lstlisting}
$ sftp user@myfiles.miun.se
Password:
Connected to myfiles.miun.se.
sftp> pwd
Remote working directory: /home/user
sftp> cd public_html
sftp> put test.txt
sftp> quit
$
		\end{lstlisting}
		Alternativ l�sning med SCP.
		\begin{lstlisting}
$ scp test.txt user@myfiles.miun.se:public_html
Password:
$
		\end{lstlisting}
	\end{solution}
\end{questions}
