\documentclass[11pt,a4paper]{miunasgn}
\usepackage[utf8]{inputenc}
\usepackage[english,swedish]{babel}
\usepackage{url,hyperref}
\usepackage{prettyref,varioref}
\usepackage{natbib}
\usepackage{listings}
\usepackage[today,nofancy]{svninfo}
\usepackage[natbib,varioref,prettyref,listings]{miunmisc}

\svnInfo $Id$
%\printanswers

\courseid{DT001G}
\course{Informationsteknologi grundkurs}
\assignmenttype{Laboration}
\title{Säkerhet}
\author{Daniel Bosk\footnote{%
	E-post: \href{mailto:daniel.bosk@miun.se}{daniel.bosk@miun.se}.
}}
\date{\svnId}

\begin{document}
\maketitle
\thispagestyle{foot}
\tableofcontents

\newtheorem{example}{Exempel}

\section{Introduktion}
\label{sec:Introduktion}
\noindent
\dots


\section{Syfte}
\label{sec:Syfte}
\noindent
Syftet med uppgiften är att du ska lära dig använda en SFTP-klient.
Anledningen är för att kunna logga in och överföra filer på ett säkert sätt.
Du ska
\begin{itemize}
    \item kunna ladda hem filer från en server,
    \item kunna ladda upp filer till en server, och
	\item kunna skapa och verifiera kontrollsummor.
\end{itemize}
Du ska kunna använda verktyg i terminalen för att göra detta.


\section{Läsanvisningar}
\label{sec:Lasanvisningar}
\noindent
\dots


\section{Genomförande}
\label{sec:Genomforande}
\noindent
Använd valfritt operativsystem.
Du behöver en SFTP-klient, ett program för att generera kontrollsummor och en
webbläsare.
\begin{description}
	\item[Windows] För en SFTP-klient kan du installera PuTTY\footnote{%
			URL: \url{ftp://ftp.chiark.greenend.org.uk/users/sgtatham/putty-latest/x86/putty-0.61-installer.exe}.
		}.
		SFTP-klienten används genom kommandot \emph{psftp}.
		PuTTY innehåller även en SSH-klient (kommandot \emph{putty}) och en
		SCP-klient (kommandot \emph{pscp}).
		Det finns även grafiska SFTP-klienter, ett exempel är WinSCP\footnote{%
			URL: \url{http://winscp.net/download/winscp433setup.exe}.
		}

		För att beräkna kontrollsummor kan du använda
		\emph{sha256sum}\footnote{%
			URL: \url{http://blog.nfllab.com/archives/152-Win32-native-md5sum,-sha1sum,-sha256sum-etc..html}.
		} som kompilerats för Windows från samma källkod som för UNIX.
	\item[UNIX] De flesta UNIX-baserade system har som standard en SFTP-klient
		(kommandot \emph{sftp}), SSH-klient (kommandot \emph{ssh}), och
		SCP-klient (kommandot \emph{scp}).
		De flesta UNIX-baserade system har även som standard program för att
		generera och verifiera kontrollsummor.
		Du kan använda kommandot \emph{sha256sum}.
\end{description}


\section{Examination}
\label{sec:Examination}
\noindent
Sammanfatta i en textfil de olika kommandona du använde för att genomföra
uppgifterna.
Döp filen till \emph{dt001g-inlupp3.3.txt}.
Skapa en ny fil innehållandes SHA256-kontrollsumman, döp denna till
\emph{dt001g-inlupp3.3.txt.sha256}.
Ladda upp de två filerna till \emph{public\_html}-katalogen i din hemkatalog på
\url{myfiles.miun.se}.
Låt filerna ligga kvar tills att kursen är avslutad.

I WebCT ska du lämna in kontrollsumman.
Öppna filen och kopiera själva kontrollsumman, klistra därefter in den i fältet
\emph{Inlämnad uppgift}. (Det vill säga inte som kommentar.)
Skriv också om kontrollsumman skapats med SHA256 eller SHA1.
\begin{example}
	Om du använt SHA256 kan du skriva:
	\begin{center}
		\emph{SHA256
		e3b0c44298fc1c149afbf4c8996fb92427ae41e4649b934ca495991b7852b855}
	\end{center}
\end{example}


\bibliography{../../itgrund}
\end{document}
