\section{Genomf�rande}
\label{sec:Instructions}
\noindent
Anv�nd valfritt operativsystem.
Du beh�ver en SFTP-klient, ett program f�r att generera kontrollsummor och en
webbl�sare.
\begin{description}
	\item[Windows] F�r en SFTP-klient kan du installera PuTTY\footnote{%
			URL: \url{ftp://ftp.chiark.greenend.org.uk/users/sgtatham/putty-latest/x86/putty-0.61-installer.exe}.
		}.
		SFTP-klienten anv�nds genom kommandot \emph{psftp}.
		PuTTY inneh�ller �ven en SSH-klient (kommandot \emph{putty}) och en
		SCP-klient (kommandot \emph{pscp}).
		Det finns �ven grafiska SFTP-klienter, ett exempel �r WinSCP\footnote{%
			URL: \url{http://winscp.net/download/winscp433setup.exe}.
		}

		F�r att ber�kna kontrollsummor kan du anv�nda
		\emph{sha256sum}\footnote{%
			URL: \url{http://blog.nfllab.com/archives/152-Win32-native-md5sum,-sha1sum,-sha256sum-etc..html}.
		} som kompilerats f�r Windows fr�n samma k�llkod som f�r UNIX.
	\item[UNIX] De flesta UNIX-baserade system har som standard en SFTP-klient
		(kommandot \emph{sftp}), SSH-klient (kommandot \emph{ssh}), och
		SCP-klient (kommandot \emph{scp}).
		De flesta UNIX-baserade system har �ven som standard program f�r att
		generera och verifiera kontrollsummor.
		Du kan anv�nda kommandot \emph{sha256sum}.
\end{description}
