Som inledande läsning till laborationen kan ni läsa \emph{Just what is \TeX?} 
\citep{TUG2011jwi}.
Därefter, för att komma in i \LaTeX, bör följande kapitel i WikiBooks 
\emph{\LaTeX} \citep{Wikibooks2012l} läsas igenom:
\begin{itemize}
	\item 1.1 Introduction,
	\item 1.2 Basics,
	\item 1.3 Errors and Warnings,
	\item 2.1 Document Structure,
	\item 2.7 List Structures,
	\item 2.8 Tables,
	\item 2.11 Importing Graphics,
	\item 2.12 Floats, Figures and Captions,
	\item 2.13 Footnotes and Margin Notes,
	\item 2.15 Labels and Cross-referencing,
	\item 4.1 Mathematics,
	\item 4.5 Algorithms and Pseudocode, och
	\item 5.3 Bibliography Management.
\end{itemize}

Slutligen ska universitetets rapportmall för examensarbeten som är anpassad för 
\LaTeX{} \citep{Bosk2012etr} läsas igenom.
