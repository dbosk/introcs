\documentclass[11pt,a4paper]{miunasgn}
\usepackage[utf8]{inputenc}
\usepackage[english,swedish]{babel}
\usepackage{url,hyperref}
\usepackage{prettyref,varioref}
\usepackage{natbib}
\usepackage{listings}
\usepackage[today]{rcsinfo}
\usepackage[natbib,varioref,prettyref,listings]{miunmisc}

%\printanswers

\courseid{DT001G}
\course{Informationsteknologi grundkurs}
\assignmenttype{Laboration}
\title{\LaTeX}
\author{Daniel Bosk\footnote{%
	E-post: \href{mailto:daniel.bosk@miun.se}{daniel.bosk@miun.se}.
}}
\date{\today\footnote{%
	\rcsInfo $Id$
}}

\begin{document}
\maketitle
\thispagestyle{foot}
\tableofcontents


\section{Introduktion}
\label{sec:Introduktion}
\noindent
\LaTeX\ är ett dokumentpreparationssystem, det skapades 1985 av Leslie Lamport 
och bygger på \TeX.
Det är implementerat som ett enormt bibliotek av \TeX-makron.

\TeX, i sin tur, skapades av Donald E. Knuth i slutet av 1970-talet när han 
skulle revidera sitt livsverk \emph{The Art of Computer Programming} 
\citep{TUG2011jwi}, en bibel inom datalogin.
Han var missnöjd med hur förlaget hade typsatt den andra upplagan av boken och 
började därför att skriva Metafont och \TeX.
Det är alltså utvecklat för att skriva matematiska och tekniska texter.
Några exempel på vad som kan åstadkommas med \TeX\ kan ses i The \TeX\ Users 
Group (TUG) \emph{The \TeX showcase}\footnote{%
	URL: \url{http://www.tug.org/texshowcase/}.
}.


\section{Syfte}
\label{sec:Syfte}
\noindent
Syftet med laborationen är att lära er att skriva en rapport med universitetets 
dokumentklass för \LaTeX, få insikt i att det finns alternativ till de olika 
officepaketen, det vill säga ordbehandlare.


\section{Läsanvisningar}
\label{sec:Lasanvisningar}
\noindent
Som inledande läsning till laborationen kan ni läsa \emph{Just what is \TeX?} 
\citep{TUG2011jwi}.
Därefter, för att komma in i \LaTeX, bör följande kapitel i WikiBooks 
\emph{\LaTeX} \citep{Wikibooks2012l} läsas igenom:
\begin{itemize}
	\item 1.1 Introduction,
	\item 1.2 Basics,
	\item 1.3 Errors and Warnings, och
	\item 2.1 Document Structure.
\end{itemize}

Slutligen ska universitetets rapportmall för examensarbeten \citep{MiUn2012rft} 
läsas igenom.


\section{Genomförande}
\label{sec:Genomforande}
\noindent
\dots


\section{Examination}
\label{sec:Examination}
\noindent
Ladda upp kompilerad PDF-fil med tillhörande källkod för vardera av de två 
exempeldokumenten till inlämningslådan i lärplattformen.


\bibliography{../../itgrund}
\end{document}
