\documentclass[11pt,a4paper]{miunasgn}
\usepackage[utf8]{inputenc}
\usepackage[english,swedish]{babel}
\usepackage{url,hyperref}
\usepackage{prettyref,varioref}
\usepackage{natbib}
\usepackage{listings}
\usepackage{color}
\usepackage{dtklogos}
\usepackage[today,nofancy]{svninfo}
\usepackage[natbib,varioref,prettyref,listings]{miunmisc}

\svnInfo $Id$
%\printanswers
\lstset{basicstyle=\footnotesize}

\courseid{DT001G}
\course{Informationsteknologi grundkurs}
\assignmenttype{Laboration}
\title{\LaTeX}
\author{Daniel Bosk\footnote{%
	E-post: \href{mailto:daniel.bosk@miun.se}{daniel.bosk@miun.se}.
}}
\date{\svnId}

\begin{document}
\maketitle
\thispagestyle{foot}
\tableofcontents


\section{Introduktion}
\label{sec:Introduktion}
\noindent
\LaTeX\ är ett dokumentpreparationssystem, det skapades 1985 av Leslie Lamport 
och bygger på \TeX.
Det är implementerat som ett väldigt omfattande bibliotek av \TeX-makron.

\TeX, i sin tur, skapades av Donald E. Knuth i slutet av 1970-talet när han 
skulle revidera sitt livsverk \emph{The Art of Computer Programming} 
\citep{TUG2011jwi}, en bibel inom datalogin.
Han var missnöjd med hur förlaget hade typsatt den andra upplagan av boken och 
började därför att skriva Metafont och \TeX.
Det är alltså utvecklat för att skriva matematiska och tekniska texter.
Några exempel på vad som kan åstadkommas med \TeX\ kan ses i The \TeX\ Users 
Group (TUG) \emph{The \TeX\ showcase}\footnote{%
	URL: \url{http://www.tug.org/texshowcase/}.
}.


\section{Syfte}
\label{sec:Syfte}
\noindent
Syftet med laborationen är att lära er att skriva en rapport med universitetets 
dokumentklass för \LaTeX, få insikt i att det finns alternativ till de olika 
officepaketen, det vill säga ordbehandlare.


\section{Läsanvisningar}
\label{sec:Lasanvisningar}
\noindent
Innan du påbörjar laborationen ska du ha läst kapitel 1, 2, 5 och 6.1--6.4 
i \cite{Brookshear2012csa} och avsnitt 2--4 i \cite{pythonkramaren1}.



\section{Genomförande}
\label{sec:Genomforande}
\noindent
Öppna en ny .tex-fil för redigering, exempelvis genom följande kommandorad:
\begin{lstlisting}
$ gedit lab-latex.tex
\end{lstlisting}
Skriv ett kort exempeldokument med \emph{article} som dokumentklass där du 
testar lite olika funktionalitet.

Leta fram kurslitteraturen i Kungliga Bibliotekets katalog Libris\footnote{%
	URL: \url{http://libris.kb.se}.
} och skapa en referens för \BibTeX i en .bib-fil.

Installera universitetets dokumentklasser.
Detta gör du genom att gå in på följande URL:
\begin{center}
	\url{http://ver.miun.se/latexmallar/}.
\end{center}
Läs igenom filen \emph{README} och installera därefter paketen 
\emph{miunmisc.tar.gz}, \emph{miunart.tar.gz} och \emph{miunthes.tar.gz}.

Skriv ett nytt exempeldokument som använder \emph{miunthes} som dokumentklass.
Detta exempeldokument ska
\begin{itemize}
	\item ha minst en figur,
	\item ha minst en tabell med tabellhuvud, några rader och kolumner,
	\item ha minst en referens med hjälp av .bib-filen som skapades tidigare,
	\item ha minst en matematisk formel,
	\item ha minst två huvudrubriker med minst en underrubrik vardera, samt
	\item ha en innehållsförteckning.
\end{itemize}


\section{Examination}
\label{sec:Examination}
\noindent
Ladda upp kompilerad PDF-fil med tillhörande källkod för vardera av de två 
exempeldokumenten till inlämningslådan i lärplattformen.


\bibliography{../../itgrund}
\end{document}
