\documentclass[11pt,a4paper]{miunasgn}
\usepackage[utf8]{inputenc}
\usepackage[english,swedish]{babel}
\usepackage{url,hyperref}
\usepackage{prettyref,varioref}
\usepackage{natbib}
\usepackage{verbatim}
\usepackage{listings}
\usepackage{color}
\usepackage[today,nofancy]{svninfo}
\usepackage[natbib,varioref,prettyref,listings]{miunmisc}

\svnInfo $Id$
%\printanswers

\courseid{DT001G}
\course{Informationsteknologi grundkurs}
\assignmenttype{Laboration}
\title{Terminalen}
\author{Daniel Bosk\footnote{%
	E-post: \href{mailto:daniel.bosk@miun.se}{daniel.bosk@miun.se}.
}}
\date{\svnId}

\lstset{basicstyle=\footnotesize}

\begin{document}
\maketitle
\thispagestyle{foot}
\tableofcontents


\section{Introduktion}
\label{sec:Introduktion}
\noindent
Denna laboration introducerar terminalen i ett UNIX-likt system, i detta fall 
Ubuntu.
Instruktioner om hur motsvarande kan genomföras i Windows finns också, men det 
är obligatoriskt att genomföra laborationen i Ubuntu.

Terminalen är ett kraftfullt verktyg som varit huvudgränssnitt och har 
kontinuerligt utvecklats sedan den första versionen av UNIX släpptes av Bell 
Laboratories år 1969.
Det är ett kommandodrivet gränssnitt där varje kommandon är implementerade som 
enkla program som utför en enkel specialiserad uppgift, det finns dock enstaka 
kommandon som är inbyggda i skalet.
Dessa enkla program kan sedan användas tillsammans för att åstadkomma mer 
komplexa resultat.


\section{Syfte}
\label{sec:Syfte}
\noindent
Syftet med denna inlämningsuppgift är att
\begin{itemize}
	\item få en inblick i textbaserade användargränssnitt,
	\item se sambandet mellan vad som händer i det textbaserade och det
		grafiska gränssnittet,
	\item få en förståelse för skillnaden mellan absoluta och relativa
		sökvägar, samt
	\item få en ökad förståelse för hur filsystemet fungerar.
\end{itemize}


\section{Läsanvisningar}
\label{sec:Lasanvisningar}
\noindent
Innan du påbörjar laborationen ska du ha läst kapitel 1, 2, 5 och 6.1--6.4 
i \cite{Brookshear2012csa} och avsnitt 2--4 i \cite{pythonkramaren1}.



\section{Genomförande}
\label{sec:Genomforande}
\noindent
% XXX skriv om uppgifterna så att studenterna fokuserar mindre på > och >>.
% XXX byt ordning på uppgifterna: fråga om kommandonas funktioner först,
% XXX   använd dem sedan.
Du kommer att behöva använda ett terminalfönster och en grafisk filhanterare.
För att använda dessa i de olika operativsystemen gör du följande:
\begin{description}
	\item[Ubuntu] För Ubuntu kan du använda programmet \emph{Terminal} 
		(gnome-terminal).
		Grafisk filhanterare varierar beroende på system, men leta efter 
		\emph{Files} eller \emph{File manager}.
	\item[UNIX-lika system] Använd valfri terminal, exempelvis \emph{xterm}.
	\item[Windows XP] Kör kommandotolken genom att välja \emph{Kör} från
		Startmenyn, skriv \emph{cmd.exe} och klicka på \emph{OK}.
		Använd \emph{Den här datorn} eller \emph{Utforskaren} som grafisk
		filhanterare.
	\item[Windows Vista/7] Sök efter \emph{kommandotolken} eller
		\emph{powershell} i Startmenyns sökfunktion.
		Använd \emph{Dator} eller \emph{Utforskaren} som grafisk filhanterare.
\end{description}

\begin{questions}
	\uplevel{%
		Starta ett terminalfönster och besvara följande frågor.
	}
	\question\label{q:ls}
	Vad gör kommandot ls(1)?
	Läs manualsidan och testa själv med och utan olika argument.
	(Motsvarigheten i Windows heter \emph{dir}.)
	\begin{solution}
		Lista innehållet i kataloger.
	\end{solution}

	\question\label{q:cd}
	Vad gör kommandot cd?
	Detta kommando är inbyggt i skalet, bash(1), och har därför ingen egen 
	manualsida, se istället bash(1).
	(Motsvarigheten i Windows heter också \emph{cd}.)
	\begin{solution}
		Byt arbetskatalog till den som anges, kan vara relativ eller absolut 
		sökväg.
	\end{solution}

	\question\label{q:mkdir}
	Vad gör kommandona mkdir(1) och rmdir(1)?
	Läs manualsidan och testa själv med och utan olika argument.
	(Motsvarigheten i Windows heter också \emph{mkdir} respektive \emph{rmdir}.)
	\begin{solution}
		Skapar respektive tar bort tomma kataloger.
	\end{solution}

	\question\label{q:rm}
	Vad gör kommandot rm(1)?
	Läs manualsidan och testa själv med och utan olika argument.
	(Motsvarigheten i Windows heter \emph{del}.)
	\begin{solution}
		Tar bort filer och även kataloger om rätt argument används.
	\end{solution}

	\question\label{q:gedit}
	Vad gör följande kommandorad?
	\begin{lstlisting}
$ gedit test.txt
	\end{lstlisting}
	\begin{solution}
		Öppnar filen test.txt för redigering i en GUI-baserad textredigerare.
		Om filen inte finns skapas den när den sparas.
	\end{solution}

	\question\label{q:cat}
	Vad gör kommandot cat(1)?
	Läs manualsidan och testa själv med exempelvis följande kommandorad, om du 
	har kvar \emph{test.txt} från föregående uppgift.
	\begin{lstlisting}
$ cat test.txt
	\end{lstlisting}
	\begin{solution}
		Skriver ut innehållet i filerna som ges som argument till stdout 
		(vanligtvis terminalen).
	\end{solution}

	\question\label{q:echo}
	Vad gör kommandot echo(1)?
	Läs manualsidan och testa själv med och utan olika argument.
	(Motsvarigheten i Windows heter detsamma, men den skiljer sig en aning.)
	\begin{solution}
		Skriver ut alla argument till stdout (vanligtvis terminalen).
	\end{solution}

	\question\label{q:find}
	Vad gör kommandot find(1)?
	Läs manualsidan och testa själv med och utan olika argument.
	(För motsvarigheten i Windows måste kommandot \emph{dir} användas med 
	argumentet \emph{/s}.)
	\begin{solution}
		Söker igenom filstrukturen efter filer som matchar de givna sökkriterierna.
	\end{solution}

	\question\label{q:findless}
	Vad gör följande kommandorad?
	\begin{lstlisting}
$ find / -type d -print | less
	\end{lstlisting}
	Läs manualsidorna för de olika programmen, find(1) respektive less(1), och 
	testa att köra den.
	\begin{solution}
		Sök igenom filstrukturen, med början i roten, efter kataloger.
		Dessa skickas via en pipeline till less(1) som skriver ut en skärmsida 
		i taget.
	\end{solution}

	\question\label{q:fs}
	\begin{parts}\label{qp:relative}
		\part Använd ls(1), pwd(1), cd och eventuella andra program eller kommandon 
		du behöver för att undersöka vad de speciella katalogerna ''.'' respektive 
		''..'' som finns i varje katalog är.
		(Dessa speciella kataloger finns i de flesta filsystem, även i Windows.)
		\begin{solution}
			Katalogen ''.'' hänvisar till den nuvarande arbetskatalogen.
			Katalogen ''..'' hänvisar till föräldern till den nuvarande 
			arbetskatalogen.
		\end{solution}

		\part\label{qp:exploring} Använd cd och ls(1) för att utforska filsystemet.
		Använd den grafiska filhanteraren för att göra samma utforskning parallelt, 
		på så vis tappar du inte lika lätt orienteringen om du är ovan.
	\end{parts}

	\question\label{q:RelativeWWW}
	Använd din webbläsare och gå in på adressen
	\begin{center}
		\url{http://www.miun.se/denna/sida/finns/ej/}.
	\end{center}
	Jämför och förklara skillnaden när du istället går in på adressen
	\begin{center}
		\url{http://www.miun.se/denna/sida/finns/ej/../../../../}.
	\end{center}
	\begin{solution}
		Den andra adressen leder tillbaka till huvudsidan tack vare de relativa
		sökvägarna ''..''.
	\end{solution}
	\begin{solution}
		Den senare URL:en är en relativ sökväg till webbplatsens rotkatalog, vilket 
		gör att webbläsaren visar universitetets startsida.
	\end{solution}

	\uplevel{%
		När skalet tolkar kommandoraden läses första ordet som filnamnet för ett 
		program.
		För att veta vilket program, med sökväg, som ska köras finns ett flertal 
		kataloger inlagda i miljövariabeln \emph{PATH}.
		Dessa kataloger söks igenom i ordning och en programfil med rätt namn 
		eftersöks.
		Kommandot which(1) kan användas för att undersöka vilket program, med 
		fullständig sökväg, som skulle köras om ett visst kommando angavs i skalet.
	}
	\question\label{q:which}
	Undersök sökvägarna till de olika programmen du använt under denna 
	laboration.
	Läs om which(1) för att veta hur programmet fungerar.
	\begin{solution}
		\begin{lstlisting}
(0):danbos@ID20809793:assgn-terminal$ which ls cat mkdir rmdir rm gedit \
> find echo bash pwd
/bin/ls
/bin/cat
/bin/mkdir
/bin/rmdir
/home/danbos/bin/rm
/usr/bin/gedit
/usr/bin/find
/bin/echo
/bin/bash
/bin/pwd
(0):danbos@ID20809793:assgn-terminal$
		\end{lstlisting}
	\end{solution}

\end{questions}


\section{Examination}
\label{sec:Examination}
\noindent
Frågorna i \prettyref{sec:Genomforande} besvaras i textfältet i laborationens 
inlämningslåda i lärplattformen.


\bibliography{../../itgrund}
\end{document}
