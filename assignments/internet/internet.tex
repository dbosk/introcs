\documentclass[11pt,a4paper]{miunasgn}
\usepackage[utf8]{inputenc}
\usepackage[english,swedish]{babel}
\usepackage{url,hyperref}
\usepackage{prettyref,varioref}
\usepackage{natbib}
\usepackage{listings}
\usepackage[amssymb,noams,binary]{SIunits}
\usepackage[today,nofancy]{svninfo}
\usepackage[natbib,varioref,prettyref,listings]{miunmisc}

\svnInfo $Id$
%\printanswers

\courseid{DT001G}
\course{Informationsteknologi grundkurs}
\assignmenttype{Laboration}
\title{Internet}
\author{Daniel Bosk\footnote{%
	E-post: \href{mailto:daniel.bosk@miun.se}{daniel.bosk@miun.se}.
}}
\date{\svnId}

\begin{document}
\maketitle
\thispagestyle{foot}
\tableofcontents


\section{Introduktion}
\label{sec:Introduktion}
\noindent
Denna laboration behandlar nätverk och internet.
Du ska i denna uppgift undersöka din internetanslutning och bekanta dig med
internets adresserings- och domännamnssystem (DNS).
Du kommer att göra några olika mätningar och använda några vanliga verktyg som 
används i nätverkssammanhang.


\section{Syfte}
\label{sec:Syfte}
\noindent
Syftet med denna uppgift är
\begin{itemize}
		\item att ge en inblick i internetanslutningen som används och dess
		tillförlitlighet,
    \item att ge grundläggande förståelse för IPv4-adressering och domännamn,
		samt
	\item att ge erfarenhet av att räkna på överföringshastighet och
		enhetsbyten mellan hastighetsenheter.
\end{itemize}


\section{Läsanvisningar}
\label{sec:Lasanvisningar}
\noindent
% $Id$
Föreläsningen går igenom kursstruktur och organisation.
Den ger en översikt över undervisning och examination.
Den motsvarar således att läsa igenom allt kursmaterial och lite därtill.

Utöver detta täcks även kapitel 0 i \citetitle{Brookshear2012csa} 
\cite{Brookshear2012csa}.
Kapitlet introducerar ämnena datateknik och datalogi (datavetenskap).
Det ger även en historisk överblick av området vilket är bra för att förstå 
varför området är som det är och dess framtida utveckling.



\section{Genomförande}
\label{sec:Genomforande}
\noindent
Genomför och besvara följande uppgifter i den ordning de ges.

\begin{questions}

\fullwidth{
\subsection{IP-adresser}
\noindent
Denna del kommer att fokusera på internetprotokollets (IP) adresser.
}

\uplevel{
Glöm inte att ange referenser.
}
\question
Hur används IP-adresser?
\begin{solution}
	De används för adressering på Internet.
	Varje anslutet system har en unik IP-adress.
	Paket adresseras med dessa adresser, både för mottagare och avsändare.
\end{solution}

\question
Hur många versioner finns av IP idag?
\begin{solution}
	Det finns två: IPv4 och IPv6.
\end{solution}

\question
Vad är de väsentliga skillnaderna mellan dessa versioner?
\begin{solution}
	Främst adressutrymmet.
	IPv4 har adresser som är 32 bitar (4 bytes) lång och
	IPv6 har adresser som är 128 bitar (16 bytes).
	Multicasting är en standardfunktion, istället för tillägg som i
	IPv4.
	IPsec är också en del av IPv6.
\end{solution}

\question
Vilka tre IP-adressområden är reserverade som privata adresser (för LAN etc.)?
\begin{solution}
	192.168.0.0-192.168.255.255 (192.168/16),
	172.16.0.0-172.31.255.255 (172.16/12),
	10.0.0.0-10.255.255.255 (10/8).

	RFC 5735 säger:
	\begin{verbatim}
Address Block       Present Use                Reference
------------------------------------------------------------------
0.0.0.0/8           "This" Network             RFC 1122, Section 3.2.1.3
10.0.0.0/8          Private-Use Networks       RFC 1918
127.0.0.0/8         Loopback                   RFC 1122, Section 3.2.1.3
169.254.0.0/16      Link Local                 RFC 3927
172.16.0.0/12       Private-Use Networks       RFC 1918
192.0.0.0/24        IETF Protocol Assignments  RFC 5736
192.0.2.0/24        TEST-NET-1                 RFC 5737
192.88.99.0/24      6to4 Relay Anycast         RFC 3068
192.168.0.0/16      Private-Use Networks       RFC 1918
198.18.0.0/15       Network Interconnect
                    Device Benchmark Testing   RFC 2544
198.51.100.0/24     TEST-NET-2                 RFC 5737
203.0.113.0/24      TEST-NET-3                 RFC 5737
224.0.0.0/4         Multicast                  RFC 3171
240.0.0.0/4         Reserved for Future Use    RFC 1112, Section 4
255.255.255.255/32  Limited Broadcast          RFC 919, Section 7
                                               RFC 922, Section 7
	\end{verbatim}
\end{solution}

\uplevel{
För att se vilken IP-adress som ditt system använder kan verktygen
ifconfig(8), för UNIX-lika system, och \emph{ipconfig}, för Windows, vara 
användbara.
Dessa används från terminalen.
}
\question
Är din dators IP-adress i något av de privata områdena?
\begin{solution}
	Troligtvis \dots
\end{solution}


\fullwidth{
\subsection{Domännamn och IP-adresser}
\noindent
Följande avsnitt kommer att behandla Internets domännamnssystem (DNS).
}

\uplevel{
För följande uppgift kan kommandot hostname(1) användas för UNIX-lika system 
och \emph{ipconfig} för Windows.
}
\question
Vad har din dator för
\begin{parts}
	\part värddatornamn (hostname)?
	\begin{solution}
		\code{hostname -s}\\
		ID20809793\\
		\code{hostname -f}\\
		ID20809793.itm.miun.se.
	\end{solution}
	\part domännamn eller DNS-suffix?\footnote{%
		Om du inte finner något kan du använda \emph{nslookup} på din externa
		IP-adress.
	}
	\begin{solution}
		\code{hostname -d}
		itm.miun.se
	\end{solution}
	\part finns det en ''top-level domain'' (TLD) i domännamnet, i sådant fall
	vilken?
	\begin{solution}
		se
	\end{solution}
\end{parts}

\uplevel{
För följande uppgifter kan kommandot nslookup(1) användas för både UNIX-lika 
system och Windows.
}
\question
Ange vad följande domännamn har för IP-adress, ange även DNS-servern som
svarade.
\begin{parts}
	\part
	\url{www.iis.se}

	\part
	\url{www.sunet.se}
	
	\part
	\url{ftp.sunet.se}
\end{parts}
\begin{solution}
	\begin{lstlisting}
/home/danbos$ nslookup www.iis.se
Server:         10.2.1.2
Address:        10.2.1.2#53

Non-authoritative answer:
Name:   www.iis.se
Address: 212.247.7.221

/home/danbos$ nslookup www.sunet.se
Server:         10.2.1.2
Address:        10.2.1.2#53

Non-authoritative answer:
www.sunet.se    canonical name = vision.sunet.se.
Name:   vision.sunet.se
Address: 192.36.171.155
Name:   vision.sunet.se
Address: 192.36.171.154

/home/danbos$ nslookup ftp.sunet.se
Server:         10.2.1.2
Address:        10.2.1.2#53

Non-authoritative answer:
Name:   ftp.sunet.se
Address: 194.71.11.69

/home/danbos$ 
	\end{lstlisting}
\end{solution}

\question\label{q:IPLookup}
Vad har följande IP-adresser för domännamn?
\begin{parts}
	\part 193.10.112.13
	\part 130.237.32.143
	\part 18.9.22.69
\end{parts}
\begin{solution}
	\begin{lstlisting}
/home/danbos$ for h in 193.10.112.13 130.237.32.143 18.9.22.69; do \
	nslookup $h; done
Server:         10.2.1.2
Address:        10.2.1.2#53

13.112.10.193.in-addr.arpa      name = dionychanlb1.miun.se.
13.112.10.193.in-addr.arpa      name = miun.se.

Server:         10.2.1.2
Address:        10.2.1.2#53

Non-authoritative answer:
143.32.237.130.in-addr.arpa     name = lvs-vip-6.sys.kth.se.

Authoritative answers can be found from:

Server:         10.2.1.2
Address:        10.2.1.2#53

Non-authoritative answer:
69.22.9.18.in-addr.arpa name = web.mit.edu.

Authoritative answers can be found from:

/home/danbos$ 
	\end{lstlisting}
\end{solution}

\question
Använd tjänsten IP2Location\footnote{%
	URL: \url{http://www.ip2location.com/demo.aspx}.
} och ange var IP-adresserna från \prettyref{q:IPLookup} befinner sig
geografiskt.
Kolla även upp vilken geografisk plats din IP-adress är associerad med.
\begin{solution}
	Sundsvall, Sweden. Stockholm, Sweden. Massachusetts, USA.
\end{solution}

\question
Gå till webbsidan för Stiftelsen för Internetinfrastruktur (IIS)\footnote{%
	URL: \url{http://www.iis.se}.
}, undersök var du ska vända dig om du vill registrera en .se-domän.
\begin{solution}
	Någon av .se's registrar, exempelvis Loopia eller Telia.
\end{solution}

\uplevel{
För följande uppgift kan du använda kommandot whois(1).
För Windows krävs dock tredjepartsverktyg, ett exempel är 
DomainTools\footnote{%
	URL: \url{http://whois.domaintools.com}.
}.
}
\question
Välj tre domännamn och undersök när de registrerades.
\begin{solution}
	\code{whois miun.se} ger 2004,
	kth.se: 1985,
	mit.edu: 1985,
	facebook.com: 1997.
\end{solution}


\fullwidth{
\subsection{Anslutningens hastighet}
\noindent
I detta avsnitt ska du undersöka din uppkopplingshastighet.
}

\question
Vilken hastighet säger ditt operativsystem att du har på din anslutning?
Använd ifconfig(8).
\begin{solution}
	Troligtvis \unit{100}{\mega\bit\per\second}.
\end{solution}

\question
Mät din verkliga uppkopplingshastighet på Post- och Telestyrelsens webbplats
Bredbandskollen\footnote{URL: \url{http://www.bredbandskollen.se}.}.
Ange hastigheterna nedströms och uppströms i
\begin{parts}
	\part \mega\bit\per\second\ (megabit per sekund)
	\part \mega\byte\per\second\ (megabyte per sekund)
\end{parts}
Ange även uträkningarna för omräkningen mellan \mega\bit\per\second\ och 
\mega\byte\per\second.
\begin{solution}
	Troligtvis \unit{10}{\mega\bit\per\second} 
	(\unit{1.25}{\mega\byte\per\second}) nedströms och
	\unit{0.8}{\mega\bit\per\second} (\unit{0.1}{\mega\byte\per\second}) 
	uppströms.
\end{solution}

\question\label{q:AnalyseraUppkoppplingshastigheter}
Skiljer sig hastigheterna mellan det operativsystemet anger, din operatör lovar
och vad bredbandskollen uppmäter, och i sådant fall, varför? (Ta hänsyn till
både ned- och uppströms hastigheter, om de skiljer sig.)
\begin{solution}
	För att operatören inte släpper igenom mer trafik än så.
	Om det är lägre än vad operatören lovar beror det på störningar i
	överföringarna, t.ex. att det går över en brusig telefonledning istället
	för en fiberkabel.
	Dessutom kanske operatören tillhandahåller en asymmetrisk uppkoppling.
	Operativsystemet anger vad nätverkskortet har för ethernetprotokoll.
\end{solution}

\question
Antag att du vill köra en webbserver på din dator.
Förklara vilka av och hur de olika hastigheterna som analyserades i
\prettyref{q:AnalyseraUppkoppplingshastigheter} påverkar den prestanda som
besökarna upplever.
\begin{solution}
	Om det är många som besöker sidan och skickar många begäran kommer
	nedströmshastigheten att påverka (med antagandet att vi har oändlig
	uppströmshastighet).
	Mest kommer uppströmshastigheten att påverka eftersom att det är denna som
	avgör hur snabbt servern kan skicka sidinnehåll till besökarna.
	Varje begäran är förhållandevis mycket liten jämfört med sidans innehåll,
	därför kommer nedströmshastigheten knappt att märkas av för besökarna.
\end{solution}

\question
Beräkna vilken nedladdningshastighet (i \mega\bit\per\second) du skulle behöva 
för att ladda
ner
\begin{parts}
	\part\label{qp:giga}
	\unit{1}{\giga\byte} på kortare tid än en halvtimme (\unit{30}{\min}).

	\part\label{qp:gibi}
	\unit{1}{\gibi\byte} på kortare tid än en halvtimme (\unit{30}{\min}).

	\part
	Hur stor blir skillnaden mellan \prettyref{qp:giga} och
	\prettyref{qp:gibi}?
\end{parts}

Redogör för dina beräkningar.
\begin{solution}
	Vi har först att \[
		\unit{1}{\giga\byte} = \unit{1000}{\mega\byte} =
		\unit{1000\cdot 8}{\mega\bit} = \unit{8000}{\mega\bit}.
	\]
	Därefter har vi att \[
		\unit{30}{\min} = \unit{30\cdot 60}{\second} =
		\unit{1800}{\second}.
	\]
	Vi får således att vi behöver minst en hastighet på \[
		\frac{\unit{8000}{\mega\bit}}{\unit{1800}{\second}} \approx
		\unit{4.45}{\mega\bit\per\second}.
	\]

	För nästa deluppgift har vi först att \[
		\unit{1}{\gibi\byte} = \unit{1024}{\mega\byte} =
		\unit{1024\cdot 8}{\mega\bit}.
	\]
	Det följer då att \[
		\frac{\unit{8192}{\mega\bit}}{\unit{1800}{\second}} \approx
		\unit{4.55}{\mega\bit\per\second}.
	\]

	Den slutgiltiga skillnaden är alltså cirka \unit{0.10}{\mega\bit\per\second}.
\end{solution}

\end{questions}


\section{Examination}
\label{sec:Examination}
\noindent
Besvara de olika frågorna och uppgifterna direkt i textfältet i inlämningslådan 
i lärplattformen.


\bibliography{../../literature}
\end{document}
