% $Id$
% Author:	Daniel Bosk <daniel.bosk@miun.se>
\documentclass[a4paper]{miunasgn}
\usepackage[utf8]{inputenc}
\usepackage[T1]{fontenc}
\usepackage[english,swedish]{babel}
\usepackage[hyphens]{url}
\usepackage{hyperref}
\usepackage{prettyref,varioref}
\usepackage{listings}
\usepackage[today,nofancy]{svninfo}
\usepackage[natbib,style=alphabetic]{biblatex}
\addbibresource{literature.bib}
\usepackage[varioref,prettyref,listings]{miunmisc}

\svnInfo $Id$
%\printanswers

\courseid{DT001G}
\course{Informationsteknologi grundkurs}
\assignmenttype{Laboration}
\title{Installationer}
\author{Daniel Bosk\footnote{%
	Detta verk är tillgängliggjort under licensen Creative Commons 
	Erkännande-DelaLika 2.5 Sverige (CC BY-SA 2.5 SE).
	För att se en sammanfattning och kopia av licenstexten besök URL 
	\url{http://creativecommons.org/licenses/by-sa/2.5/se/}.
}}
\date{\svnId}

\begin{document}
\maketitle
\thispagestyle{foot}
\tableofcontents


\section{Introduktion}
\label{sec:Introduktion}
I denna laboration ska du installera program som du behöver för kursen och 
eventuellt för kommande kurser på din dator.
Du ska även installera ett operativsystem, Ubuntu eller annat UNIX-likt system, 
som kommer att användas under kursens gång.

De programvaror du kommer att installera är följande:
\begin{itemize}
	\item TeX Live för Ubuntu och eventuellt MikTeX för Windows,
	\item eventuellt LibreOffice för Windows (redan installerat för Ubuntu), och
	\item 7-zip för Windows.
\end{itemize}


\section{Syfte}
\label{sec:Syfte}
Syftet med laborationen är:
\begin{itemize}
  % $Id$
% Author:	Daniel Bosk <daniel.bosk@miun.se>
\item få en inblick i textbaserade användargränssnitt,
\item se sambandet mellan vad som händer i det textbaserade och det
	grafiska gränssnittet,
\item få en förståelse för skillnaden mellan absoluta och relativa
	sökvägar, samt
\item få en ökad förståelse för hur filsystemet fungerar.

\end{itemize}


\section{Läsanvisningar}
\label{sec:Lasanvisningar}
Innan du påbörjar laborationen ska du ha läst kapitel 1, 2, 5 och 6.1--6.4 
i \cite{Brookshear2012csa} och avsnitt 2--4 i \cite{pythonkramaren1}.



\section{Genomförande}
\label{sec:Genomforande}
Här följer genomförandet för laborationen.
Det rekommenderas att du läser igenom hela genomförandet innan du sätter igång.

Hela instruktionen är skriven för Ubuntu, men det är valfritt vilket UNIX-likt 
operativsystem som används.
Exempel på andra UNIK-lika system är Debian och OpenBSD.

\subsection{Ubuntu Desktop}
\noindent
Börja med att installera den senaste versionen av Ubuntu Desktop.
Ubuntu går att ladda hem på URL
\begin{center}
	\url{http://www.ubuntu.com/download/}.
\end{center}
Det rekommenderas att du installerar Ubuntu parallelt med redan befintligt 
operativsystem om du har ett sådant, men du väljer själv vilken form du vill 
installera \citep[för detaljer, se][]{UbuntuInstall}.
Att enbart använda LiveCD rekommenderas inte eftersom att det blir problem med 
att installera programvaror och att du inte kan spara filer annat än på 
USB-minnen.

\subsection{Programvaror}
\noindent
När du har loggat in i din Ubuntu-installation är det dags att installera de 
programvaror du behöver.
Den enda programvara du behöver installera i Ubuntu är TeX Live för att senare 
kunna använda \LaTeX.
I GNU/Linux\footnote{%
	Ubuntu är en linuxdistribution, det vill säga det är Linux som används som 
	kärna.
} kallas installationsfiler för paket och de installeras med en pakethanterare.
Det finns flera sätt att installera paket.
Det sätt som rekommenderas enligt \citet{UbuntuDesktop} för nya användare av 
Ubuntu är att använda Ubuntu Software Centre.
Det alternativa sättet är att använda pakethanteraren direkt från terminalen.
Detta görs genom att först starta ett terminalfönster, det vill säga starta 
programmet \emph{Terminal}, därefter skrivs \emph{sudo apt-get install vim}, 
för att installera paketet \emph{vim}, med följande som resultat:
\begin{terminal}
$ sudo apt-get install vim
[sudo] password for danbos: 
Reading package lists... Done
Building dependency tree       
Reading state information... Done
vim is already the newest version.
The following package was automatically installed and is no longer required:
  john-data
Use 'apt-get autoremove' to remove them.
0 upgraded, 0 newly installed, 0 to remove and 59 not upgraded.
$
\end{terminal}
(Observera att dollartecknet är en del av prompten.)

Med valfritt tillvägagångssätt, installera paketet \texttt{texlive-full}.

Om du vill ha tillgång till dessa program i Windows kan du installera MikTeX 
\citep{Bosk2012lui} och LibreOffice\footnote{%
	URL: \url{http://www.libreoffice.org/}.
}.
För MikTeX kan du hoppa över steget att installera universitetets 
dokumentklasser, detta tas upp i en senare laboration.
Du behöver också programmet 7-zip för Windows, detta finns tillgängligt från 
URL
\begin{center}
	\url{http://www.7-zip.org/}.
\end{center}


\section{Examination}
\label{sec:Examination}
Skriv en sammanfattande text med dina erfarenheter av denna laboration och ger 
en kort jämförelse mellan Ubuntu och Windows, eller MacOS om du har mer 
erfarenhet av det än av Windows.
Du kan skriva texten direkt i inlämningslådan i lärplattformen.


\printbibliography
\end{document}
