\documentclass[a4paper]{miunasgn}
\usepackage[utf8]{inputenc}
\usepackage[english]{babel}
\usepackage{csquotes}
\usepackage[hyphens]{url}
\usepackage{hyperref}
\usepackage[today,nofancy]{svninfo}
\usepackage{listings} 
\usepackage{prettyref,varioref}
\usepackage[natbib,style=alphabetic,maxbibnames=99]{biblatex}
\addbibresource{literature.bib}
\usepackage[listings,prettyref,varioref]{miunmisc}

\svnInfo $Id$

\courseid{DT001G}
\course{Informationsteknologi grundkurs}
\assignmenttype{Laboration}
\title{Introduktion till Linuxsystem}
\author{Lennart Franked\footnote{%
    Ursprungsförfattare.
    Denna laboration användes tidigare som repetition i kursen DT043G Avancerad 
    administration av Linuxsystem.
  }
  \and
  Daniel Bosk\footnote{%
    Detta verk är tillgängliggjort under licensen Creative Commons 
    Erkännande-DelaLika 2.5 Sverige (CC BY-SA 2.5 SE).
    För att se en sammanfattning och kopia av licenstexten besök URL 
    \url{http://creativecommons.org/licenses/by-sa/2.5/se/}.
  }
}
\date{\svnId}

\begin{document}
\maketitle
\thispagestyle{foot}
\tableofcontents

\section{Introduktion}
\label{sec:intro}
This laboratory assignment is meant as a preparatory assignment for the following
mandatory labs.

\subsection{Using a previous installation}
If you already have a running GNU/Linux installation that you feel comfortable
using as a laboratory environment and are familiar with the 
basic commands commonly used you can skip this assignment, however 
\emph{you should however still read the chapters listed in 
  \prettyref{sec:reading}}.
For a later laboratory assignment it is also important that you have some 
unpartitioned space on your harddrive. If you run your system in a virtual 
machine you can just create a new virtual harddrive and add to your system at 
a later point.


\section{Aim}
\label{sec:aim}
After completion of this assignment you will have:
\begin{itemize}
  % $Id$
\item genomföra en enklare objektiv undersökning och dra en slutsats av 
resultatet.
\item med akademisk svenska eller engelska skriva en rapport och hålla en 
muntlig presentation av densamma.

\end{itemize}


\section{Reading instructions}
\label{sec:reading}
% $Id$
Föreläsningen går igenom kursstruktur och organisation.
Den ger en översikt över undervisning och examination.
Den motsvarar således att läsa igenom allt kursmaterial och lite därtill.

Utöver detta täcks även kapitel 0 i \citetitle{Brookshear2012csa} 
\cite{Brookshear2012csa}.
Kapitlet introducerar ämnena datateknik och datalogi (datavetenskap).
Det ger även en historisk överblick av området vilket är bra för att förstå 
varför området är som det är och dess framtida utveckling.



\section{Tasks}
\label{sec:work}
In this course we are going to base the lab instructions on the ubuntu-server
operating system which is based on the Debian distribution \citep{debian},
because of this, some instructions will be refering to Debian instead of
Ubuntu.
You can however choose any other flawor of Linux (or BSD) if you like, 
however then you will have to adapt the instructions for your system.
\subsection{Installation}
\label{subsec:Installation}
You can choose to either install your operating system using dual boot or by
using a virtual machine such as virtualbox\citep{vbox}.

For help in installing the Ubuntu server operating system, please refer to the
official documentation for Ubuntu Server\citep{ubuntuinstall}.

During the partitioning of the harddrive make sure to leave 5-10GB unpartitioned
for a later assignment.

\subsubsection{Setting up the network}
\label{subsubsec:Network}
Once your system is up and running you will have to ensure that you have a
working network connection. Debian uses the command \texttt{ip(8)} for managing
network interface related configuration.  e.g. if you would like to view your
current ip configuration for your network interfaces you can use the command:\\
\begin{center}
\texttt{ip address}
\end{center}
for more usage examples see \texttt{ip(8)}.

\texttt{ip(8)} is part of the iproute2 tool kit\citep{iproute2} that will 
eventually replace \texttt{ifconfig(8)} and \texttt{route(8)} so you should
start to familiarize yourself with this command as well as the old \texttt{ifconfig(8)} and \texttt{route(8)} commands.

If you haven't gotten an IP-address you might have to manually configure this.
This is done in the interfaces configuration file which is located at
\texttt{/etc/network/interfaces}. See interfaces(5) for information
how to set up your network card. 

You might also have to configure your DNS-server. This is done in the
 \texttt{/etc/resolv.conf}, see resolv.conf(5) for information on how to 
configure the dns resolver.

\subsubsection{Installation of complementary programs}
\label{subsubsec:complProg}
Once the connection to the Internet is working, we can start to install software
to our server.
In most GNU/Linux systems there is some form of package manager. Debian uses dpkg. 
Since there are alot of packages available to the Debian distribution APT 
(Advanced Packaging Tool) was created for easy access and installation for the
users, to get more information about APT, see apt-get(8).
When using APT its important to first make sure that the package index is
synchronized, for this we use the update command, \texttt{apt-get update}, after
which we can start to install any software that might be needed. 
See \citep{debiangnome} for information on how to install Gnome desktop manager using APT.
\newpage
\subsection{Fundamental UNIX commands}
The following section contains a list of some fundamental UNIX commands that you
need to have a knowledge of for proper usage of the system. See the man page for
each of the commands to get familiar with the usage.
File management
\begin{itemize}
  \item \texttt{ls(1)}
  \item \texttt{cd}
  \item \texttt{pwd(1)}
  \item \texttt{mkdir(1)}
  \item \texttt{rmdir(1)}
  \item \texttt{cp(1)}
  \item \texttt{mv(1)}
  \item \texttt{rm(1)}
  \item \texttt{find(1)}
  \item \texttt{which(1)}
  \item \texttt{touch(1)}
  \item \texttt{stat(1)}
\end{itemize}

Working with files
\begin{itemize}
  \item \texttt{cat(1)}
  \item \texttt{more(1)}
  \item \texttt{less(1)}
  \item \texttt{head(1)}
  \item \texttt{tail(1)}
  \item \texttt{grep(1)}
  \item \texttt{vi(1)}
  \item \texttt{nano(1)}
\end{itemize}

For more commands see coreutils in GNUs info manual by running 
\texttt{info coreutils}.
\section{Examination}
\label{sec:exam}
This lab is not a mandatory lab that needs to be handed in for grading.


\printbibliography
\end{document}
