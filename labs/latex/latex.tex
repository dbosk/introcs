\documentclass[a4paper]{miunasgn}
\usepackage[utf8]{inputenc}
\usepackage[english,swedish]{babel}
\usepackage{url,hyperref}
\usepackage{prettyref,varioref}
\usepackage{listings}
\usepackage{color}
\usepackage{dtklogos}
\usepackage[today,nofancy]{svninfo}
\usepackage[natbib,style=alphabetic,maxbibnames=99]{biblatex}
\addbibresource{literature.bib}
\usepackage[varioref,prettyref,listings]{miunmisc}

\svnInfo $Id$
%\printanswers
\lstset{basicstyle=\footnotesize}

\courseid{DT001G}
\course{Informationsteknologi grundkurs}
\assignmenttype{Laboration}
\title{\LaTeX}
\author{Daniel Bosk\footnote{%
  Detta verk är tillgängliggjort under licensen Creative Commons 
  Erkännande-DelaLika 2.5 Sverige (CC BY-SA 2.5 SE).
  För att se en sammanfattning och kopia av licenstexten besök URL 
  \url{http://creativecommons.org/licenses/by-sa/2.5/se/}.
}}
\date{\svnId}

\begin{document}
\maketitle
\thispagestyle{foot}
\tableofcontents


\section{Introduktion}
\label{sec:Introduktion}
\noindent
\LaTeX\ är ett dokumentpreparationssystem, det skapades 1985 av Leslie Lamport 
och bygger på \TeX.
Det är implementerat som ett väldigt omfattande bibliotek av \TeX-makron.

\TeX, i sin tur, skapades av Donald E. Knuth i slutet av 1970-talet när han 
skulle revidera sitt livsverk \emph{The Art of Computer Programming} 
\citep{TUG2011jwi}, en bibel inom datalogin.
Han var missnöjd med hur förlaget hade typsatt den andra upplagan av boken och 
började därför att skriva Metafont och \TeX.
Det är alltså utvecklat för att skriva matematiska och tekniska texter.
Några exempel på vad som kan åstadkommas med \TeX\ kan ses i The \TeX\ Users 
Group (TUG) \emph{The \TeX\ showcase}\footnote{%
  URL: \url{http://www.tug.org/texshowcase/}.
}.


\section{Syfte}
\label{sec:Syfte}
Syftet med laborationen är följande:
\begin{itemize}
  % $Id$
% Author:	Daniel Bosk <daniel.bosk@miun.se>
\item få en inblick i textbaserade användargränssnitt,
\item se sambandet mellan vad som händer i det textbaserade och det
	grafiska gränssnittet,
\item få en förståelse för skillnaden mellan absoluta och relativa
	sökvägar, samt
\item få en ökad förståelse för hur filsystemet fungerar.

\end{itemize}


\section{Läsanvisningar}
\label{sec:Lasanvisningar}
\noindent
Innan du påbörjar laborationen ska du ha läst kapitel 1, 2, 5 och 6.1--6.4 
i \cite{Brookshear2012csa} och avsnitt 2--4 i \cite{pythonkramaren1}.



\section{Genomförande}
\label{sec:Genomforande}
\noindent
Öppna en ny .tex-fil för redigering, exempelvis genom följande kommandorad:
\begin{terminal}
$ gedit lab-latex.tex
$
\end{terminal}
Skriv ett kort exempeldokument med \emph{article} som dokumentklass där du 
testar lite olika funktionalitet.

Leta fram kurslitteraturen i Kungliga Bibliotekets katalog Libris\footnote{%
  URL: \url{http://libris.kb.se}.
} och skapa en referens för \BibTeX i en .bib-fil.

Skriv ett nytt exempeldokument, fortfarande med \emph{article} som 
dokumentklass.
Detta exempeldokument ska
\begin{itemize}
  \item ha en sammanfattning,
  \item ha en innehållsförteckning,
  \item ha minst en figur,
  \item ha minst en tabell med tabellhuvud, några rader och kolumner,
  \item ha minst en referens med hjälp av .bib-filen som skapades tidigare,
  \item ha minst en matematisk formel,
  \item ha minst två huvudrubriker med minst en underrubrik vardera, samt
  \item ha en innehållsförteckning och en dokumenttitel (\texttt{\textbackslash 
    maketitle}).
\end{itemize}
Dokumentet ska naturligtvis ha en korrekt dokumentstruktur.


\section{Examination}
\label{sec:Examination}
\noindent
Ladda upp till inlämningslådan i lärplattformen en zipfil innehållandes en 
kompilerad PDF-fil med tillhörande källkod för det större exempeldokumentet.
Med källkod menas dokumentet innehållandes TeX-koden och alla figurer och 
BibTeX-filer.


\printbibliography
\end{document}
