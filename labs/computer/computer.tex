% $Id$
% Author:	Daniel Bosk <daniel.bosk@miun.se>
\documentclass[a4paper]{miunasgn}
\usepackage[utf8]{inputenc}
\usepackage[T1]{fontenc}
\usepackage[english,swedish]{babel}
\usepackage[hyphens]{url}
\usepackage{hyperref}
\usepackage{prettyref,varioref}
\usepackage{listings}
\usepackage[today,nofancy]{svninfo}
\usepackage{csquotes}
\usepackage[natbib,style=numeric-comp,maxbibnames=99]{biblatex}
\addbibresource{literature.bib}
\usepackage[varioref,prettyref,listings]{miunmisc}

\svnInfo $Id$
%\printanswers

\courseid{DT001G}
\course{Informationsteknologi grundkurs}
\assignmenttype{Laboration}
\title{Datorn}
\author{Daniel Bosk\footnote{%
	Detta verk är tillgängliggjort under licensen Creative Commons 
	Erkännande-DelaLika 2.5 Sverige (CC BY-SA 2.5 SE).
	För att se en sammanfattning och kopia av licenstexten besök URL 
	\url{http://creativecommons.org/licenses/by-sa/2.5/se/}.
}}
\date{\svnId}

\begin{document}
\maketitle
\thispagestyle{foot}
\tableofcontents


\section{Introduktion}
\label{sec:Introduktion}
Datorn blir en allt viktigare del av samhällets infrastruktur.
Det är idag få delar av samhället som skulle fungera utan datorkomponenter.
Persondatorn är en relativt ny uppfinning, 1980-talet, den börjar nu ersättas 
med en mer portabel persondator, smarttelefonen.
Väldigt många använder datorn dagligen, det är dock ytterst få som vet hur de 
fungerar.
Med denna uppgift ska du råda bot på detta.


\section{Syfte}
\label{sec:Syfte}
\noindent
Syftet med denna laboration är att examinera att du kan:
\begin{itemize}
  % $Id$
% Author:	Daniel Bosk <daniel.bosk@miun.se>
\item få en inblick i textbaserade användargränssnitt,
\item se sambandet mellan vad som händer i det textbaserade och det
	grafiska gränssnittet,
\item få en förståelse för skillnaden mellan absoluta och relativa
	sökvägar, samt
\item få en ökad förståelse för hur filsystemet fungerar.

\end{itemize}
Utöver detta syftar uppgiften till att vara förberedande för kursens avslutande 
projekt.


\section{Läsanvisningar}
\label{sec:Lasanvisningar}
Innan du påbörjar laborationen ska du ha läst kapitel 1, 2, 5 och 6.1--6.4 
i \cite{Brookshear2012csa} och avsnitt 2--4 i \cite{pythonkramaren1}.



\section{Genomförande}
\label{sec:Genomforande}
Börja med att förbereda ett tomt dokument med LaTeX.
Dokumentet ska ha titel, författare och datum.
Det ska även finnas en innehållsförteckning, som för tillfället kommer att vara 
tom.

Texten du är på väg att skriva ska beskriva hur en dator fungerar, den ska 
förklara allt som händer från att användaren trycker på strömbrytaren till att 
denne loggar in på sin webbpost.
Texten ska vara riktad till vardagsmänniskan som bara använder datorn och 
smarttelefonen i vardagen, men den ska vara skriven på ett akademiskt sätt med 
korrekt använd och förklarad terminologi.

Nästa steg är att skapa en disposition.
Skapa de avsnitt du tror att du behöver in din text, detta ger dig en överblick 
över vad du anser viktigt att ta med.
När du har en disposition du känner dig nöjd med kan du börja fylla avsnitten 
med text och delavsnitt.
Du är naturligtvis inte låst till din ursprungliga disposition, denna är bara 
till för att hjälpa dig att strukturera dina tankar innan du börjar skriva.

När du har ett första utkast av texten bör du först läsa igenom den själv en 
gång, från början till slut.
Därefter, när du gjort dina initiala åtgärder, lämnar du texten till någon, som 
helst inte är insatt i området, och ber dem läsa den och kommentera den.


\section{Examination}
\label{sec:Examination}
Du lämnar in ditt dokument (PDF-format) med tillhörande källkod (alla filer som 
behövs för kompilering i en tarboll) i lärplattformen för bedömning.
Ladda upp PDF-dokumentet först och därefter tarbollen med källkoden och 
figurer.

Krav på dokumentet, utöver att du ska visa att du uppfyller målen 
i \prettyref{sec:Syfte}, är följande:
\begin{itemize}
  \item Dokumentet ska vara typsatt med LaTeX, använd dokumentklassen article.
    Typsnittstorlek och marginaler ska vara LaTeX:s standardinställningar.
  \item Dokumentet ska ha en passande titel, författare, datum och det ska inte 
    finnas några sidbrytningar, likt i denna lydelse.
  \item Alla bilder ska vara numrerade figurer med tillhörande figurtexter, de 
    ska även hänvisas till från texten.
  \item Tabeller ska även de vara numrerade med beskrivande texter och 
    hänvisning.
  \item Dokumentet ska ha minst en figur eller tabell.
    Figurerna ska placeras av LaTeX i den övre eller nedre delen av sidan, 
    alternativt på en ''page of floats''.
  \item Dokumentets referenser ska göras med diverse \texttt{\textbackslash 
    cite}-kommandon tillsammans med en BibTeX-databas med alla referenser.
    Ett tips är att använda kursens BibTeX-databas\footnote{%
      Du finner kursens BibTeX-databas på URL 
      \url{http://ver.miun.se/courses/itgrund/literature.bib}.
    }.
  \item Referenser ska skrivas enligt \citetitle{IEEEcitation} 
    \cite{IEEEcitation}.
  \item Dokumentets disposition ska vara tydlig och översiktlig i en 
    innehållsförteckning och dokumentet ska vara skriven på formell akademisk 
    svenska eller engelska.
  \item Omfattningen ska vara omkring sex (6) sidor totalt.
\end{itemize}


\printbibliography
\end{document}
