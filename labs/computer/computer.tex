% $Id$
% Author:	Daniel Bosk <daniel.bosk@miun.se>
\documentclass[a4paper]{miunasgn}
\usepackage[utf8]{inputenc}
\usepackage[T1]{fontenc}
\usepackage[english,swedish]{babel}
\usepackage[hyphens]{url}
\usepackage{hyperref}
\usepackage{prettyref,varioref}
\usepackage{listings}
\usepackage[today,nofancy]{svninfo}
\usepackage{csquotes}
\usepackage[natbib,style=numeric-comp,maxbibnames=99]{biblatex}
\addbibresource{literature.bib}
\usepackage[varioref,prettyref,listings]{miunmisc}

\svnInfo $Id$
%\printanswers

\courseid{DT001G}
\course{Informationsteknologi grundkurs}
\assignmenttype{Laboration}
\title{Datorn}
\author{Daniel Bosk\footnote{%
	Detta verk är tillgängliggjort under licensen Creative Commons 
	Erkännande-DelaLika 2.5 Sverige (CC BY-SA 2.5 SE).
	För att se en sammanfattning och kopia av licenstexten besök URL 
	\url{http://creativecommons.org/licenses/by-sa/2.5/se/}.
}}
\date{\svnId}

\begin{document}
\maketitle
\thispagestyle{foot}
\tableofcontents


\section{Introduktion}
\label{sec:Introduktion}
\noindent
\dots


\section{Syfte}
\label{sec:Syfte}
\noindent
Syftet med denna laboration är att examinera att du kan:
\begin{itemize}
  % $Id$
\item genomföra en enklare objektiv undersökning och dra en slutsats av 
resultatet.
\item med akademisk svenska eller engelska skriva en rapport och hålla en 
muntlig presentation av densamma.

\end{itemize}
Utöver detta syftar uppgiften till att vara förberedande för kursens avslutande 
projekt.


\section{Läsanvisningar}
\label{sec:Lasanvisningar}
\noindent
% $Id$
Föreläsningen går igenom kursstruktur och organisation.
Den ger en översikt över undervisning och examination.
Den motsvarar således att läsa igenom allt kursmaterial och lite därtill.

Utöver detta täcks även kapitel 0 i \citetitle{Brookshear2012csa} 
\cite{Brookshear2012csa}.
Kapitlet introducerar ämnena datateknik och datalogi (datavetenskap).
Det ger även en historisk överblick av området vilket är bra för att förstå 
varför området är som det är och dess framtida utveckling.



\section{Genomförande}
\label{sec:Genomforande}
\noindent
\dots


\section{Examination}
\label{sec:Examination}
\noindent
\dots


\printbibliography
\end{document}
