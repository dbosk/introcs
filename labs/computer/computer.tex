% $Id$
% Author:	Daniel Bosk <daniel.bosk@miun.se>
\documentclass[a4paper]{miunasgn}
\usepackage[utf8]{inputenc}
\usepackage[T1]{fontenc}
\usepackage[english,swedish]{babel}
\usepackage[hyphens]{url}
\usepackage{hyperref}
\usepackage{prettyref,varioref}
\usepackage{listings}
\usepackage[today,nofancy]{svninfo}
\usepackage{csquotes}
\usepackage[natbib,style=numeric-comp,maxbibnames=99]{biblatex}
\addbibresource{literature.bib}
\usepackage[varioref,prettyref,listings]{miunmisc}

\svnInfo $Id$
%\printanswers

\courseid{DT001G}
\course{Informationsteknologi grundkurs}
\assignmenttype{Laboration}
\title{Datorn}
\author{Daniel Bosk\footnote{%
	Detta verk är tillgängliggjort under licensen Creative Commons 
	Erkännande-DelaLika 2.5 Sverige (CC BY-SA 2.5 SE).
	För att se en sammanfattning och kopia av licenstexten besök URL 
	\url{http://creativecommons.org/licenses/by-sa/2.5/se/}.
}}
\date{\svnId}

\begin{document}
\maketitle
\thispagestyle{foot}
\tableofcontents


\section{Introduktion}
\label{sec:Introduktion}
\noindent
\dots


\section{Syfte}
\label{sec:Syfte}
\noindent
Syftet med denna laboration är att examinera att du kan:
\begin{itemize}
  % $Id$
% Author:	Daniel Bosk <daniel.bosk@miun.se>
\item få en inblick i textbaserade användargränssnitt,
\item se sambandet mellan vad som händer i det textbaserade och det
	grafiska gränssnittet,
\item få en förståelse för skillnaden mellan absoluta och relativa
	sökvägar, samt
\item få en ökad förståelse för hur filsystemet fungerar.

\end{itemize}
Utöver detta syftar uppgiften till att vara förberedande för kursens avslutande 
projekt.


\section{Läsanvisningar}
\label{sec:Lasanvisningar}
\noindent
Innan du påbörjar laborationen ska du ha läst kapitel 1, 2, 5 och 6.1--6.4 
i \cite{Brookshear2012csa} och avsnitt 2--4 i \cite{pythonkramaren1}.



\section{Genomförande}
\label{sec:Genomforande}
\noindent
\dots


\section{Examination}
\label{sec:Examination}
\noindent
\dots


\printbibliography
\end{document}
