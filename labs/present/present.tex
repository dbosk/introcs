% $Id$
% Author: Daniel Bosk <daniel.bosk@miun.se>
\documentclass[a4paper,nocourse]{miunasgn}
\usepackage[utf8]{inputenc}
\usepackage[T1]{fontenc}
\usepackage[english,swedish]{babel}
\usepackage[hyphens]{url}
\usepackage{hyperref}
\usepackage{prettyref,varioref}
\usepackage{verbatim}
\usepackage[binary]{SIunits}
\usepackage[today,nofancy]{svninfo}
\usepackage[natbib,style=numeric-comp,maxbibnames=99]{biblatex}
\addbibresource{literature.bib}
\usepackage{pdfpages}
\usepackage[varioref,prettyref]{miunmisc}

\svnInfo $Id$

\courseid{DT001G}
\course{Informationsteknologi grundkurs}
\assignmenttype{Laboration}
\title{Presentationsteknik}
\author{Daniel Bosk\thanks{%
	Detta verk är tillgängliggjort under licensen Creative Commons 
	Erkännande-DelaLika 2.5 Sverige (CC BY-SA 2.5 SE).
	För att se en sammanfattning och kopia av licenstexten besök URL 
	\url{http://creativecommons.org/licenses/by-sa/2.5/se/}.
	}
}
\date{\svnId}

\begin{document}
\maketitle
\thispagestyle{foot}
\tableofcontents


\section{Introduktion}

Det är inte enbart inom utbildning som presentationer används.
Dessa kommer till nytta även utanför akademin, exempelvis för att presentera 
olika lösningsförslag till problem inom företag.
Det är följaktligen av intresse att öva denna färdighet, speciellt då många är 
ovana och det krävs en hel del övning för att bli bekväm med att ge 
presentationer.


\section{Syfte}

Syftet med uppgiften är att examinera att studenten ska kunna:
\begin{itemize}
  % $Id$
% Author:	Daniel Bosk <daniel.bosk@miun.se>
\item få en inblick i textbaserade användargränssnitt,
\item se sambandet mellan vad som händer i det textbaserade och det
	grafiska gränssnittet,
\item få en förståelse för skillnaden mellan absoluta och relativa
	sökvägar, samt
\item få en ökad förståelse för hur filsystemet fungerar.

\end{itemize}


\section{Läsanvisningar}

Innan du påbörjar laborationen ska du ha läst kapitel 1, 2, 5 och 6.1--6.4 
i \cite{Brookshear2012csa} och avsnitt 2--4 i \cite{pythonkramaren1}.



\section{Genomförande}

Din presentation ska handla om lösningen av ett programmeringsproblem.
De problem du har att välja mella är övningarna 42, 43, 44, 45 eller 46 
i \citetitle{pythonkramaren1} \cite{pythonkramaren1}.

När du löst övningen och programmet fungerar som det ska, då skapar du en 5-10 
minuter lång presentation av problemet och lösningen.
Du bör även ta upp för- och nackdelar samt förslag på förbättringar till din 
lösning.


\section{Examination}

Presentationen görs muntligen vid tillfälle för helklass, omfattning på 
presentationen är maximalt 10 minuter.
Efter 10 minuter blir du avbruten och du får göra om din presentation (efter 
att du förkortat den) vid senare tillfälle.
Slides är obligatoriskt för godkänd presentation.
Du måste även ha webbkamera och giltig legitimation vid 
presentationstillfället.


\printbibliography
\end{document}
