% $Id$
% Author:	Daniel Bosk <daniel.bosk@miun.se>
För att genomföra denna laboration bör du ha läst kapitlet om operativsystem 
\citep[kapitel 3]{Brookshear2012csa}.
Det rekommenderas att du även läser kapitel 1, 12.1, 12.5--12.7 och 12.10 
i \citetitle{nemeth2011ual} \cite{nemeth2011ual}.
Den senare boken ingår egentligen inte i kurslitteraturen för denna kurs, men 
då boken ingår i kommande kurser kan ni nyttja den redan nu.

För dokumentation om olika program i UNIX-lika system används kommandot 
\emph{man}.
Det tar namnet på att annat kommando som du vill ha dokumentation för som 
argument.
Om vi till exempel vill ha dokumentation om just \emph{man} självt skriver vi 
\emph{man man} och får resultatet i \prettyref{lst:manman}.
\begin{terminal}[float=tb,label={lst:manman},caption={Listningen av resultatet vid körningen av kommandot \code{man man}.}]
/home/danbos$ man man
MAN(1)						Manual pager utils						MAN(1)

NAME
	   man - an interface to the on-line reference manuals

SYNOPSIS
	   man  [-C  file]  [-d]  [-D]  [--warnings[=warnings]]  [-R encoding] [-L
	   locale] [-m system[,...]] [-M path] [-S list]  [-e  extension]  [-i|-I]
	   [--regex|--wildcard]   [--names-only]  [-a]  [-u]  [--no-subpages]  [-P
	   pager] [-r prompt] [-7] [-E encoding] [--no-hyphenation] [--no-justifi-
	   cation]  [-p  string]  [-t]  [-T[device]]  [-H[browser]] [-X[dpi]] [-Z]
	   [[section] page ...] ...
	   man -k [apropos options] regexp ...
	   man -K [-w|-W] [-S list] [-i|-I] [--regex] [section] term ...
	   man -f [whatis options] page ...
	   man -l [-C file] [-d] [-D] [--warnings[=warnings]]  [-R  encoding]  [-L
	   locale]  [-P  pager]  [-r  prompt]  [-7] [-E encoding] [-p string] [-t]
	   [-T[device]] [-H[browser]] [-X[dpi]] [-Z] file ...
	   man -w|-W [-C file] [-d] [-D] page ...
	   man -c [-C file] [-d] [-D] page ...
	   man [-hV]

DESCRIPTION
	   man is the system's manual pager. Each page argument given  to  man  is
	   normally  the  name of a program, utility or function.  The manual page
	   associated with each of these arguments is then found and displayed.  A
	   section,  if  provided, will direct man to look only in that section of
	   the manual.  The default action is to search in all  of  the  available
	   sections, following a pre-defined order and to show only the first page
	   found, even if page exists in several sections.

	   The table below shows the section numbers of the manual followed by the
	   types of pages they contain.
[...]
/home/danbos$
\end{terminal}
Manualsidorna är indelade i sektioner, denna ges som en siffra inom parentes 
efter namnet på manualsidan -- exempelvis \emph{man(1)}.
För att specificera en särskild sektion anges sektionen innan namnet på 
manualsidan (kommandot) som argument till \emph{man}, exempelvis \emph{man 
1 man}.
Oftast behövs dock inte detta, det är bara när ett uppslagsnamn finns i flera 
sektioner.
Det framgår i \prettyref{lst:manman} att det är man(1) som ges av \emph{man 
man}, alltså samma resultat som vid \emph{man 1 man}.
Dessa manualer finns även tillgängliga online på URL
\begin{center}
	\url{https://www.kernel.org/doc/man-pages/}.
\end{center}
Då finns de att läsa som förberedelse till genomförandet.
De manualsidor som bör läsas översiktligt i förväg är bash(1) och man(1).

För dokumentation om kommandona i Windows kan följande sida användas:
\begin{quote}
	\url{http://technet.microsoft.com/en-us/library/bb490890.aspx}
\end{quote}
