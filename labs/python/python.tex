\documentclass[a4paper]{miunasgn}
\usepackage[utf8]{inputenc}
\usepackage[english,swedish]{babel}
\usepackage{csquotes}
\usepackage[hyphens]{url}
\usepackage{hyperref}
\usepackage[today,nofancy]{svninfo}
\usepackage{listings} 
\usepackage{prettyref,varioref}
\usepackage[natbib,style=numeric-comp,maxnames=99]{biblatex}
\addbibresource{literature.bib}
\usepackage[listings,prettyref,varioref]{miunmisc}

\svnInfo $Id$
%\printanswers

\courseid{DT001G}
\course{Informationsteknologi grundkurs}
\assignmenttype{Laboration}
\title{Programmering med Python}
\author{
  Daniel Bosk\footnote{%
    Detta verk är tillgängliggjort under licensen Creative Commons 
    Erkännande-DelaLika 2.5 Sverige (CC BY-SA 2.5 SE).
    För att se en sammanfattning och kopia av licenstexten besök URL 
    \url{http://creativecommons.org/licenses/by-sa/2.5/se/}.
  }
}
\date{\svnId}

\begin{document}
\maketitle
\thispagestyle{foot}
\tableofcontents

\section{Introduktion}
\label{sec:intro}
Ett av de fundamentala områdena inom datatekniken är programmering, utan denna 
vore datorer helt oanvändbara.
Denna laboration behandlar grundläggan\-de programmering, en kort introduktion 
till detta enorma område.

Programmeringsspråket som används i laborationen är Python.
Detta är ett enkelt skriptspråk som är mycket populärt \cite{PythonWeb} och 
effektivt avseende kodmängd och hastighet \cite{PythonWeb,Shootout}, enkelhet 
att lära \cite{PythonWeb,grandell2006complicate}, öppet och tillgängligt på en 
mängd olika plattformar \cite{PythonWeb}.


\section{Syfte}
\label{sec:aim}
Syftet med denna laboration är att examinera att du kan:
\begin{itemize}
  % $Id$
% Author:	Daniel Bosk <daniel.bosk@miun.se>
\item få en inblick i textbaserade användargränssnitt,
\item se sambandet mellan vad som händer i det textbaserade och det
	grafiska gränssnittet,
\item få en förståelse för skillnaden mellan absoluta och relativa
	sökvägar, samt
\item få en ökad förståelse för hur filsystemet fungerar.

\end{itemize}
Detta görs genom att du får visa att du kan använda grundläggande 
programmeringskonstruktioner;
\begin{itemize}
  \item variabler,
  \item flödeskontrollstrukturer, och
  \item iterationer;
\end{itemize}
och med dessa skriva enklare program för att behandla radbaserad textdata.


\section{Läsanvisningar}
\label{sec:reading}
Innan du påbörjar laborationen ska du ha läst kapitel 1, 2, 5 och 6.1--6.4 
i \cite{Brookshear2012csa} och avsnitt 2--4 i \cite{pythonkramaren1}.



\section{Genomförande}
\label{sec:work}
När du har läst litteraturen, gjort några av övningarna däri och börjar känna 
dig bekant med materialet, då är det dags att ta itu med denna uppgift.

Det som ska åstadkommas är ett program för att hjälpa nätverksdriftstudent\-er 
att repetera inför sina Cisco-prov.
Programmet ska innehålla ett tiotal frågor med tillhörande svar.
När det körs ska det ställa en slumpmässigt vald fråga till användaren, denne 
svarar och får veta om det är rätt eller fel.
Därefter frågar programmet om användaren känner sig klar och vill 
avsluta.
Om användaren vill fortsätta ställs en på nytt slumpmässigt vald fråga och så 
vidare.
Om användaren vill avsluta skrivs statistik ut för denna övningssession.
Statistiken ska innehålla hur många procent av frågorna som blev korrekt 
besvarade och om detta skulle bli godkänt eller ej.
Se \prettyref{lst:output} för ett exempel på utmatning.

\begin{terminal}[float,label={lst:output},caption={Exempel på utmatning från 
  programmet.}]
$ python cisco.py
Hur många portar med terabitstöd har en Cisco X500 SuperPro GrooveRoute?
5
Fel!
Vill du fortsätta? (Ja/Nej)
Ja
Vad heter Ciscos nuvarande VD?
Rutger Blinka
Fel!
Vill du fortsätta? (Ja/Nej)
Nej
Du hade 0% rätt.  Du är underkänd!
$
\end{terminal}

Notera att programmet ska fortsätta att köra tills att användaren väljer att 
avsluta.


\section{Examination}
\label{sec:exam}
Lämna in din källkod tillika programfil (.py-fil) i inlämningslådan 
i lärplattform\-en för bedömning.


\printbibliography
\end{document}
