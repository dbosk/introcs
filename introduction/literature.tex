% $Id$
% Author:	Daniel Bosk <daniel.bosk@miun.se>
\documentclass[a4paper,logo]{miunart}
\usepackage[utf8]{inputenc}
\usepackage{natbib}
\usepackage{prettyref,varioref}
\usepackage[swedish]{babel}
\usepackage{url,hyperref}
\usepackage[today,nofancy]{svninfo}
\usepackage{listings}
\usepackage[natbib,prettyref,varioref,listings]{miunmisc}

\svnInfo $Id$

\title{Kurslitteratur för\\DT001G Informationsteknologi grundkurs}
\author{Daniel Bosk\footnote{%
	Detta verk är tillgängliggjort under licensen Creative Commons 
	Erkännande-DelaLika 2.5 Sverige (CC BY-SA 2.5 SE).
	För att se en sammanfattning och kopia av licenstexten besök URL 
	\url{http://creativecommons.org/licenses/by-sa/2.5/se/}.
}}
\date{\svnId}

\begin{document}
\maketitle
\noindent
Den bok som används som huvudlitteratur på kursen är Brookshears bok 
\emph{Computer Science: An Overview} \citep{Brookshear2012csa}.
Därutöver tillkommer litteratur för en del av laborationerna, denna framgår av 
läsanvisningarna nedan och i respektive laboration.
%Utöver dessa finns
%RFC 791 \citep{RFC791},
%5735 \citep{RFC5735} och
%2460 \citep{RFC2460}
%som referenslitteratur.


\section*{Läsanvisningar}
\noindent
\citet{Brookshear2012csa} tar upp grunden inom datateknik, den är värd att läsa 
i sin helhet -- detta rekommenderas!
De kapitel och avsnitt som behandas i denna kurs är dock enbart kapitlen 0 till 
och med 5 i sina helheter, avsnitten 6.1 till och med 6.4, samt avsnitten 9.1 
och 9.2.

Det går att använda tidigare upplagor av boken, kapitlen som behandlas i kursen 
utgår från den nu senaste upplagan \citep{Brookshear2012csa} och de behandlar 
följande områden:
\begin{itemize}
	\item kapitel 0: introduktion och historia,
	\item kapitel 1: datalagring och -representation,
	\item kapitel 2: datamanipulering, datorarkitektur, programexekvering,
	\item kapitel 3: operativsystem,
	\item kapitel 4: nätverk och internet,
	\item kapitel 5: algoritmer,
	\item avsnitten 6.1 till och med 6.4: programspråk och programmering,
	\item avsnitten 9.1 och 9.2: grunder för databaser och relationsdatabaser.
\end{itemize}

För referens om talsystem och logik kan ni läsa kapitel 2 och 
6 i \emph{Matematik 1c} \citep{Bosk2011m1c}.


\subsection*{Laboration 0: Installationer}
\noindent
% $Id$
Föreläsningen går igenom kursstruktur och organisation.
Den ger en översikt över undervisning och examination.
Den motsvarar således att läsa igenom allt kursmaterial och lite därtill.

Utöver detta täcks även kapitel 0 i \citetitle{Brookshear2012csa} 
\cite{Brookshear2012csa}.
Kapitlet introducerar ämnena datateknik och datalogi (datavetenskap).
Det ger även en historisk överblick av området vilket är bra för att förstå 
varför området är som det är och dess framtida utveckling.



\subsection*{Laboration 1: Terminalen}
\noindent
% $Id$
Föreläsningen går igenom kursstruktur och organisation.
Den ger en översikt över undervisning och examination.
Den motsvarar således att läsa igenom allt kursmaterial och lite därtill.

Utöver detta täcks även kapitel 0 i \citetitle{Brookshear2012csa} 
\cite{Brookshear2012csa}.
Kapitlet introducerar ämnena datateknik och datalogi (datavetenskap).
Det ger även en historisk överblick av området vilket är bra för att förstå 
varför området är som det är och dess framtida utveckling.



\subsection*{Laboration 2: \LaTeX}
\noindent
Som inledande läsning till laborationen kan ni läsa \emph{Just what is \TeX?} 
\citep{TUG2011jwi}.
Därefter, för att komma in i \LaTeX, bör följande kapitel i WikiBooks 
\emph{\LaTeX} \citep{Wikibooks2012l} läsas igenom:
\begin{itemize}
	\item 1.1 Introduction,
	\item 1.2 Basics,
	\item 1.3 Errors and Warnings,
	\item 2.1 Document Structure,
	\item 2.7 List Structures,
	\item 2.8 Tables,
	\item 2.11 Importing Graphics,
	\item 2.12 Floats, Figures and Captions,
	\item 2.13 Footnotes and Margin Notes,
	\item 2.15 Labels and Cross-referencing,
	\item 4.1 Mathematics,
	\item 4.5 Algorithms and Pseudocode, och
	\item 5.3 Bibliography Management.
\end{itemize}

Slutligen ska universitetets rapportmall för examensarbeten som är anpassad för 
\LaTeX{} \citep{Bosk2012etr} läsas igenom.



\subsection*{Laboration 3: Datorn}
\noindent
% $Id$
Föreläsningen går igenom kursstruktur och organisation.
Den ger en översikt över undervisning och examination.
Den motsvarar således att läsa igenom allt kursmaterial och lite därtill.

Utöver detta täcks även kapitel 0 i \citetitle{Brookshear2012csa} 
\cite{Brookshear2012csa}.
Kapitlet introducerar ämnena datateknik och datalogi (datavetenskap).
Det ger även en historisk överblick av området vilket är bra för att förstå 
varför området är som det är och dess framtida utveckling.



\subsection*{Laboration 4: Internet}
\noindent
% $Id$
Föreläsningen går igenom kursstruktur och organisation.
Den ger en översikt över undervisning och examination.
Den motsvarar således att läsa igenom allt kursmaterial och lite därtill.

Utöver detta täcks även kapitel 0 i \citetitle{Brookshear2012csa} 
\cite{Brookshear2012csa}.
Kapitlet introducerar ämnena datateknik och datalogi (datavetenskap).
Det ger även en historisk överblick av området vilket är bra för att förstå 
varför området är som det är och dess framtida utveckling.



\subsection*{Laboration 5: Säkerhet}
\noindent
% $Id$
Föreläsningen går igenom kursstruktur och organisation.
Den ger en översikt över undervisning och examination.
Den motsvarar således att läsa igenom allt kursmaterial och lite därtill.

Utöver detta täcks även kapitel 0 i \citetitle{Brookshear2012csa} 
\cite{Brookshear2012csa}.
Kapitlet introducerar ämnena datateknik och datalogi (datavetenskap).
Det ger även en historisk överblick av området vilket är bra för att förstå 
varför området är som det är och dess framtida utveckling.



\subsection*{Laboration 6: SFTP}
\noindent
% $Id$
Föreläsningen går igenom kursstruktur och organisation.
Den ger en översikt över undervisning och examination.
Den motsvarar således att läsa igenom allt kursmaterial och lite därtill.

Utöver detta täcks även kapitel 0 i \citetitle{Brookshear2012csa} 
\cite{Brookshear2012csa}.
Kapitlet introducerar ämnena datateknik och datalogi (datavetenskap).
Det ger även en historisk överblick av området vilket är bra för att förstå 
varför området är som det är och dess framtida utveckling.



\bibliography{../literature}
\end{document}
