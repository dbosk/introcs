\documentclass[a4paper,11pt,logo]{miunart}
\usepackage[utf8]{inputenc}
\usepackage{natbib}
\usepackage{prettyref,varioref}
\usepackage[swedish]{babel}
\usepackage{url,hyperref}
\usepackage[today,nofancy]{svninfo}
\usepackage{listings}
\usepackage[natbib,prettyref,varioref,listings]{miunmisc}

\svnInfo $Id$

\title{Kurslitteratur för\\DT001G Informationsteknologi grundkurs}
\author{Daniel Bosk\footnote{%
	E-post: \href{mailto:daniel.bosk@miun.se}{daniel.bosk@miun.se}.
}}
\date{\svnId}

\begin{document}
\maketitle
\noindent
Den bok som används som huvudlitteratur på kursen är Brookshears bok 
\emph{Computer Science: An Overview} \citep{Brookshear2012csa}.
Därutöver tillkommer litteratur för en del av laborationerna, denna framgår av 
läsanvisningarna nedan och i respektive laboration.
%Utöver dessa finns
%RFC 791 \citep{RFC791},
%5735 \citep{RFC5735} och
%2460 \citep{RFC2460}
%som referenslitteratur.


\section*{Läsanvisningar}
\noindent
\citet{Brookshear2012csa} tar upp grunden inom datateknik, den är värd att läsa 
i sin helhet -- detta rekommenderas!
De kapitel och avsnitt som behandas i denna kurs är dock enbart kapitlen 0 till 
och med 5 i sina helheter, avsnitten 6.1 till och med 6.4, samt avsnitten 9.1 
och 9.2.

Det går att använda tidigare upplagor av boken, kapitlen som behandlas i kursen 
utgår från den nu senaste upplagan \citep{Brookshear2012csa} och de behandlar 
följande områden:
\begin{itemize}
	\item kapitel 0: introduktion och historia,
	\item kapitel 1: datalagring och -representation,
	\item kapitel 2: datamanipulering, datorarkitektur, programexekvering,
	\item kapitel 3: operativsystem,
	\item kapitel 4: nätverk och internet,
	\item kapitel 5: algoritmer,
	\item avsnitten 6.1 till och med 6.4: programspråk och programmering,
	\item avsnitten 9.1 och 9.2: grunder för databaser och relationsdatabaser.
\end{itemize}


\subsection*{Laboration 0: Installationer}
\noindent
För att genomföra denna laboration bör du ha läst kapitlet om operativsystem 
och bootprocessen \citep[kapitel 3]{Brookshear2012csa}.

Innan du genomför laborationen bör du också läsa igenom dokumentationen för 
installationen av Ubuntu \citep{UbuntuInstall}, detta är inte för att 
installationen är svår att genomföra utan för att du ska kunna fundera igenom 
dina beslut på förhand.

När Ubuntu väl är installerat finns dokumentationen \citep{UbuntuDesktop} som 
stöd för att börja använda systemet.
Det kan vara bra att orientera sig i denna för senare enkelt hitta vid behov.
Det rekommenderas att läsa igenom de första fem avsnitten -- \emph{Welcome to 
Ubuntu 12.04} till och med \emph{Log out, power off, switch users} -- innan 
installationen.
Du ska även läsa om hur program installeras \citep[se Install additional 
software]{UbuntuDesktop}.

Om du använder Windows och vill testa att installera programmen även där finns 
en instruktion för att installera \LaTeX\ för Windows \citep{Bosk2012lui}.
Hur \LaTeX\ installeras under Ubuntu täcks senare i denna lydelse.



\subsection*{Laboration 1: Terminalen}
\noindent
För att genomföra denna laboration bör du ha läst kapitlet om operativsystem 
\citep[kapitel 3]{Brookshear2012csa}.

För dokumentation om olika program i UNIX-lika system används kommandot 
\emph{man}.
Det tar namnet på att annat kommando som du vill ha dokumentation för som 
argument.
Om vi till exempel vill ha dokumentation om just \emph{man} självt skriver vi 
\emph{man man} och får resultat i \prettyref{lst:manman}.
\begin{lstlisting}[float=tbp,caption={Resultatet av kommandoraden \emph{man 
	man}.},label={lst:manman}]
/home/danbos$ man man
MAN(1)						Manual pager utils						MAN(1)

NAME
	   man - an interface to the on-line reference manuals

SYNOPSIS
	   man  [-C  file]  [-d]  [-D]  [--warnings[=warnings]]  [-R encoding] [-L
	   locale] [-m system[,...]] [-M path] [-S list]  [-e  extension]  [-i|-I]
	   [--regex|--wildcard]   [--names-only]  [-a]  [-u]  [--no-subpages]  [-P
	   pager] [-r prompt] [-7] [-E encoding] [--no-hyphenation] [--no-justifi-
	   cation]  [-p  string]  [-t]  [-T[device]]  [-H[browser]] [-X[dpi]] [-Z]
	   [[section] page ...] ...
	   man -k [apropos options] regexp ...
	   man -K [-w|-W] [-S list] [-i|-I] [--regex] [section] term ...
	   man -f [whatis options] page ...
	   man -l [-C file] [-d] [-D] [--warnings[=warnings]]  [-R  encoding]  [-L
	   locale]  [-P  pager]  [-r  prompt]  [-7] [-E encoding] [-p string] [-t]
	   [-T[device]] [-H[browser]] [-X[dpi]] [-Z] file ...
	   man -w|-W [-C file] [-d] [-D] page ...
	   man -c [-C file] [-d] [-D] page ...
	   man [-hV]

DESCRIPTION
	   man is the system's manual pager. Each page argument given  to  man  is
	   normally  the  name of a program, utility or function.  The manual page
	   associated with each of these arguments is then found and displayed.  A
	   section,  if  provided, will direct man to look only in that section of
	   the manual.  The default action is to search in all  of  the  available
	   sections, following a pre-defined order and to show only the first page
	   found, even if page exists in several sections.

	   The table below shows the section numbers of the manual followed by the
	   types of pages they contain.
[...]
/home/danbos$
\end{lstlisting}
Manualsidorna är indelade i sektioner, denna ges som en siffra inom parentes 
efter namnet på manualsidan -- exempelvis \emph{man(1)}.
För att specificera en särskild sektion anges sektionen innan namnet på 
manualsidan (kommandot) som argument till \emph{man}, exempelvis \emph{man 
1 man}.
Oftast behövs dock inte detta, det är bara när ett uppslagsnamn finns i flera 
sektioner.
Det framgår i \prettyref{lst:manman} att det är man(1) som ges av \emph{man 
man}, alltså samma resultat som vid \emph{man 1 man}.
Dessa manualer finns även tillgängliga online på URL
\begin{center}
	\url{https://www.kernel.org/doc/man-pages/}.
\end{center}
Då finns de att läsa som förberedelse till genomförandet.
De manualsidor som bör läsas översiktligt i förväg är bash(1) och man(1).

För dokumentation om kommandona i Windows kan följande sida användas:
\begin{quote}
	\url{http://technet.microsoft.com/en-us/library/bb490890.aspx}
\end{quote}



\subsection*{Laboration 2: \LaTeX}
\noindent
Som inledande läsning till laborationen kan ni läsa \emph{Just what is \TeX?} 
\citep{TUG2011jwi}.
Därefter, för att komma in i \LaTeX, bör följande kapitel i WikiBooks 
\emph{\LaTeX} \citep{Wikibooks2012l} läsas igenom:
\begin{itemize}
	\item 1.1 Introduction,
	\item 1.2 Basics,
	\item 1.3 Errors and Warnings,
	\item 2.1 Document Structure,
	\item 2.7 List Structures,
	\item 2.8 Tables,
	\item 2.11 Importing Graphics,
	\item 2.12 Floats, Figures and Captions,
	\item 2.13 Footnotes and Margin Notes,
	\item 2.15 Labels and Cross-referencing,
	\item 4.1 Mathematics,
	\item 4.5 Algorithms and Pseudocode, och
	\item 5.3 Bibliography Management.
\end{itemize}

Slutligen ska universitetets rapportmall för examensarbeten som är anpassad för 
\LaTeX{} \citep{Bosk2012etr} läsas igenom.



\subsection*{Laboration 3: Datorn}
\noindent
\dots



\subsection*{Laboration 4: Internet}
\noindent
Inför denna laboration bör du ha läst kapitlet om nätverk och internet 
i kurslitteraturen \citep[kapitel 4]{Brookshear2012csa}.
Utöver denna kan
RFC 791 \citep{RFC791},
5735 \citep{RFC5735} och
2460 \citep{RFC2460}
vara bra att ha som referenslitteratur för några delar av laborationen.



\subsection*{Laboration 5: Säkerhet}
\noindent
% $Id$
% Author:	Daniel Bosk <daniel.bosk@miun.se>
Du ska inför laborationen ha läst igenom avsnittet om säkerhet för nätverk och 
internet i kurslitteraturen \citep[avsnitt 4.5]{Brookshear2012csa}.
Därefter kan du läsa Post- och Telestyrelsens (PTS) \emph{Tolv goda råd} 
\citep{PTStgr}.

Två intressanta artiklar om lösenord ska också läsas, en från 2011 
\citep{Hunt2011abs} och en från 2012 \citep{Cluley2012twp}.

För att få en bra perspektiv på vad som enkelt kan hända ska du läsa om 
Wired-journalisten Mat Honans digitala livs öde \citep{Honan2012haa} samt några 
tips om att undvika samma öde \citep{Zetter2012hnt}.

En annan viktig del av säkerheten är informationen som finns tillgänglig om 
dig, en transkriberad intervju av Steven Cherry \citep{Cherry2012fky} ska också 
läsas.
Samy Kamkars föreläsning \emph{How I met your girlfriend} \citep{Kamkar2010him} 
handlar också om hur känslig denna information är, den ska också ses i sin 
helhet.



\subsection*{Laboration 6: SFTP}
\noindent
Inför laborationen ska du läsa manualsidan för sha256sum(1) och sftp(1).



\bibliography{../itgrund}
\end{document}
