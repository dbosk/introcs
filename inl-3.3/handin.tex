\section{Redovisning}
\label{sec:Handin}
\noindent
Sammanfatta i en textfil de olika kommandona du anv�nde f�r att genomf�ra
uppgifterna.
D�p filen till \emph{dt001g-inlupp3.3.txt}.
Skapa en ny fil inneh�llandes SHA256-kontrollsumman, d�p denna till
\emph{dt001g-inlupp3.3.txt.sha256}.
Ladda upp de tv� filerna till \emph{public\_html}-katalogen i din hemkatalog p�
\url{myfiles.miun.se}.
L�t filerna ligga kvar tills att kursen �r avslutad.

I WebCT ska du l�mna in kontrollsumman.
�ppna filen och kopiera sj�lva kontrollsumman, klistra d�refter in den i f�ltet
\emph{Inl�mnad uppgift}. (Det vill s�ga inte som kommentar.)
Skriv ocks� om kontrollsumman skapats med SHA256 eller SHA1.
\begin{example}
	Om du anv�nt SHA256 kan du skriva:
	\begin{center}
		\emph{SHA256
		e3b0c44298fc1c149afbf4c8996fb92427ae41e4649b934ca495991b7852b855}
	\end{center}
\end{example}
