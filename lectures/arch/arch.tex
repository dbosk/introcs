% $Id$
% Author:  Daniel Bosk <daniel.bosk@miun.se>
\documentclass{beamer}
\usepackage[utf8]{inputenc}
\usepackage[T1]{fontenc}
\usepackage[english,swedish]{babel}
\usepackage{url}
\usepackage{varioref,prettyref}
\usepackage{graphicx}
\usepackage[today,nofancy]{svninfo}
\usepackage{natbib}
\usepackage{booktabs}
\usepackage[varioref,prettyref]{miunmisc}
\setcitestyle{numbers,square}
\bibliographystyle{swealpha}

\renewcommand{\qedsymbol}{Q.E.D.}
\newcommand{\N}{\mathbb{N}}
\newcommand{\Q}{\mathbb{Q}}
\newcommand{\R}{\mathbb{R}}
\newcommand{\Z}{\mathbb{Z}}
\newcommand{\powerset}{\mathcal{P}}
\newcommand{\U}{\mathcal{U}}
\newcommand{\V}{\mathcal{V}}
\DeclareMathOperator{\card}{card}
\DeclareMathOperator{\tnot}{icke}
\DeclareMathOperator{\tor}{eller}
\DeclareMathOperator{\tand}{och}
\DeclareMathOperator{\lequiv}{\Longleftrightarrow}
\DeclareMathOperator{\congruent}{\equiv}
\DeclareMathOperator{\xor}{\oplus}

\providetranslation[to=swedish]{Theorem}{Sats}
\providetranslation[to=swedish]{Corollary}{Korollarium}
\theoremstyle{definition}
\newenvironment{axiom}[1]{\begin{block}{Postulat (#1)}}{\end{block}}
\providetranslation[to=swedish]{Example}{Exempel}
\newtheorem{exercise}{Övning}
\theoremstyle{remark}
\newtheorem{remark}{Anmärkning}

\svnInfo $Id$

\mode<presentation>
{
  \usetheme{Frankfurt}
  \setbeamercovered{transparent}
  \usecolortheme{seagull}
}
\setbeamertemplate{footline}{\insertframenumber}

\title{%
  Datorarkitektur
}
\author{Daniel Bosk\footnote{%
  \tiny
  Detta verk är tillgängliggjort under licensen Creative Commons 
  Erkännande-DelaLika 2.5 Sverige (CC BY-SA 2.5 SE).
  För att se en sammanfattning och kopia av licenstexten besök URL 
  \url{http://creativecommons.org/licenses/by-sa/2.5/se/}.
}}
\institute[MIUN ITM]{%
  %Department of Information and Communication Systems (ICS),\\
  %Mid Sweden University, Sundsvall.
  %
  Avdelningen för informations- och kommunikationssytem (IKS),\\
  Mittuniversitetet, Sundsvall.
}
\date{\svnId}

\pgfdeclareimage[height=0.65cm]{university-logo}{MU_logotyp_int_CMYK.pdf}
\logo{\pgfuseimage{university-logo}}

\AtBeginSection[]{%
  \begin{frame}<beamer>{Översikt}
    \tableofcontents[currentsection]
  \end{frame}
}

\begin{document}

\begin{frame}
  \titlepage
\end{frame}

\begin{frame}{Översikt}
  \tableofcontents
  % You might wish to add the option [pausesections]
\end{frame}

\begin{frame}
  % $Id$
Föreläsningen går igenom kursstruktur och organisation.
Den ger en översikt över undervisning och examination.
Den motsvarar således att läsa igenom allt kursmaterial och lite därtill.

Utöver detta täcks även kapitel 0 i \citetitle{Brookshear2012csa} 
\cite{Brookshear2012csa}.
Kapitlet introducerar ämnena datateknik och datalogi (datavetenskap).
Det ger även en historisk överblick av området vilket är bra för att förstå 
varför området är som det är och dess framtida utveckling.

\end{frame}


% Since this a solution template for a generic talk, very little can
% be said about how it should be structured. However, the talk length
% of between 15min and 45min and the theme suggest that you stick to
% the following rules:  

% - Exactly two or three sections (other than the summary).
% - At *most* three subsections per section.
% - Talk about 30s to 2min per frame. So there should be between about
%   15 and 30 frames, all told.


\section{Processorn}

\subsection{CPU}

\begin{frame}{\insertsubsectionhead}
\end{frame}

\subsection{Maskinkod}

\begin{frame}{\insertsubsectionhead}
\end{frame}

\begin{frame}{\insertsubsectionhead}{Representera data}
  \begin{description}
    \item[Little endian] \dots
    \item[Big endian] \dots
  \end{description}
\end{frame}

\subsection{Programexekvering}

\begin{frame}{\insertsubsectionhead}
\end{frame}


\section{Minne}

\subsection{Primärminne}

\begin{frame}{\insertsubsectionhead}{Kapacitet}
\end{frame}

\subsection{Sekundärminne}

\begin{frame}{\insertsubsectionhead}{Magnetiska system}
\end{frame}

\begin{frame}{\insertsubsectionhead}{Optiska system}
\end{frame}

\begin{frame}{\insertsubsectionhead}{Flashbaserade system}
\end{frame}

\subsection{Representera data}

\begin{frame}{\insertsubsectionhead}{Bitmönster och teckenkodning}
\end{frame}

\begin{frame}{\insertsubsectionhead}{Tvåkomplementsnotation}
\end{frame}

\begin{frame}{\insertsubsectionhead}{Flyttal}
\end{frame}

\subsection{Filsystem}

\begin{frame}{\insertsubsectionhead}{Filer}
\end{frame}


\section{Andra enheter}

\subsection{Kontrollerenheter}

\begin{frame}{\insertsubsectionhead}
\end{frame}


%%%%%%%%%%%%%%%%%%%%%%

\begin{frame}{Referenser}
  \bibliography{literature}
\end{frame}

\end{document}

