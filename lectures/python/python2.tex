% $Id$
% Author:  Daniel Bosk <daniel.bosk@miun.se>
\documentclass{beamer}
\usepackage[utf8]{inputenc}
\usepackage[T1]{fontenc}
\usepackage[english,swedish]{babel}
\usepackage{url}
\usepackage{varioref,prettyref}
\usepackage{graphicx}
\usepackage[today,nofancy]{svninfo}
\usepackage{natbib}
\usepackage{booktabs}
\usepackage{listings}
\usepackage{multicol}
\usepackage[listings,varioref,prettyref]{miunmisc}
\setcitestyle{numbers,square}
\bibliographystyle{swealpha}

\lstset{language=python}

\svnInfo $Id$

\mode<presentation>
{
  \usetheme{Frankfurt}
  \setbeamercovered{transparent}
  \usecolortheme{seagull}
}
\setbeamertemplate{footline}{\insertframenumber}

\title{%
  Python, del 2
}
\author{Daniel Bosk\footnote{%
  \tiny
  Detta verk är tillgängliggjort under licensen Creative Commons 
  Erkännande-DelaLika 2.5 Sverige (CC BY-SA 2.5 SE).
  För att se en sammanfattning och kopia av licenstexten besök URL 
  \url{http://creativecommons.org/licenses/by-sa/2.5/se/}.
}}
\institute[MIUN IKS]{%
  %Department of Information and Communication Systems (ICS),\\
  %Mid Sweden University, Sundsvall.
  %
  Avdelningen för informations- och kommunikationssytem (IKS),\\
  Mittuniversitetet, Sundsvall.
}
\date{\svnId}

\pgfdeclareimage[height=0.65cm]{university-logo}{MU_logotyp_int_CMYK.pdf}
\logo{\pgfuseimage{university-logo}}

\AtBeginSection[]{%
  \begin{frame}<beamer>{Översikt}
    \begin{multicols}{2}
      \tableofcontents[currentsection]
    \end{multicols}
  \end{frame}
}

\begin{document}

\begin{frame}
  \titlepage
\end{frame}

\begin{frame}{Översikt}
  \begin{multicols}{2}
    \tableofcontents
    % You might wish to add the option [pausesections]
  \end{multicols}
\end{frame}

%\begin{frame}
%  Innan du påbörjar laborationen ska du ha läst kapitel 1, 2, 5 och 6.1--6.4 
i \cite{Brookshear2012csa} och avsnitt 2--4 i \cite{pythonkramaren1}.

%\end{frame}


% Since this a solution template for a generic talk, very little can
% be said about how it should be structured. However, the talk length
% of between 15min and 45min and the theme suggest that you stick to
% the following rules:  

% - Exactly two or three sections (other than the summary).
% - At *most* three subsections per section.
% - Talk about 30s to 2min per frame. So there should be between about
%   15 and 30 frames, all told.


\section{Logik}

\subsection{Datatypen bool}

\begin{frame}{\insertsubsectionhead}
  \begin{itemize}
    \item \code{bool} är en datatyp.

    \item Till skillnad från t.ex.\ \code{int}, kan ta enbart två olika värden.
      \begin{itemize}
        \item \code{True} och \code{False} (skrivs också vanligtvis som 
          1 respektive 0)\footnote{Detta genom typkonvertering.}.
      \end{itemize}

  \end{itemize}
\end{frame}

\subsection{Logiska operatorer}

\begin{frame}{\insertsubsectionhead}
  \begin{itemize}
    \item På denna datatyp \code{bool} finns några operatorer definierade, likt 
      \code{+ - * /} för flyttal.
    \item Dessa operatorer är \code{and or not}.
    \item \code{not} är en operator som skiljer sig något från de tidigare; den 
      är unär, d.v.s. den opererar enbart på en operand.
    \item (Till skillnad från binära operatorer som opererar på två operander.)
  \end{itemize}
\end{frame}

\begin{frame}{\insertsubsectionhead}
  \begin{table}[h]
    \begin{tabular}{l|l}
      Operation & Resultat \\
      \hline
      not True & False \\
      not False & True
    \end{tabular}
    \label{tbl:not}
    \caption{Sanningtabell för Not}
  \end{table}

  \begin{table}[h]
    \begin{tabular}{l|ll}
      and & True & False \\
      \hline
      True & True & False \\
      False & False & False
    \end{tabular}
    \begin{tabular}{l|ll}
      or & True & False \\
      \hline
      True & True & True \\
      False & True & False
    \end{tabular}
    \label{tbl:andor}
    \caption{Sanningstabeller för And och Or}
  \end{table}
\end{frame}

\begin{frame}[fragile,allowframebreaks]{\insertsubsectionhead}
  Några exempel från Python-tolken:
  \begin{terminal}
Python 2.2.3 (#1, Jan  5 2005, 16:36:30)
[GCC 3.4.2] on sunos5
Type "help", "copyright", "credits" or "license" for more information.
>>> True or False
True
>>> True and False
False
>>> not True
False
>>> not False
True
>>> not True and False
False
>>> not (True and False)
True
>>> not False and True
True
>>> not (False and True)
True
>>>
  \end{terminal}
\end{frame}

\subsection{Jämförelseoperationer}

\begin{frame}{\insertsubsectionhead}
  \begin{itemize}
    \item Precis som de aritmetiska och logiska operatorerna finns det 
      operatorer för jämförelse definierade för de vanliga datatyperna 
      i Python.

    \item Dessa är \code{== < > <= >= !=}.

    \item De används på samma sätt som de aritmetiska operatorerna, men skiljer 
      sig genom att de ger upphov till ett värde som är av boolesk typ 
      (\code{bool}) istället för ett värde av samma typ som operanderna.

  \end{itemize}
\end{frame}

\begin{frame}[fragile]{\insertsubsectionhead}
  Några exempel:
  \begin{terminal}
>>> 5+5
10
>>> 5 < 5
False
>>> 3 < 5
True
>>> 5 == 5
True
>>> 4 <= 5
True
>>> 5 <= 5
True
>>>
  \end{terminal}
\end{frame}


\section{Villkorssatser och slingor}

\subsection{Villkorssatser}

\begin{frame}{\insertsubsectionhead}
  \begin{itemize}
    \item Villkorssatser används för att kontrollera programflödet, d.v.s.  
      i vissa fall vill vi att programmet ska göra en sak och i andra fall 
      något annat.

    \item För detta behöver vi olika villkor.

    \item Ett villkor är ett uttryck som kan evalueras till antingen 
      \code{True} eller \code{False}.

    \item Exempelvis \code{x < 5} eller \code{x < 5 and x != 0}.

  \end{itemize}
\end{frame}

\begin{frame}[fragile]{\insertsubsectionhead}
  \begin{itemize}
    \item En typ av villkorssats är \code{if-elif-else}-satsen.

    \item Den fungerar på följande sätt:
      \begin{src}[language=python]
if <villkor>:
    # kod
# en eller flera elif-satser
elif <annat_villkor>:
    # kod
else:
    # kod
      \end{src}

    \item Notera att man inte nödvändigtvis behöver ha med varken \code{elif} 
      eller \code{else}, och man kan ta med hur många \code{elif} man önskar.

  \end{itemize}
\end{frame}

\begin{frame}[fragile]{\insertsubsectionhead}
  \begin{itemize}
    \item När något villkor är sant ignoreras alla efterföljande \code{elif} 
      och \code{else} även om deras villkor skulle vara sanna.

    \item Ett exempel:
      \begin{terminal}
>>> x=5
>>> if x < 5:
...     print( "less than 5" )
... elif x < 10:
...     print( "less than 10" )
... else:
...     print( "too large" )
...
less than 10
>>>
    \end{terminal}

  \end{itemize}
\end{frame}

\subsection{Slingor}

\begin{frame}{\insertsubsectionhead}
  \begin{itemize}
    \item Slingor används för att få dynamiska upprepningar i programmet, 
      t.ex.\ när vi vill läsa in en fil eller gå igenom en lista.

    \item Det finns två olika iterationskonstruktioner som tas upp här, de är 
      \code{for}- och \code{while}-satserna.

  \end{itemize}
\end{frame}

\begin{frame}[fragile]{\insertsubsectionhead}
  \begin{itemize}
    \item \code{for}-satsen fungerar på följande sätt:
      \begin{src}[language=python]
for <variabel> in <lista>:
   # kod
      \end{src}

    \item Koden i \code{for}-satsen kommer att köras en gång för varje element 
      i listan, varje gång kommer variabeln ha värdet av ett element i listan.

  \end{itemize}
\end{frame}

\begin{frame}[fragile]{\insertsubsectionhead}
  Ett exempel i python-tolken:
  \begin{terminal}
>>> for x in [1, 2, 3]:
...     print( "hej"*x )
...
hej
hejhej
hejhejhej
>>> for i in range(10):
...     print( i )
...
0 1 2 3 4 5 6 7 8 9
>>>
  \end{terminal}
\end{frame}

\begin{frame}[fragile]{\insertsubsectionhead}
  \begin{itemize}
    \item Den andra iterationskonstruktionen är \code{while}-satsen.

    \item Den kör sin kod så länge ett givet villkor är sant.

    \item Den börjar med att kolla villkoret, om det är sant körs koden, annars 
      fortsätter exekveringen direkt efter while-loopen.
      \begin{src}[language=python]
while <villkor>:
   # kod
      \end{src}

    \item Notera att den enbart kollar villkoret en gång per varv, så hela 
      koden körs trots att villkorets värde ändras under körningens gång.

  \end{itemize}
\end{frame}

\begin{frame}[fragile]{\insertsubsectionhead}
  Exempel:
  \begin{terminal}
>>> i=0
>>> while i<10:
...     print( i )
...     i += 1
...
0 1 2 3 4 5 6 7 8 9
>>>
  \end{terminal}
\end{frame}

\subsection{Programexempel}

\begin{frame}[fragile,allowframebreaks]{\insertsubsectionhead}{Beräkna porto}
  \lstinputlisting[style=code,language=Python]{porto.py}
\end{frame}

\begin{frame}[fragile]{\insertsubsectionhead}
  \begin{terminal}
(0):danbos@ID20809793:python\$ python porto.py
Välkommen till Brevvågen
Hur många brev vill du beräkna porto för (0 för att avsluta): 4
Hur mycket väger brev 1: 123
Du måste betala 24 SEK i porto.
Hur mycket väger brev 2: 23
Du måste betala 12 SEK i porto.
Hur mycket väger brev 3: 1
Du måste betala 6 SEK i porto.
Hur mycket väger brev 4: 10345
Du måste betala 275 SEK i porto.
Det blir 317 SEK för alla brev, tack!
Välkommen till Brevvågen
Hur många brev vill du beräkna porto för (0 för att avsluta): 0
(0):danbos@ID20809793:python\$
  \end{terminal}
\end{frame}


\section{Några fler datatyper}

\subsection{Datatypen list}

\begin{frame}{\insertsubsectionhead}
  \begin{itemize}
    \item List är ytterligare en datatyp i Python, den kan användas för att 
      lagra ett godtyckligt antal värden eller objekt i.

    \item T.ex.\ när man samlar data och har en datamängd som är större än ett 
      (1) element.

    \item En lista har ett antal metoder, en slags funktion definierad på 
      objektet självt.

  \end{itemize}
\end{frame}

\begin{frame}{\insertsubsectionhead}
  Om vi skapar en tom lista \code{l}, genom att skriva \code{l = list()} 
  alternativt \code{l = []}, kan vi använda följande metoder:

  \begin{description}
    \item[l.append(e)] Lägg till \code{e} som ett element sist i listan.

    \item[l.extend(l2)] Lägg till listan \code{l2} sist i listan.

    \item[l.insert(p, e)] Sätt in \code{e} som ett element framför elementet på 
      positionen \code{p}.

    \item[l.index(e)] Returnera det lägsta möjliga index för elementet som 
      matchar \code{e}.

    \item[l.remove(e)] Ta bort första förekomsten av ett element som matchar 
      \code{e}.

  \end{description}

  Man kan också skapa en lista som redan innehåller några element genom koden
  \code{l = [1, 2, 3, "hej"]}.
\end{frame}

\begin{frame}[fragile,allowframebreaks]{\insertsubsectionhead}
  \begin{terminal}
Python 2.2.3 (#1, Jan  5 2005, 16:36:30)
[GCC 3.4.2] on sunos5
Type "help", "copyright", "credits" or "license" for more information.
>>> l = [1, 2, 3]
>>> print( l )
[1, 2, 3]
>>> print( l[0] )
1
>>> print( l[2] )
3
>>> l.append(5)
>>> print( l )
[1, 2, 3, 5]
>>> l.extend([2,3])
>>> print( l )
[1, 2, 3, 5, 2, 3]
>>> l.insert(0, 10)
>>> print( l )
[10, 1, 2, 3, 5, 2, 3]
>>> l.index(3)
3
>>> l.remove(3)
>>> l.index(3)
5
>>>
  \end{terminal}
\end{frame}

\subsection{Datatypen dictionary}

\begin{frame}{\insertsubsectionhead}
  \begin{itemize}
    \item Ännu en datatyp hos Python, denna kan användas till att para ihop 
      saker.

    \item T.ex.\ namn och telefonnummer, man kan söka på ett namn och få ut ett 
      telefonnummer.

  \end{itemize}
\end{frame}

\begin{frame}{\insertsubsectionhead}
  Om vi skapar ett dictionary \code{d}, genom koden \code{d = \{\}}, kan vi 
  sedan använda följande metoder:

  \begin{description}
    \item[d.keys()] Returnerar en lista med alla lagrade nycklar.

    \item[d.values()] Returnerar en lista med alla lagrade värden.

    \item[k in d] Evalueras till sant (\code{True}) om nyckeln \code{k} finns 
      i \code{d}.

  \end{description}

  För att skapa en dictionary innehållandes några par skriver man följande:
  \code{d = \{"nyckel" : "värde", "programmering" : "python", 1 : -1\}}.
\end{frame}

\begin{frame}[fragile]{\insertsubsectionhead}
  \begin{terminal}
>>> d = {"hej" : "svejs", "kth" : "kungl tekn högskol"}
>>> d.keys()
['kth', 'hej']
>>> d.values()
['kungl tekn högskol', 'svejs']
>>> "kth" in d
1
>>> "su" in d
0
>>>
  \end{terminal}
\end{frame}

\subsection{Lite mer om strängar}

\begin{frame}{\insertsubsectionhead}
  Strängar har, likt listorna och dictionaries, också metoder. Några av dessa
  metoder beskrivs här:

  \begin{description}
    \item[s.split(c)] Delar upp strängen vid varje \code{c}, returnerar en 
      lista med alla delarna.

    \item[s.strip(c)] Returnerar en sträng där alla \code{c} tagits bort från 
      början och slut av strängen.

    \item[s.rstrip(c)] Samma som strip(), men behandlar enbart slutet av
      strängen -- början lämnas orörd.

  \end{description}
\end{frame}

\begin{frame}[fragile,allowframebreaks]{\insertsubsectionhead}
  \begin{terminal}
>>> "hej svejs i lingonskogen".split()
['hej', 'svejs', 'i', 'lingonskogen']
>>> s = "hej svejs i lingonskogen"
>>> print( s )
hej svejs i lingonskogen
>>> s.split()
['hej', 'svejs', 'i', 'lingonskogen']
>>> s.split('e')
['h', 'j sv', 'js i lingonskog', 'n']
>>> s = "   hej svejs i lingonskogen    "
>>> print( s.strip() )
hej svejs i lingonskogen
>>> print( s.rstrip() )
   hej svejs i lingonskogen
>>> s = "...hej svejs i lingonskogen...."
>>> print( s.strip('.') )
hej svejs i lingonskogen
>>> print( s.rstrip('.') )
...hej svejs i lingonskogen
>>>
  \end{terminal}
\end{frame}


\section{Felhantering}

\subsection{Översikt}

\begin{frame}{\insertsubsectionhead}
  \begin{itemize}
    \item Förutom den grundläggande felhantering man kan skriva själv i ett 
      program finns det vissa fel som man inte lika enkelt kan skydda sig emot, 
      där erbjuder Python något som kallas för särfall (eng. exceptions).

    \item Ett \emph{särfall} är en slags signal som ''sänds ut'' när ett fel 
      uppstår, detta särfall kan man sedan fånga upp på ett speciellt ställe 
      i koden och där behandla.

  \end{itemize}
\end{frame}

\begin{frame}[fragile]{\insertsubsectionhead}
  Man hanterar särfall enligt följande:

  \begin{src}[language=python]
try:
   # kod som kan ge upphov till särfall
except <typ0>:
   # kod som ska köras när ett exception av typ <typ0> fångas
except <typ1>:
   # körs när man fångar ett exception av typ <typ1>
except:
   # kod som körs för alla andra särfall
  \end{src}

  Det är helt frivilligt hur många \code{except} man använder sig av, men man 
  måste ha minst ett (1).
\end{frame}

\subsection{Exempelkod}

\begin{frame}[fragile,allowframebreaks]{\insertsubsectionhead}
  \begin{terminal}
Python 2.6.2 (r262:71600, Aug 12 2009, 11:11:06)
[GCC 3.3.5 (propolice)] on openbsd4
Type "help", "copyright", "credits" or "license" for more information.
>>> try:
...     int("hej")
... except:
...     print( "exception" )
...
exception
>>> int("hej")
Traceback (most recent call last):
  File "<stdin>", line 1, in <module>
ValueError: invalid literal for int() with base 10: 'hej'
>>> try:
...     int("hej")
... except ValueError:
...     print( "valueerror" )
... except:
...     print( "exception" )
...
valueerror
>>> def f(a,b):
...     return float(a)/float(b)
...
>>> f("5","3")
1.6666666666666667
>>> f("hej", "du")
Traceback (most recent call last):
  File "<stdin>", line 1, in <module>
  File "<stdin>", line 2, in f
ValueError: invalid literal for float(): hej
>>> f("5","0")
Traceback (most recent call last):
  File "<stdin>", line 1, in <module>
  File "<stdin>", line 2, in f
ZeroDivisionError: float division
>>> def f(a,b):
...     try:
...             return float(a)/float(b)
...     except ValueError:
...             print( "valueerror" )
...     except ZeroDivisionError:
...             print( "zerodivisionerror" )
...
>>> f("5","4")
1.25
>>> f("a","b")
valueerror
>>> f("5","0")
zerodivisionerror
>>>
  \end{terminal}
\end{frame}


\section{Filhantering}

\subsection{Översikt}

\begin{frame}{\insertsubsectionhead}
  \begin{itemize}
    \item Ibland kan man vilja ''komma ihåg'' saker mellan körningarna av 
      programmet, när alla variabler försvinner när programmet avslutas, eller 
      läsa in stora mängder data som är otympligt att mata in från 
      tangentbordet.

    \item Där kommer filhantering lämpligt in i bilden.

    \item Vi kan skriva det vi vill komma ihåg till en fil, och sedan läsa in 
      den nästa gång vi startar programmet.

  \end{itemize}
\end{frame}

\begin{frame}{\insertsubsectionhead}
  \begin{itemize}
    \item För att öppna en fil i Python använder man sig av funktionen 
      \code{open()}, denna funktion tar som första argument en sträng, som är 
      sökvägen till filen som ska öppnas.

    \item Sökvägen kan vara absolut eller relativ. Om den är relativ utgår man 
      från den mapp som programfilen finns i.

    \item Det andra argumentet är sättet vi vill öppna filen på, för läsning 
      eller för skrivning.

  \end{itemize}
\end{frame}

\begin{frame}[fragile]{\insertsubsectionhead}
  \begin{itemize}
    \item Open returnerar ett objekt som representerar filen. Objektet har 
      några metoder för att kunna läsa och skriva data från respektive till 
      filen.

    \item Vi öppnar filen \term{testfil.txt} för läsning genom
      \begin{src}[language=python]
fil = open("testfil.txt", "r")
      \end{src}

    \item Vi öppnar filen för skrivning genom
      \begin{src}[language=python]
fil = open("testfil.txt", "w")
      \end{src}
  \end{itemize}
\end{frame}

\begin{frame}{\insertsubsectionhead}{Läsning}
  Följande metoder finns för att utföra olika handlingar på filen:

  \begin{description}
    \item[text = fil.read()] Läs in hela filens innehåll till variabeln
      \code{text}.

    \item[rader = fil.readlines()] Läs in hela filen, spara alla rader
      som separata element i listan \code{rader}.

    \item[rad = fil.readline()] Läs in nästa rad till variabeln
      \code{rad}.

  \end{description}

  När vi inte längre behöver använda filen stänger vi den med 
  \code{fil.close()}.
\end{frame}

\begin{frame}{\insertsubsectionhead}
  Följande metoder finns för att utföra olika handlingar på filen:

  \begin{description}
    \item[fil.write(''hello file\textbackslash n'')] Skriv \emph{hello file} 
      till filen, följt av en radbrytning.

    \item[fil.writelines(rader)] Skriv alla strängar i listan rader till filen, 
      notera att de inte skrivs på varsin rad om de inte själva innehåller 
      radbrytningstecken.

  \end{description}

  När vi inte längre behöver använda filen stänger vi den med 
  \code{fil.close()}.
\end{frame}

\begin{frame}[fragile,allowframebreaks]{\insertsubsectionhead}
  \begin{terminal}
dbosk@my:/tmp\$ ls testfil.txt
testfil.txt: No such file or directory
dbosk@my:/tmp\$ python
Python 2.2.3 (#1, Jan  5 2005, 16:36:30)
[GCC 3.4.2] on sunos5
Type "help", "copyright", "credits" or "license" for more information.
>>> fil = open("testfil.txt", "w")
>>> fil.write("hello file\nen rad till")
>>> fil.close()
>>> \^D
dbosk@my:/tmp\$ ls testfil.txt
testfil.txt
dbosk@my:/tmp\$ python
Python 2.2.3 (#1, Jan  5 2005, 16:36:30)
[GCC 3.4.2] on sunos5
Type "help", "copyright", "credits" or "license" for more information.
>>> fil = open("testfil.txt", "r")
>>> print( fil.read() )
hello file
en rad till
>>> fil.close()
>>> fil = open("testfil.txt", "r")
>>> print( fil.readline() )
hello file

>>> print( fil.readline() )
en rad till
>>> print( fil.readline() )

>>> \^D
dbosk@my:/tmp\$
  \end{terminal}
\end{frame}

\subsection{Programexempel}

\begin{frame}[fragile,allowframebreaks]{\insertsubsectionhead}
  \lstinputlisting[style=code,language=Python]{hist.py}
\end{frame}


%%%%%%%%%%%%%%%%%%%%%%

\begin{frame}{Referenser}
  \bibliography{literature}
\end{frame}

\end{document}

