% $Id$
% Author:  Daniel Bosk <daniel.bosk@miun.se>
\documentclass{beamer}
\usepackage[utf8]{inputenc}
\usepackage[T1]{fontenc}
\usepackage[english,swedish]{babel}
\usepackage{url}
\usepackage{varioref,prettyref}
\usepackage{graphicx}
\usepackage[today,nofancy]{svninfo}
\usepackage{natbib}
\usepackage{booktabs}
\usepackage{listings}
\usepackage[listings,varioref,prettyref]{miunmisc}
\setcitestyle{numbers,square}
\bibliographystyle{swealpha}

\svnInfo $Id$

\mode<presentation>
{
  \usetheme{Frankfurt}
  \setbeamercovered{transparent}
  \usecolortheme{seagull}
}
\setbeamertemplate{footline}{\insertframenumber}

\title{%
  Python del 1
}
\author{Daniel Bosk\footnote{%
  \tiny
  Detta verk är tillgängliggjort under licensen Creative Commons 
  Erkännande-DelaLika 2.5 Sverige (CC BY-SA 2.5 SE).
  För att se en sammanfattning och kopia av licenstexten besök URL 
  \url{http://creativecommons.org/licenses/by-sa/2.5/se/}.
}}
\institute[MIUN IKS]{%
  %Department of Information and Communication Systems (ICS),\\
  %Mid Sweden University, Sundsvall.
  %
  Avdelningen för informations- och kommunikationssytem (IKS),\\
  Mittuniversitetet, Sundsvall.
}
\date{\svnId}

\pgfdeclareimage[height=0.65cm]{university-logo}{MU_logotyp_int_CMYK.pdf}
\logo{\pgfuseimage{university-logo}}

\AtBeginSection[]{%
  \begin{frame}<beamer>{Översikt}
    \tableofcontents[currentsection]
  \end{frame}
}

\begin{document}

\begin{frame}
  \titlepage
\end{frame}

\begin{frame}{Översikt}
  \tableofcontents
  % You might wish to add the option [pausesections]
\end{frame}

%\begin{frame}
%  Innan du påbörjar laborationen ska du ha läst kapitel 1, 2, 5 och 6.1--6.4 
i \cite{Brookshear2012csa} och avsnitt 2--4 i \cite{pythonkramaren1}.

%\end{frame}


% Since this a solution template for a generic talk, very little can
% be said about how it should be structured. However, the talk length
% of between 15min and 45min and the theme suggest that you stick to
% the following rules:  

% - Exactly two or three sections (other than the summary).
% - At *most* three subsections per section.
% - Talk about 30s to 2min per frame. So there should be between about
%   15 and 30 frames, all told.


\section{Introduktion}

\subsection{Vad är programmering?}

\begin{frame}{\insertsubsectionhead}
\end{frame}

\subsection{Ett enkelt exempel}

\begin{frame}{\insertsubsectionhead}
\end{frame}


\section{Variabler}

\subsection{Datatyper}

\begin{frame}{\insertsubsectionhead}
  Man kan säga att nästan allt i Python är av någon datatyp. Dessa datatyper
  håller Python själv reda på, men man måste även som programmerare hålla koll
  på dessa för att programmet ska fungera som man vill.

  De datatyper som Python definierar är bl.a.
  \begin{description}
    \item[int] Heltal (Integers),
    \item[float] Flyttal (Floating point numbers),
    \item[str] Strängar (Strings).
  \end{description}
\end{frame}


\subsection{Aritmetiska operatorer}

\begin{frame}{\insertsubsectionhead}
  På dessa datatyper finns operatorer definierade, t.ex. \code{+} är en 
  operator definierad för heltal.
  Andra operatorer som finns är \code{+ - * / // \%}.
\end{frame}

\begin{frame}[fragile]{\insertsubsectionhead}
  Några exempel (som körs i Pythons tolk från terminalen):
  \begin{terminal}
\$ python
Python 2.2.3 (#1, Jan  5 2005, 16:36:30)
[GCC 3.4.2] on sunos5
Type "help", "copyright", "credits" or "license" for more information.
>>> 5+5
10
>>> 3*2
6
>>> 5//3
1
>>> 5\%3
2
>>> "hej"+"svejs"
'hejsvejs'
>>> "hej"*3
'hejhejhej'
>>>
  \end{terminal}
\end{frame}

\begin{frame}{\insertsubsectionhead}
  Alla operatorer är dock inte definierade för alla datatyper, en mycket kort
  och ej fullständig sammanfattning är:
  \begin{description}
    \item[Heltal] kan använda operatorerna \code{+ - * / // \%},
    \item[Flyttal] kan använda operatorerna \code{+ - * /}, heltalsdivision
    och modulo går inte att beräkna för flyttal,
    \item[Strängar] har bara operatorerna \code{+ *}.
  \end{description}
\end{frame}


%\section{Konstanter}
%\noindent
%För att förenkla beräkningar etc. har vi något som kallas för konstanter.
%Dessa används för att underlätta när vi har återkommande värden, t.ex. \(\pi\).
%En konstant i Python skapas på följande vis \code{PI=3.14}, och kan sedan
%användas genom att man skriver \code{PI} istället för \code{3.14} överallt i
%sin kod. En exempelkörning i Python kan se ut enligt följande.
%\begin{lstlisting}[style=text]
%>>> PI=3.14
%>>> 2*PI
%6.2800000000000002
%>>>
%\end{lstlisting}
%
%Konstanter kan också användas för vissa restriktioner som man kanske har i sin
%kod, t.ex. att ett namn måste vara 32 tecken. Om man senare kommer på att
%namnet borde kunna vara 64 tecken långt behöver man bara ändra värdet på
%konstanten\footnote{Med att ändra värde på konstanten menare jag att man ändrar
%i koden, värdet på en konstant kan inte ändras under programkörning.} istället
%för att leta upp alla ställen i koden där det används.


\subsection{Identifierare}

\begin{frame}{\insertsubsectionhead}
  \begin{itemize}
    \item Vilka regler gäller då för namnet på konstanterna?
    \item De får bestå av bokstäver, siffror och även understreck (\code{_}).
    \item Men de får dock inte börja med siffror.
    \item De får heller inte vara något av följande reserverade ord:
      \begin{code}
and assert break class continue def del elif
else except exec finally for from global if
import in is lambda not or pass print raise
return try while
      \end{code}
    \item Viktigt att tänka på är att man skiljer på gemener och versaler, 
      d.v.s.\ \code{PI} är inte samma sak som \code{Pi}.
  \end{itemize}
\end{frame}

\subsection{Variabler}

\begin{frame}{\insertsubsectionhead}
  \begin{itemize}
    \item En variabel är en identifierare.
    \item Den representerar ett ett minnesutrymme i vilket man kan lagra data.
    \item Namnet uppfyller kravet för en identifierare, men får inte vara 
      enbart versaler (för det tolkar Python som en konstant).
  \end{itemize}
\end{frame}

\begin{frame}[fragile]{\insertsubsectionhead}
  \begin{terminal}
>>> x=5
>>> y=3
>>> z=x*y*3
>>> print x, y, z
5 3 45
>>> print x
5
>>> print y
3
>>> print z
45
>>> x=x+1
>>> print x
6
>>> print z
45
>>>
  \end{terminal}
\end{frame}

\begin{frame}{\insertsubsectionhead}
  \begin{itemize}
    \item Notera att värdet på \code{z} inte ändras när vi ändrar värdet på 
      \code{x}.
    \item Detta för att det är värdet \code{45} som lagras i \code{z} och inte 
      relationen \code{x*y*3}.
  \end{itemize}
\end{frame}

\subsection{Typkonvertering}

\begin{frame}{\insertsubsectionhead}
  \begin{itemize}
    \item Ibland kan man vilja konvertera vissa typer till andra.
    \item Det kan vara att man läst in en sträng från tangentbordet och vill 
      konvertera den till ett tag (om användaren matade in ett tal).
  \end{itemize}
\end{frame}

\begin{frame}[fragile]{\insertsubsectionhead}
  \begin{terminal}
>>> x="3.14"
>>> pi=float(x)
>>> print pi
3.14
>>> print x+2
Traceback (most recent call last):
  File "<stdin>", line 1, in ?
  TypeError: cannot concatenate 'str' and 'int' objects
>>> print pi+2
5.14
>>>
  \end{terminal}
\end{frame}

\begin{frame}[fragile]{\insertsubsectionhead}
  Ett annat bra exempel är procentberäkningar,
  \begin{terminal}
>>> 99/100*100
0
>>> float(99)/100*100
99.0
>>>
  \end{terminal}
  där man får fel svar om man inte explicit typkonverterar.
\end{frame}


\section{Funktioner}

\subsection{Översikt}

\begin{frame}[fragile]{\insertsubsectionhead}
  \begin{itemize}
    \item Funktioner utgör en viktig byggsten i programmeringen.
    \item Liksom inom matematiken kan den användas för att dela upp större 
      problem i mindre och bidra till en bättre ordning.
    \item De används dessutom på samma sätt.
    \item \(f(x) = 2 \cdot g(x) + 3\) och \(g(x) = x^2\) inom matematiken 
      skulle kunna skrivas i Python som
      \begin{code}
def g(x):
    return x*x

def f(x):
    return 2*g(x)+3
      \end{code}
  \end{itemize}
\end{frame}

\begin{frame}{\insertsubsectionhead}
  \begin{itemize}
    \item När vi talar om funktioner består de av flera delar, funktionen 
      består av ett funktionshuvud och en funktionskropp.
    \item Funktionshuvudet är den första raden i funktionsdefinitionen.
    \item Den består av funktionsnamn och formella parametrar.
      D.v.s.
  \[
  \overbrace{\text{def functionname(}
    \underbrace{\text{argument1, argument2, ...}}_{\text{formella
      parametrar}}\text{):}}^{\text{funktionshuvud}}
  \]
    \item Resten av funktionen är funktionskroppen, d.v.s.\ all kod som hör 
      funktionen till.
  \end{itemize}
\end{frame}


\section{Ett programexempel}

\subsection{Beräkning av energi hos en boll}

\begin{frame}[allowframebreaks,fragile]{\insertsubsectionhead}
  \lstinputlisting[language=Python]{./O1.py}
\end{frame}

\begin{frame}[fragile]{\insertsubsectionhead}
  En körning av detta program (från en terminal) kan se ut så här:
  \begin{terminal}
  $ python O1.py
  Ange bollens massa: 2
  Ange bollens höjd: 3
  Ange bollens hastighet: 4
  Bollens rörelseenergi =  16.0 Joule
  Bollens potentiella energi 58.92 Joule
  $

  \end{terminal}
\end{frame}


\subsection{Tips och kommentarer}

\begin{frame}{\insertsubsectionhead}
  Några punkter att tänka på:
  \begin{itemize}
    \item Blanda aldrig språk, antingen skriver ni alla variabelnamn och
    funktioner på engelska eller på svenska -- aldrig båda!
    \item Använd förklarande namn till alla variabler och funktioner!
    \item Skriv kommentarer!
    \item Dela upp programmen i mindre delar -- funktioner! Det blir
    lättare att följa och det blir snyggare kod, men framför allt mycket
    enklare att programmera!
  \end{itemize}
\end{frame}



%%%%%%%%%%%%%%%%%%%%%%

\begin{frame}{Referenser}
  \bibliography{literature}
\end{frame}

\end{document}

