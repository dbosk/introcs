% $Id$
Föreläsningen tar upp kapitel 1 ''Data Storage'' i \citep{Brookshear2012csa}, 
som handlar om fysisk lagring av data på olika media.

Innehållet kommer även att kompletteras med delar från kapitlen 2 och 
6 i \cite{Bosk2011m1c} som förklarar den logiska representationen av data.
Kapitel 2 tar logik och bevis.
Då en dator är strikt baserad på logiska operationer är detta en viktig grund.
Bevis är helt enkelt tillämpning av de logiska reglerna och detta behövs för 
att förstå bevisen i kapitel 6.

Kapitel 6 täcker talsystem och förklarar varför vi kan använda datorns logiska 
system för att räkna med godtyckliga tal.
Logiska operationer samt representation och beräkningar av tal är vad en dator 
gör när program körs.
