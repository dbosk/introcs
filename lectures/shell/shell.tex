%\documentclass[handout]{beamer}
\documentclass{beamer}
\usepackage[utf8]{inputenc}
\usepackage[T1]{fontenc}
\usepackage[swedish,english]{babel}
\usepackage{verbatim}
\usepackage{url}
\usepackage{natbib}
\usepackage{varioref,prettyref}
\usepackage{amsmath,amsthm,amssymb}
\usepackage[amssymb,noams,binary]{SIunits}
\usepackage{listings}
\usepackage{graphicx}
\usepackage[today,nofancy]{svninfo}
\usepackage[listings,varioref,prettyref]{miunmisc}
\setcitestyle{numbers,square}
\bibliographystyle{alpha}

\svnInfo $Id$

\lstset{style=text}

\mode<presentation>
{
	%\usetheme{Berkeley}
  %\usetheme{Dresden}
  \usetheme{Frankfurt}
  %\usetheme{Madrid}
  %\usetheme{PaloAlto}
  \setbeamercovered{transparent}
  %\usecolortheme{default}
  %\usecolortheme{crane}
  \usecolortheme{seagull}
}
\def\newblock{\hskip .11em plus .33em minus .07em}

\title[UNIX-like shells]{%
	Introduction to Operating Systems:\\
	Lecture on UNIX-like shells
}
\author{Daniel Bosk}
\institute{%
	%Division of Information and Communication Systems (ICS),\\
	%Avdelningen för informations- och kommunikationssytem (IKS),\\
	Department of Information Technology and Media (ITM),\\
	%Institutionen för informationsteknologi och medier (ITM),\\
	Mid Sweden University, Sundsvall.
	%Mittuniversitetet, Sundsvall.
}
\date{\svnId}

\pgfdeclareimage[height=0.65cm]{university-logo}{MU_logotyp_int_CMYK.pdf}
\logo{\pgfuseimage{university-logo}}

\AtBeginSection[]{%
	\begin{frame}<beamer>{Overview}
		\tableofcontents[currentsection]
	\end{frame}
}

\begin{document}

\begin{frame}
  \titlepage
\end{frame}

\begin{frame}{Overview}
	\tableofcontents
	% You might wish to add the option [pausesections]
\end{frame}


% Since this a solution template for a generic talk, very little can
% be said about how it should be structured. However, the talk length
% of between 15min and 45min and the theme suggest that you stick to
% the following rules:  

% - Exactly two or three sections (other than the summary).
% - At *most* three subsections per section.
% - Talk about 30s to 2min per frame. So there should be between about
%   15 and 30 frames, all told.


\section{UNIX shells}
\begin{frame}{UNIX shells}
	\begin{itemize}
		\item<1> The shell interprets commands from the user and executes them.
		\item<2> The UNIX design of the shell is to implement all commands as 
			separate programs -- \emph{each of which does one thing and does that 
			thing well}.
		\item<3> These programs are located in /bin, /sbin, /usr/bin, etc.
		\item<4> Shells are also programs, standard shells are located in /bin.
		\item<5> The simplistic design of UNIX makes many different shells 
			available, e.g.
			\begin{itemize}
				\item Korn Shell, ksh(1),
				\item Bourne Shell, sh(1),
				\item Bourne Again Shell, bash(1), and
				\item the X window system (X11), Xorg(1).
			\end{itemize}
	\end{itemize}
\end{frame}
\begin{frame}[fragile]{UNIX shells}{Manual pages}
	\begin{itemize}
		\item<1-2> Access manual pages (man-pages) by using the command man(1).
		\item<2> Usage: \code{man [section] name}
		\item<3> The section is given within parentheses directly after the command 
			name, e.g. man(1) or sh(1).
			\begin{lstlisting}
$ man 1 sh
[man-page of sh]
$ man sh
[man-page of sh]
$
			\end{lstlisting}
		\item<4> A man-page with the same name can occur in different sections, 
			e.g. printf(1) and printf(3).
	\end{itemize}
\end{frame}

\subsection{Input, output and error streams}
\begin{frame}{Input, output and error streams}
	\begin{itemize}
		\item<1-> Three special (and always open) files (streams):
			\begin{description}
				\item[stdin] input from e.g. terminal (i.e. keyboard).
				\item[stdout] output from process to e.g. to terminal (i.e.  display).
				\item[stderr] error messages are written to stderr.
			\end{description}
		\item<2> Both stdout and stderr are output streams usually displayed in the 
			terminal.
		\item<3> Occationally these three streams are referred to by numbers, their 
			file descriptors:
			\begin{description}
				\item[stdin] filedescriptor 0
				\item[stdout] filedescriptor 1
				\item[stderr] filedescriptor 2
			\end{description}
	\end{itemize}
\end{frame}

\subsection{Redirections}
\begin{frame}[fragile]{Redirections}
	\begin{itemize}
		\item<1> $>$ redirects output.
			\begin{lstlisting}
$ echo testing to redirect some output > /tmp/test.txt
$ cat /tmp/test.txt
testing to redirect some output
$
			\end{lstlisting}
		\item<2> $<$ redirects input.
			\begin{lstlisting}
$ cat
test 1 2 3
test 1 2 3
$ cat < /tmp/test.txt
testing to redirect some output
$
			\end{lstlisting}
	\end{itemize}
\end{frame}

\subsection{Pipelines}
\begin{frame}[fragile]{Pipelines}
	\begin{itemize}
		\item The pipe: redirects stdout of one process to stdin of another.
			\begin{lstlisting}
$ echo test 1 2 3 | cat
test 1 2 3
$ ls / | cat
bin
etc
usr
[...]
$
			\end{lstlisting}
	\end{itemize}
\end{frame}

\subsection{Environment variables}
\begin{frame}{Environment variables}
	\begin{itemize}
		\item<1> Can be accessed by all processes.
		\item<2> Can be used to store settings for some tools and utilities.
			\begin{description}
				\item[PAGER] path to user's preferred pager.
				\item[EDITOR] path to user's preferred editor.
				\item[VISUAL] path to user's preferred visual editor.
				\item[PATH] a colon separated list of paths to directories containing 
					executable files (commands).
				\item[HOME] the path to the user's home directory.
				\item[PS1] sets prompt of the shell, see e.g. sh(1).
			\end{description}
		\item<3> More can be read about this in the man-page for your shell and 
			environ(7).
	\end{itemize}
\end{frame}
\begin{frame}{Variable substitution}
	\begin{itemize}
		\item<1-3> A variable is created and assigned by doing: 
			\code{VARIABLE=value}.
		\item<2-3> A variable can be referenced by its name prefixed with 
			a dollar-sign (\$), i.e. \code{\$VARIABLE}.
		\item<2-3> An alternative way is \code{\$\{VARIABLE\}}.
		\item<3> The variable reference is substituted with the value of the 
			variable.
		\item<4> There are some special purpose variables:
			\begin{description}
				\item[*] expands to all positional parameters \code{\$1 \$2 \$3 ...}.
				\item[\#] expands to the number of positional parameters.
				\item[0] expands to the name of the shell or shell script.
			\end{description}
	\end{itemize}
\end{frame}
\begin{frame}[fragile]{Variable substitution, continued}
	\begin{lstlisting}
$ export EDITOR=vim
$ export PATH=${HOME}/bin:${PATH}
$ echo $PATH
/home/danbos/bin:/usr/bin:/bin:[...]
$ $EDITOR /tmp/test.txt
[opens /tmp/test.txt for editing with vim]
$ EDITOR=emacs $EDITOR /tmp/test.txt
$ echo $0 ${0}
/bin/pdksh /bin/pdksh
$
	\end{lstlisting}
\end{frame}
\begin{frame}[fragile]{Command substitution}
	\begin{itemize}
		\item<1-2> Output from commands can be substituted into environment 
			variables.
		\item<2> This is done using \code{\$(<command> <arguments>)}.
			\begin{lstlisting}
$ username=$(whoami)
$ echo $username
danbos
$
$ old_time=$(date)
$ sleep 5
$ echo the time 5 seconds ago was $old_time
the time 5 seconds ago was Fri Apr 13 06:56:57 CEST 2012
			\end{lstlisting}
	\end{itemize}
\end{frame}


\section{Some useful tools}
\begin{frame}{Some useful tools}
	\begin{description}
		\item[echo(1)] display a line of text
		\item[test(1)] check file types and compare values
		\item[find(1)] search for files in a directory hierarchy
		\item[tr(1)] translate or delete characters
		\item[uniq(1)] report or omit repeated lines
		\item[sort(1)] sort lines of text files
		\item[wc(1)] print newline, word, and byte counts for each file
		\item[cut(1)] remove sections from each line of files
		\item[join(1)] join lines of two files on a common field
		\item[paste(1)] merge lines of files
		\item[xargs(1)] build and execute command lines from standard input
		\item[grep(1)] print lines matching a pattern
		\item[sed(1)] stream editor for filtering and transforming text
	\end{description}
\end{frame}

\begin{comment}
\subsection{A plethora of tools}
\begin{frame}[fragile]{test(1)}
	\begin{lstlisting}
$
	\end{lstlisting}
\end{frame}
\begin{frame}[fragile]{find(1)}
	\begin{lstlisting}
$
	\end{lstlisting}
\end{frame}
\begin{frame}[fragile]{tr(1)}
	\begin{lstlisting}
$
	\end{lstlisting}
\end{frame}
\begin{frame}[fragile]{uniq(1)}
	\begin{lstlisting}
$
	\end{lstlisting}
\end{frame}
\begin{frame}[fragile]{sort(1)}
	\begin{lstlisting}
$
	\end{lstlisting}
\end{frame}
\begin{frame}[fragile]{wc(1)}
	\begin{lstlisting}
$
	\end{lstlisting}
\end{frame}
\begin{frame}[fragile]{cut(1)}
	\begin{lstlisting}
$
	\end{lstlisting}
\end{frame}
\begin{frame}[fragile]{join(1)}
	\begin{lstlisting}
$
	\end{lstlisting}
\end{frame}
\begin{frame}[fragile]{paste(1)}
	\begin{lstlisting}
$
	\end{lstlisting}
\end{frame}
\begin{frame}[fragile]{xargs(1)}
	\begin{lstlisting}
$
	\end{lstlisting}
\end{frame}
\end{comment}

\subsection{Regular expressions (regex)}
\begin{frame}{Regular expressions (regex)}
	\begin{itemize}
		\item Is a pattern matching language, some details in regex(7).
		\item Both grep(1) and sed(1) uses regular expressions.
		\item Searching within man-pages are done using regex.
	\end{itemize}
\end{frame}
\begin{frame}{Regex, continued}
	\begin{itemize}
		\item Ordinary characters are matched by themselves.
		\item There are special characters which must be escaped:
			\begin{center}
				\code{\{\}[].*^\$?()\|}.
			\end{center}
	\end{itemize}
\end{frame}
\begin{frame}{Regex, continued}{Quantifiers}
	\begin{itemize}
		\item Asterisk (*): 0 or more.
		\item Question mark (?): 0 or 1.
		\item Braces (\(\{n\}\)): exactly \(n\).
		\item Braces (\(\{n,m\}\)): either \(n, n+1, \ldots,\) or \(m\).
		\item Braces (\(\{n,\}\)): \(n\) or more.
	\end{itemize}
\end{frame}
\begin{frame}{Regex, continued}{Ranges}
	\begin{itemize}
		\item Dot (.): any character.
		\item Parantheses ((a$|$b)): a or b.
		\item Square parantheses ([abc]): either a or b or c.
		\item Square parantheses ([\^\,abc]): not a nor b nor c.
		\item Square parantheses ([a-z]): one letter between a and z.
	\end{itemize}
\end{frame}
\begin{frame}{Regex examples}
	\uncover<1>{
		The assignment instruction said \emph{hand in three files named 
		kommandon.txt, inl.txt and svar\_1.3.doc}.
		Quite straight forward, right?
	}

	\uncover<2->{
		These are the regular expressions I needed in my script:
	}
	\begin{itemize}
		\item<3>\relax 
			[Kk]omm?andon?\textbackslash.txt(\textbackslash.txt|\textbackslash.doc|\textbackslash.docx)?
		\item<4>\relax [Ii]nl\textbackslash.txt(\textbackslash.txt)?
		\item<5>\relax 
			.*([Ss]var|inlupp).*1.*\textbackslash..*3.*\textbackslash.(odt|doc|docx)*
	\end{itemize}
\end{frame}

\begin{comment}
\subsection{Two more tools}
\begin{frame}[fragile]{grep(1)}
	\begin{lstlisting}
$
	\end{lstlisting}
\end{frame}
\begin{frame}[fragile]{sed(1)}
	\begin{lstlisting}
$
	\end{lstlisting}
\end{frame}
\end{comment}


\section{Some shell examples}

\subsection{A shell example courtesy of McIlroy}
\begin{frame}[fragile]{A shell example courtesy of McIlroy}
	\begin{lstlisting}
$ cat long_text.txt | \
> tr -cs A-Za-z '\n' | \
> tr A-Z a-z | \
> sort | \
> uniq -c | \
> sort -k1,1nr -k2 | \
> head
[ten lines of output]
$
	\end{lstlisting}
\end{frame}

\subsection{How to transfer a set of files}
\begin{frame}[fragile]{How to transfer a set of files}
	\begin{lstlisting}
$ tar -zcf - /path/to/files | \
> ssh user@host.domain.tld tar -zxf -
$
	\end{lstlisting}
\end{frame}

\subsection{How to create a simple VoIP system}
\begin{frame}[fragile]{How to create a simple VoIP system}
	From OpenBSD to OpenBSD:
	\begin{lstlisting}
$ cat /dev/audio | compress | \
> ssh user@host.domain.tld "uncompress > /dev/audio" &
$
	\end{lstlisting}

	\pause
	From Ubuntu to Ubuntu:
	\begin{lstlisting}
$ arecord | gzip | ssh user@host.domain.tld "uncompress | aplay" &
$
	\end{lstlisting}
\end{frame}


\section{Shell scripting}
\begin{frame}{Shell scripting}
	\begin{itemize}
		\item<1-> Just shell commands, redirections, pipes etc. written to a file.
		\item<2> Thanks to the UNIX design, there is no difference reading from 
			stdin with stdin being the keyboard and stdin being redirected from 
			a file.
		\item<3-4> The file is hence read by the shell process and executed.
		\item<4> Note that it opens the file separately, it does not redirect it to 
			stdin -- although this is fully possible.
	\end{itemize}
\end{frame}

\subsection{Execution flow-control constructs}
\begin{frame}{Execution flow-control constructs}
	\begin{itemize}
		\item if-then, elif-then, else
		\item for-do
		\item while-do
		\item case
	\end{itemize}
	These can be read about in the man-page of the shell, e.g. sh(1).
\end{frame}
\begin{frame}[fragile]{if-then, elif-then, else}
	\begin{lstlisting}
$ VARIABLE=Yes
$ if [ "$VARIABLE" = "Yes" ]; then \
> echo "OK"; \
> elif [ "$VARIABLE" = "No" ]; then \
> echo "Sure thing"; \
> else
> echo "Huh?"
> fi
OK
$
	\end{lstlisting}
\end{frame}
\begin{frame}[fragile]{for-do}
	\begin{lstlisting}
$ for i in 1 2 3; do \
> echo $i; \
> done
1
2
3
$
	\end{lstlisting}
\end{frame}
\begin{comment}
\begin{frame}[fragile]{while-do}
	\begin{lstlisting}
$
	\end{lstlisting}
\end{frame}
\begin{frame}[fragile]{case}
	\begin{lstlisting}
$
	\end{lstlisting}
\end{frame}
\end{comment}

\subsection{Some example shell scripts}
\begin{frame}{Some example shell scripts}
	\begin{description}
		\item[libris] a script which fetches book information based on ISBN 
			\citep[for source see][]{Bosk2010libris}.
		\item[rm] a delayed remove command \citep[for source see][]{Bosk2012rm}.
	\end{description}
\end{frame}


%%%%%%%%%%%%%%%%%%%%%%

\begin{frame}{References}
	\bibliography{../../literature}
\end{frame}

\end{document}

