% $Id$
% Author:	Daniel Bosk <daniel.bosk@miun.se>
%\documentclass[handout]{beamer}
\documentclass{beamer}
\usepackage[utf8]{inputenc}
\usepackage[T1]{fontenc}
\usepackage[english,swedish]{babel}
\usepackage{url}
\usepackage{varioref,prettyref}
\usepackage{graphicx}
\usepackage[today,nofancy]{svninfo}
\usepackage[natbib,style=alphabetic,maxbibnames=99]{biblatex}
\addbibresource{literature.bib}
\usepackage[varioref,prettyref]{miunmisc}

\svnInfo $Id$

\mode<presentation>
{
  \usetheme{Frankfurt}
  \setbeamercovered{transparent}
  \usecolortheme{seagull}
}
%\def\newblock{\hskip .11em plus .33em minus .07em}
\setbeamertemplate{footline}{\insertframenumber}

\title[Introduktion]{%
  Introduktion till \\
  datateknik och datavetenskap
}
\author{Daniel Bosk\footnote{%
    \tiny
    Detta verk är tillgängliggjort under licensen Creative Commons 
    Erkännande-DelaLika 2.5 Sverige (CC BY-SA 2.5 SE).
    För att se en sammanfattning och kopia av licenstexten besök URL 
    \url{http://creativecommons.org/licenses/by-sa/2.5/se/}.
  } och
  Jimmy Åhlander
}
\institute[MIUN IKS]{%
  %Department of Information and Communication Systems (ICS),\\
  %Mid Sweden University, Sundsvall.
	%
  Avdelningen för informations- och kommunikationssytem,\\
  Mittuniversitetet, SE-851\,70 Sundsvall.
}
\date{\svnId}

\pgfdeclareimage[height=0.65cm]{university-logo}{MU_logotyp_int_CMYK.pdf}
\logo{\pgfuseimage{university-logo}}

\AtBeginSection[]{%
	\begin{frame}<beamer>{Översikt}
		\tableofcontents[currentsection]
	\end{frame}
}

\begin{document}

\begin{frame}
  \titlepage
\end{frame}

\begin{frame}{Översikt}
	\tableofcontents
	% You might wish to add the option [pausesections]
\end{frame}


% Since this a solution template for a generic talk, very little can
% be said about how it should be structured. However, the talk length
% of between 15min and 45min and the theme suggest that you stick to
% the following rules:  

% - Exactly two or three sections (other than the summary).
% - At *most* three subsections per section.
% - Talk about 30s to 2min per frame. So there should be between about
%   15 and 30 frames, all told.


\section{Formalia}

\subsection{Schema}
\begin{frame}{Schema}
  \begin{itemize}
    \item Schemat finns i det centrala schemat i Studentportalen.
    \item Detta synkroniseras ej med kalendern i lärplattformen.
    \item Det är de tillfällen som ligger i schemat som gäller!
  \end{itemize}
  \begin{itemize}
    \item Jimmy Åhlander ger föreläsningar och håller övningarna. 
    \item Jan-Erik Jonsson handleder och rättar laborationer.
  \end{itemize}
\end{frame}

\subsection{Lärplattform}
\begin{frame}{Lärplattform}
  \begin{itemize}
    \item Det är Lärplattformen 2.0 som används för kursen.
    \item Under avsnittet ''Kursmaterial'' finns inspelningarna av 
      föreläsningarna och slides.
    \item Under avsnittet ''Examination'' finns alla inlämningslådor.
    \item I respektive inlämningslåda finns en länk till lydelsen för 
      uppgiften.
  \end{itemize}
\end{frame}

\subsection{Litteratur}
\begin{frame}{Litteratur}
  \begin{itemize}
    \item \citetitle{Brookshear2012csa} \cite{Brookshear2012csa}.
    \item \citetitle{nemeth2011ual} \cite{nemeth2011ual}.
    \item \citetitle{pythonkramaren1} \cite{pythonkramaren1}.
    \item \citetitle{Oetiker2011lshort} \cite{Oetiker2011lshort}.
  \end{itemize}
\end{frame}

\subsection{Examination}
\begin{frame}{Laborationer}
  \begin{itemize}
    \item L0 Installationer
    \item L1 Terminalen
    \item L2 Programmering med Python
    \item L3 Datorn
    \item L4 Presentationsteknik
  \end{itemize}
\end{frame}
\begin{frame}{Tenta}
  \begin{itemize}
    \item Skriftlig tentamen
  \end{itemize}
\end{frame}


\section{Datateknik och datavetenskap}

\subsection{Datavetenskapens historia}
\begin{frame}{\insertsubsectionhead}
  \begin{description}
    \item[2500 f.v.t.] Euklides algoritm.
    \item[1600-talet] Blaise Pascal och Gottfried Wilhelm von Leibniz.
    \item[1800-talet] Charles Babbage och Augusta Ada Byron.
    \item[1930-talet] Alan Turing.
  \end{description}
\end{frame}


\section{Vetenskapen om algoritmer}

\subsection{Algoritmer}
\begin{frame}{\insertsubsectionhead}
\end{frame}

\subsection{Abstraktionslager}
\begin{frame}{\insertsubsectionhead}
\end{frame}


%%%%%%%%%%%%%%%%%%%%%%

\begin{frame}[allowframebreaks]{Referenser}
  \small
  \printbibliography
\end{frame}

\end{document}

