% $Id$
% Author:	Daniel Bosk <daniel.bosk@miun.se>
%\documentclass[handout]{beamer}
\documentclass{beamer}
\usepackage[utf8]{inputenc}
%\usepackage{times}
\usepackage[T1]{fontenc}
\usepackage[english,swedish]{babel}
\usepackage{url}
\usepackage{varioref,prettyref}
\usepackage{graphicx}
\usepackage[today,nofancy]{svninfo}
\usepackage{natbib}
\usepackage[varioref,prettyref]{miunmisc}
\setcitestyle{numbers,square}
\bibliographystyle{swealpha}

\svnInfo $Id$

\mode<presentation>
{
	%\usetheme{Berkeley}
  %\usetheme{Dresden}
	\usetheme{Frankfurt}
  %\usetheme{Madrid}
  %\usetheme{PaloAlto}
	\setbeamercovered{transparent}
  %\usecolortheme{default}
  %\usecolortheme{crane}
	\usecolortheme{seagull}
}
%\def\newblock{\hskip .11em plus .33em minus .07em}
\setbeamertemplate{footline}{\insertframenumber}

\title[Introduktion]{%
	Introduktion till kursen\\
	DT001G Informationsteknologi grundkurs
}
\author{Daniel Bosk\footnote{%
	\tiny
	Detta verk är tillgängliggjort under licensen Creative Commons 
	Erkännande-DelaLika 2.5 Sverige (CC BY-SA 2.5 SE).
	För att se en sammanfattning och kopia av licenstexten besök URL 
	\url{http://creativecommons.org/licenses/by-sa/2.5/se/}.
}}
\institute[MIUN ITM]{%
	%Division of Information and Communication Systems (ICS),\\
	%Department of Information Technology and Media (ITM),\\
	%Mid Sweden University, Sundsvall.
	%
	%Avdelningen för informations- och kommunikationssytem (IKS),\\
	Institutionen för informationsteknologi och medier (ITM),\\
	Mittuniversitetet, Sundsvall.
}
\date{\svnId}

\pgfdeclareimage[height=0.65cm]{university-logo}{MU_logotyp_int_CMYK.pdf}
\logo{\pgfuseimage{university-logo}}

\AtBeginSection[]{%
	\begin{frame}<beamer>{Översikt}
		\tableofcontents[currentsection]
	\end{frame}
}

\begin{document}

\begin{frame}
  \titlepage
\end{frame}

\begin{frame}{Översikt}
	\tableofcontents
	% You might wish to add the option [pausesections]
\end{frame}


% Since this a solution template for a generic talk, very little can
% be said about how it should be structured. However, the talk length
% of between 15min and 45min and the theme suggest that you stick to
% the following rules:  

% - Exactly two or three sections (other than the summary).
% - At *most* three subsections per section.
% - Talk about 30s to 2min per frame. So there should be between about
%   15 and 30 frames, all told.


\section{Formalia}

\subsection{Schema}
\begin{frame}{Schema}
	\dots
\end{frame}

\subsection{Lärplattform}
\begin{frame}{Lärplattform}
\end{frame}

\subsection{Litteratur}
\begin{frame}{Litteratur}
	\emph{Computer science: an overview} \cite{Brookshear2012csa}.
\end{frame}

\subsection{Examination}
\begin{frame}{Examination}
	\dots
\end{frame}
\begin{frame}{Laborationer}
	\dots
\end{frame}
\begin{frame}{Seminarier}
	\dots
\end{frame}
\begin{frame}{Projekt}
	\dots
\end{frame}


\section[Historia]{Datavetenskapens historia}
\begin{frame}{Datavetenskapens historia}
	\dots
\end{frame}


\section[Introduktion]{Introduktion till datavetenskap}
\begin{frame}{Introduktion till datavetenskap}
	\dots
\end{frame}


%%%%%%%%%%%%%%%%%%%%%%

\begin{frame}{Referenser}
	\bibliography{../../literature}
\end{frame}

\end{document}

