% $Id$
\documentclass[a4paper,logo]{miunart}
\usepackage[utf8]{inputenc}
\usepackage[swedish]{babel}
\usepackage{prettyref,varioref}
\usepackage[hyphens]{url}
\usepackage{hyperref}
\usepackage[today,nofancy]{svninfo}
\usepackage{booktabs}
\usepackage[natbib,style=alphabetic,maxbibnames=99]{biblatex}
\addbibresource{literature.bib}
\usepackage{listings}
\usepackage[varioref,prettyref,listings]{miunmisc}

\svnInfo $Id$

\title{Den fullständiga studiehandledningen för\\
  DT001G Informationsteknologi grundkurs}
\author{
  Daniel Bosk\footnote{%
    Detta verk är tillgängliggjort under licensen Creative Commons 
    Erkännande-DelaLika 2.5 Sverige (CC BY-SA 2.5 SE).
    För att se en sammanfattning och kopia av licenstexten besök URL 
    \url{http://creativecommons.org/licenses/by-sa/2.5/se/}.
  }
}
\date{\svnId}

\begin{document}
\maketitle
\tableofcontents


\section{Mål}
\label{sec:aim}
Kursen utgör en introduktion till datateknik och är förberedande för praktiskt 
inriktade datatekniska utbildningar.
Centrala delar är datorns konstruktion, datorkomponenter, grundläggande begrepp 
och terminologier samt mjukvara.
Kursen fokuserar på praktisk användning av datorer och programvaror.
Kursen ger även en introduktion till rapportskrivande och presentationer.

Mer specifikt ska du efter genomgången kurs uppfylla följande mål:
\begin{itemize}
  \input{install-aim.tex}
  \input{term-aim.tex}
  \input{pylab-aim.tex}
  \input{computer-aim.tex}
  \input{project-aim.tex}
\end{itemize}


\section{Kursupplägg}
\label{sec:outline}
Den bok som används som huvudlitteratur på kursen är Brookshears bok 
\emph{Computer Science: An Overview} \cite{Brookshear2012csa}.
Därutöver tillkommer litteratur för en del av laborationerna, denna framgår av 
läsanvisningarna nedan och i respektive laboration.

\citet{Brookshear2012csa} tar upp grunden inom datateknik, den är värd att läsa 
i sin helhet -- detta rekommenderas!
De kapitel och avsnitt som behandas i denna kurs är dock enbart kapitlen 0 till 
och med 5 i sina helheter, avsnitten 6.1 till och med 6.4, samt avsnitten 9.1 
och 9.2.

Det går att använda tidigare upplagor av boken, kapitlen som behandlas i kursen 
utgår från den senaste upplagan \citep{Brookshear2012csa} och de behandlar 
följande områden:
\begin{itemize}
	\item kapitel 0: introduktion och historia,
	\item kapitel 1: datalagring och -representation,
	\item kapitel 2: datamanipulering, datorarkitektur, programexekvering,
	\item kapitel 3: operativsystem,
	\item kapitel 4: nätverk och internet,
	\item kapitel 5: algoritmer,
	\item avsnitten 6.1 till och med 6.4: programspråk och programmering,
	\item avsnitten 9.1 och 9.2: grunder för databaser och relationsdatabaser.
\end{itemize}

Litteraturläsningen kompletteras av ett antal föreläsningar och övningar som 
ges under kursens gång.
Dessa kommer inte att vara heltäckande och för att förstå dem måste du läsa 
litteraturen enligt anvisningarna nedan.

Kursens lärandemål kommer att examineras med ett antal laborationer och 
avslutas med ett mindre projekt.
Detta projekt motsvaras av två veckors heltidsarbete.
Den som nyligen kommer från gymnasieskolan kan ha som referens att detta 
motsvaras precis av projektarbetet för gymnasieskolan, dock med högre krav på 
innehållet.

\subsection{Schema}
\label{sec:schedule}
Du finner en sammanställning av kursens schema i \prettyref{tbl:schema}.
Det är naturligtvis valfritt att följa detta schema sånär som på slutdatum för 
kursens uppgifter och när föreläsningarna ges.
Läsanvisningar för respektive moment följer i kommande avsnitt.
Undervisningen förutsätter att du följer dessa riktlinjer.

\begin{table}
	\centering
  \begin{tabular}{rp{9cm}}
    \toprule
		\textbf{Kursvecka}	& \textbf{Arbete} \\
    \midrule
    1	& Kursstart/Föreläsning introduktion \\
      & Laboration L0 Installation \\
    \midrule
    2 & Föreläsning om datarepresentation \\
      & Föreläsning om datorarkitektur \\
      & Handledning \\
    \midrule
    3 & Föreläsning om den UNIX-lika terminalen \\
      & Övning: den UNIX-lika terminalen \\
      & Laboration L1 Terminalen \\
      & Handledning \\
    \midrule
    4 & Föreläsning om programmering med Python, del 1 \\
      & Övning: Python \\
      & Laboration L2 Programmering med Python \\
      & Föreläsning om programmering med Python, del 2 \\
      & Övning: Python \\
      & Handledning \\
    \midrule
    5 & Föreläsning om LaTeX \\
      & Övning: LaTeX \\
      & Laboration L3 Datorn \\
      & Föreläsning om presentationsteknik \\
      & Handledning \\
    \midrule
    6 & Projekt \\
      & Handledning \\
    \midrule
    %7 & Föreläsning om sidokanaler och annat säkerhetsrelaterat \\
    7 & Handledning \\
    \midrule
    8 & Handledning \\
    \midrule
    9 & Handledning \\
    \midrule
    10  & Redovisning av projekt \\
    \bottomrule
  \end{tabular}
  \caption{En sammanställning av kursens moment och när de kommer att 
  genomföras.
  Tiden är anpassad efter studietakt om halvfart.}
	\label{tbl:schema}
\end{table}

\subsection{Introduktionsföreläsning}
\input{intro-lit.tex}

\subsection{Laboration L0 Installation}
\input{install-lit.tex}

\subsection{Föreläsning om datarepresentation}
\input{datarep-lit.tex}

\subsection{Föreläsning om datorarkitektur}
\input{arch-lit.tex}

\subsection{Föreläsning om den UNIX-lika terminalen}
\input{shell-lit.tex}

\subsection{Laboration L1 Terminalen}
\input{term-lit.tex}

\subsection{Föreläsning om programmering med Python}
\input{python-lit.tex}

\subsection{Laboration L2 Programmering med Python}
\input{pylab-lit.tex}

\subsection{Föreläsning om LaTeX}
\input{tex-lit.tex}

\subsection{Laboration L3 Datorn}
\input{computer-lit.tex}

\subsection{Föreläsning om presentationsteknik}
\input{present-lit.tex}

\subsection{Projektet}
Projektet syftar till att genomföra en mindre studie som fördjupning inom 
området datateknik.
Det tillkommer därför delvis valfri litteratur i detta moment.


\section{Examination}
\label{sec:exam}
\noindent
Kursen examineras med med laborationer och ett projekt, se lydelsen för 
respektive uppgift för detaljer.
Projektet examineras genom en skriftlig rapport och en muntlig presentation.
Alla laborationer och den muntliga presentationen betygsätts \emph{pass (P)} 
för godkänt eller \emph{fail (F)} för underkänt.
Projektrapporten betygsätts A--E för godkänt eller F--Fx för underkänt.


\section{Vad händer om jag ej blir klar i tid?}
\label{sec:late}
\noindent
Slutdatumena på denna kurs är av yttersta vikt.
Du måste ha genomfört introduktionsuppgiften inom dess givna slutdatum, om du 
inte gör detta kommer du att avregistreras från kursen och din plats kommer att 
ställas till förfogande för andra sökande.

Vad gäller den övriga examinationen på kursen kommer det för redovisningar att 
ges ett redovisningstillfälle under kursens gång, detta kommer att vara under 
ordinarie tentamensvecka.
Därefter ges ytterligare två redovisningstillfällen, dessa förläggs i de 
efterföljande omtentamensperioderna.
Alla dessa tillfällen kommer att finnas i kursens schema (i studentportalen).

De slutdatum som finns för dessa tillfällen är strikta.
Om du missar slutdatumet för ett tillfälle hänvisas du till nästa 
redovisningstillfälle.
Efter det tredje redovisningstillfället hänvisas till redovisningstillfällena 
under nästkommande kursomgång.

För skriftliga inlämningsuppgifter gäller att dessa rättas en gång under 
kursens gång, i samband med slutdatum för inlämning, därefter ytterligare två 
gånger i de kommande omtentamensperioderna.
Totalt erbjuds tre försök per år.
Därefter hänvisas till nästa kursomgång.

Ingen handledning kommer att ges efter kursens slut, det vill säga efter det 
sista schemalagda handledningstillfället.
Om du inte hinner bli klar med uppgifterna inom kursens tidsramar och du vill 
ha vidare handledning av lärare krävs att du omregistrerar dig på nästa 
kursomgång.
Omregistrering på kurs sker i mån om plats och ligger lägst i
prioriteringslistan, alla förstagångssökande och reserver kommer att ha förtur.

Om du vid kursslut har majoriteten av kursens moment kvar att göra hänvisas du 
direkt till nästa kursomgång, då krävs omregistrering.
Huruvida din prestation är tillräcklig för att enbart komplettera eller om 
omregistrering krävs bedöms av ansvarig lärare.

Om du känner att du inte kommer att hinna bli klar med kursen är det därför 
bättre att göra ett tidigt avbrott på kursen och söka om den inför nästa 
kurstillfälle.
Tidigt avbrott kan registreras senast tre veckor från kursstart och då kommer 
du att räknas som en förstagångssökande nästa gång du söker kursen.


\printbibliography
\end{document}
