% $Id$
\documentclass[a4paper,logo]{miunart}
\usepackage[utf8]{inputenc}
\usepackage[swedish]{babel}
\usepackage{prettyref,varioref}
\usepackage[hyphens]{url}
\usepackage{hyperref}
\usepackage[today,nofancy]{svninfo}
\usepackage{booktabs}
\usepackage{csquotes}
\usepackage[natbib,style=numeric-comp,maxbibnames=99]{biblatex}
\addbibresource{literature.bib}
\usepackage{listings}
\usepackage[varioref,prettyref,listings]{miunmisc}

\svnInfo $Id$

\title{Den fullständiga studiehandledningen för\\
DT001G Informationsteknologi grundkurs}
\author{
  Daniel Bosk\footnote{%
    Detta verk är tillgängliggjort under licensen Creative Commons 
    Erkännande-DelaLika 2.5 Sverige (CC BY-SA 2.5 SE).
    För att se en sammanfattning och kopia av licenstexten besök URL 
    \url{http://creativecommons.org/licenses/by-sa/2.5/se/}.
  }
}
\date{\svnId}

\begin{document}
\maketitle
\tableofcontents


\section{Mål}
\label{sec:aim}
Kursen utgör en introduktion till datateknik och är förberedande för praktiskt 
inriktade datatekniska utbildningar.
Centrala delar är datorns konstruktion, datorkomponenter, grundläggande begrepp 
och terminologier samt mjukvara.
Kursen fokuserar på praktisk användning av datorer och programvaror.
Kursen ger även en introduktion till rapportskrivande och presentationer.

Mer specifikt, efter genomgången kurs ska du kunna:
\begin{itemize}
    \input{inst-aim.tex}
    \input{term-aim.tex}
    \input{pylab-aim.tex}
    \input{computer-aim.tex}
    \input{project-aim.tex}
\end{itemize}


\section{Kursupplägg}
\label{sec:outline}
Den bok som används som huvudlitteratur på kursen är Brookshears bok 
\citetitle{Brookshear2012csa} \cite{Brookshear2012csa}.
Därutöver tillkommer litteratur för en del av laborationerna, denna framgår av 
läsanvisningarna nedan och i respektive laboration.

\citet{Brookshear2012csa} tar upp grunden inom datateknik, den är värd att läsa 
i sin helhet -- detta rekommenderas!
De kapitel och avsnitt som behandlas i denna kurs är dock enbart kapitlen 0--5 
i sina helheter, avsnitten 6.1--6.4, samt avsnitten 9.1 och 9.2.

Det går att använda tidigare upplagor av boken, kapitlen som behandlas i kursen 
utgår från den senaste upplagan \citep{Brookshear2012csa} och de behandlar 
följande områden:
\begin{itemize}
  \item Kapitel 0: introduktion och historia,
  \item kapitel 1: datalagring och -representation,
  \item kapitel 2: datamanipulering, datorarkitektur, programexekvering,
  \item kapitel 3: operativsystem,
  \item kapitel 5: algoritmer,
  \item avsnitten 6.1--6.4: programspråk och programmering, samt
  \item avsnitten 9.1 och 9.2: databaser.
\end{itemize}

Litteraturläsningen kompletteras av ett antal föreläsningar och övningar som 
ges under kursens gång.
Dessa kommer inte att vara heltäckande och för att förstå dem måste du läsa 
litteraturen enligt anvisningarna nedan.

Kursens lärandemål kommer att examineras med ett antal laborationer och 
avslutas med ett mindre projekt.
Detta projekt motsvaras av strax över två veckors heltidsarbete.
Den som nyligen kommer från gymnasieskolan kan ha som referens att detta 
motsvarar en något större omfattning än projektarbetet för gymnasieskolan, dock 
med högre krav på innehållet.

\subsection{Schema}
\label{sec:schedule}
Du finner en sammanställning av kursens schema i \prettyref{tbl:schema}.
Det är naturligtvis valfritt att följa detta schema sånär som på slutdatum för 
kursens uppgifter och när föreläsningarna ges.
Läsanvisningar för respektive moment följer i kommande avsnitt.
Undervisningen förutsätter att du följer dessa riktlinjer.
Det rekommenderas att du läser igenom materialet innan de lärarledda 
tillfällena, så att du kan ta upp dina frågor med läraren.

\begin{table}
  \centering
  \begin{tabular}{cp{9cm}}
    \textbf{Kursvecka}	& \textbf{Arbete} \\
    \toprule
    1
    & Kursstart/Föreläsning introduktion \\
    & L0 Installation \\
    \midrule
    2
    & Föreläsning om datarepresentation \\
    & Föreläsning om datorarkitektur \\
    \midrule
    3
    & Föreläsning om den UNIX-lika terminalen \\
    & Övning: den UNIX-lika terminalen \\
    & L1 Terminalen \\
    & Handledning \\
    \midrule
    4
    & Föreläsning om programmering med Python, del 1 \\
    & Övning: Python \\
    & Handledning \\
    \midrule
    5
    & Föreläsning om programmering med Python, del 2 \\
    & Övning: Python \\
    & L2 Programmering med Python \\
    & Handledning \\
    \midrule
    6
    & Handledning \\
    \midrule
    7
    & Föreläsning om LaTeX \\
    & Övning: LaTeX \\
    & Föreläsning om rapportskrivning och presentationsteknik \\
    & L3 Datorn \\
    & L4 Presentationsteknik \\
    & Handledning \\
    \midrule
    8
    & Handledning \\
    \midrule
    9
    & Handledning \\
    & Presentation L4 \\
    \midrule
    10
    & Tentamen \\
    \bottomrule
  \end{tabular}
  \caption{En sammanställning av kursens moment och när de kommer att 
    genomföras.
  Tiden är anpassad efter studietakt om halvfart.}
  \label{tbl:schema}
\end{table}

\subsection{Introduktionsföreläsning}
\input{intro-lit.tex}

\subsection{Laboration L0 Installation}
\input{inst-lit.tex}

\subsection{Föreläsning om datarepresentation}
\input{datarep-lit.tex}

\subsection{Föreläsning om datorarkitektur}
\input{arch-lit.tex}

\subsection{Föreläsning om den UNIX-lika terminalen}
\input{shell-lit.tex}

\subsection{Laboration L1 Terminalen}
\input{term-lit.tex}

\subsection{Föreläsningar om programmering med Python}
\input{python-lit.tex}

\subsection{Laboration L2 Programmering med Python}
\input{pylab-lit.tex}

\subsection{Föreläsning om \LaTeX}
\input{tex-lit.tex}

\subsection{Föreläsning om rapportskrivning och presentationsteknik}
\input{report-lit.tex}

\subsection{Laboration L3 Datorn}
\input{computer-lit.tex}

\subsection{Laboration L4 Presentationsteknik}
\input{present-lit.tex}

\subsection{Tentamen}
Tentamen i denna kurs examinerar den teori som behandlats, det vill säga 
samtlig litteratur som behandlats under kursens gång.


\section{Examination}
\label{sec:exam}
Kursen examineras med med inlämningsuppgifter (laborationer), se lydelserna för 
detaljer, och en skriftlig salstentamen.

Den första uppgiften L0 rapporteras in som moment I010 i ladok.
Övriga laborationer L\{1,2,3\} rapporteras in gemensamt som moment I110 
i ladok, det vill säga samtliga måste vara godkända innan detta kan ske, och 
detta ger 3.5 högskolepoäng.
Alla dessa uppgifter betygsätts godkänt eller underkänt.

Den muntliga presentationen rapporteras in som P110, detta ger en (1) 
högskolepoäng, och betygsätts godkänt eller underkänt.

Den avslutande salstentamen betygsätts med hela betygskalan, A-E för godkänt 
och F eller Fx för underkänt.
Tentamen rapporteras som moment T110 i ladok och ger tre (3) högskolepoäng.


\section{Vad händer om jag ej blir klar i tid?}
\label{sec:late}
Slutdatumena på denna kurs är av yttersta vikt.
Du måste ha genomfört introduktionsuppgiften L0 inom dess givna slutdatum, om 
du inte gör detta kommer du att avregistreras från kursen och din plats kommer 
att ställas till förfogande för andra sökande.

Vad gäller den övriga examinationen på kursen kommer det för redovisningar att 
ges ett presentationstillfälle under kursens gång.
Därefter ges ytterligare två presentationstillfällen, dessa förläggs inom ett 
år.
Alla dessa tillfällen kommer att finnas i kursens schema (i studentportalen).

De slutdatum som finns för dessa tillfällen är strikta.
Om du missar slutdatumet för ett tillfälle hänvisas du till nästa 
redovisningstillfälle.
Efter det tredje redovisningstillfället hänvisas till redovisningstillfällena 
under nästkommande kursomgång.

För skriftliga inlämningsuppgifter gäller att dessa rättas en gång under 
kursens gång, senast i samband med slutdatum för inlämning, därefter 
ytterligare två gånger i de kommande omtentamensperioderna.
Totalt erbjuds tre försök per år.
Därefter hänvisas till nästa kursomgång.

För skriftlig salstentamen gäller att du måste anmäla dig i förväg.
Du kan ansöka om att skriva tentamen på annan ort, detta måste du dock ansöka 
om i god tid och du måste ordna plats själv.
Tentamen skrivs endast under de tider som finns i schemat, inga undantag kan 
göras.
Se instruktionerna i studentportalen för vidare detaljer.

Ingen handledning planeras efter kursens slut, det vill säga efter det sista 
schemalagda handledningstillfället.
Om du inte hinner bli klar med uppgifterna inom kursens tidsramar och du vill 
vara garanterad handledning av lärare krävs att du omregistrerar dig på nästa 
kurstillfälle.
Omregistrering på kurstillfälle sker i mån om plats, alla förstagångssökande 
och reserver kommer att prioriteras.

Om du vid kursslut har majoriteten av kursens moment kvar att göra hänvisas du 
direkt till nästa kursomgång, då krävs omregistrering.
Huruvida din prestation är tillräcklig för att enbart komplettera eller om 
omregistrering krävs bedöms av ansvarig lärare.

Om du känner att du inte kommer att hinna bli klar med kursen är det därför 
bättre att göra ett tidigt avbrott på kursen och söka om den inför nästa 
kurstillfälle.
Tidigt avbrott kan registreras senast tre veckor från kursstart och då kommer 
du att räknas som en förstagångssökande nästa gång du söker kursen.


\printbibliography
\end{document}
