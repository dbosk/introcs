\section{Genomf�rande}
\label{sec:Instructions}
\noindent
Anv�nd valfritt operativsystem.
Du beh�ver ha tillg�ng till kommandotolk, eller terminalf�nster, en
grafisk filhanterare samt en valfri webbl�sare.
\begin{description}
    \item[Windows XP] K�r kommandotolken genom att v�lja \emph{K�r} fr�n
        Startmenyn, skriv \emph{cmd.exe} och v�lj \emph{OK}.
        Anv�nd \emph{Den h�r datorn} eller \emph{Utforskaren} som grafisk
        filhanterare.
    \item[Windows Vista/7] S�k efter \emph{kommandotolken} eller
        \emph{powershell} i Startmenyns s�kfunktion.
        Anv�nd \emph{Dator} eller \emph{Utforskaren} som grafisk filhanterare.
    \item[UNIX] K�r valfri terminal, exempelvis \emph{xterm}, med valfri shell,
        exempelvis \emph{sh}.
        Anv�nder du Ubuntu kan du anv�nda programmet \emph{Terminal}
        (gnome-terminal).
        Grafisk filhanterare varierar beroende p� system: Nautilus i Gnome,
        Konqueror i KDE.
\end{description}

F�r dokumentation om kommandona i Windows kan f�ljande sida anv�ndas:
\begin{quote}
	\url{http://technet.microsoft.com/en-us/library/bb490890.aspx}
\end{quote}
F�r dokumentation om kommandon i UNIX anv�nds kommandot \emph{man}.
Det tar kommandot du vill ha dokumentation f�r som argument.
Om vi till exempel vill ha dokumentation om just \emph{man} skriver vi
 \emph{man man}.
\begin{lstlisting}
/home/danbos$ man man
MAN(1)                        Manual pager utils                        MAN(1)

NAME
       man - an interface to the on-line reference manuals

SYNOPSIS
       man  [-C  file]  [-d]  [-D]  [--warnings[=warnings]]  [-R encoding] [-L
       locale] [-m system[,...]] [-M path] [-S list]  [-e  extension]  [-i|-I]
       [--regex|--wildcard]   [--names-only]  [-a]  [-u]  [--no-subpages]  [-P
       pager] [-r prompt] [-7] [-E encoding] [--no-hyphenation] [--no-justifi-
       cation]  [-p  string]  [-t]  [-T[device]]  [-H[browser]] [-X[dpi]] [-Z]
       [[section] page ...] ...
       man -k [apropos options] regexp ...
       man -K [-w|-W] [-S list] [-i|-I] [--regex] [section] term ...
       man -f [whatis options] page ...
       man -l [-C file] [-d] [-D] [--warnings[=warnings]]  [-R  encoding]  [-L
       locale]  [-P  pager]  [-r  prompt]  [-7] [-E encoding] [-p string] [-t]
       [-T[device]] [-H[browser]] [-X[dpi]] [-Z] file ...
       man -w|-W [-C file] [-d] [-D] page ...
       man -c [-C file] [-d] [-D] page ...
       man [-hV]

DESCRIPTION
       man is the system's manual pager. Each page argument given  to  man  is
       normally  the  name of a program, utility or function.  The manual page
       associated with each of these arguments is then found and displayed.  A
       section,  if  provided, will direct man to look only in that section of
       the manual.  The default action is to search in all  of  the  available
       sections, following a pre-defined order and to show only the first page
       found, even if page exists in several sections.

       The table below shows the section numbers of the manual followed by the
       types of pages they contain.
...
/home/danbos$
\end{lstlisting}
